%**************************************%
%* Generated from MathBook XML source *%
%*    on 2017-05-25T11:33:49-07:00    *%
%*                                    *%
%*   http://mathbook.pugetsound.edu   *%
%*                                    *%
%**************************************%
\documentclass[10pt,]{book}
%% Custom Preamble Entries, early (use latex.preamble.early)
%% Inline math delimiters, \(, \), need to be robust
%% 2016-01-31:  latexrelease.sty  supersedes  fixltx2e.sty
%% If  latexrelease.sty  exists, bugfix is in kernel
%% If not, bugfix is in  fixltx2e.sty
%% See:  https://tug.org/TUGboat/tb36-3/tb114ltnews22.pdf
%% and read "Fewer fragile commands" in distribution's  latexchanges.pdf
\IfFileExists{latexrelease.sty}{}{\usepackage{fixltx2e}}
%% Text height identically 9 inches, text width varies on point size
%% See Bringhurst 2.1.1 on measure for recommendations
%% 75 characters per line (count spaces, punctuation) is target
%% which is the upper limit of Bringhurst's recommendations
%% Load geometry package to allow page margin adjustments
\usepackage{geometry}
\geometry{letterpaper,total={340pt,9.0in}}
%% Custom Page Layout Adjustments (use latex.geometry)
\geometry{papersize={6in,9in}, hmargin={0.85in, 0.5in}, height=7.75in, top=0.75in, twoside, ignoreheadfoot}
%% This LaTeX file may be compiled with pdflatex, xelatex, or lualatex
%% The following provides engine-specific capabilities
%% Generally, xelatex and lualatex will do better languages other than US English
%% You can pick from the conditional if you will only ever use one engine
\usepackage{ifthen}
\usepackage{ifxetex,ifluatex}
\ifthenelse{\boolean{xetex} \or \boolean{luatex}}{%
%% begin: xelatex and lualatex-specific configuration
%% fontspec package will make Latin Modern (lmodern) the default font
\ifxetex\usepackage{xltxtra}\fi
\usepackage{fontspec}
%% realscripts is the only part of xltxtra relevant to lualatex 
\ifluatex\usepackage{realscripts}\fi
%% 
%% Extensive support for other languages
\usepackage{polyglossia}
%% Main document language is US English
\setdefaultlanguage{english}
%% Spanish
\setotherlanguage{spanish}
%% Vietnamese
\setotherlanguage{vietnamese}
%% end: xelatex and lualatex-specific configuration
}{%
%% begin: pdflatex-specific configuration
%% translate common Unicode to their LaTeX equivalents
%% Also, fontenc with T1 makes CM-Super the default font
%% (\input{ix-utf8enc.dfu} from the "inputenx" package is possible addition (broken?)
\usepackage[T1]{fontenc}
\usepackage[utf8]{inputenc}
%% end: pdflatex-specific configuration
}
%% Symbols, align environment, bracket-matrix
\usepackage{amsmath}
\usepackage{amssymb}
%% allow page breaks within display mathematics anywhere
%% level 4 is maximally permissive
%% this is exactly the opposite of AMSmath package philosophy
%% there are per-display, and per-equation options to control this
%% split, aligned, gathered, and alignedat are not affected
\allowdisplaybreaks[4]
%% allow more columns to a matrix
%% can make this even bigger by overriding with  latex.preamble.late  processing option
\setcounter{MaxMatrixCols}{30}
%%
%% Color support, xcolor package
%% Always loaded.  Used for:
%% mdframed boxes, add/delete text, author tools
\PassOptionsToPackage{usenames,dvipsnames,svgnames,table}{xcolor}
\usepackage{xcolor}
%%
%% Semantic Macros
%% To preserve meaning in a LaTeX file
%% Only defined here if required in this document
%% Subdivision Numbering, Chapters, Sections, Subsections, etc
%% Subdivision numbers may be turned off at some level ("depth")
%% A section *always* has depth 1, contrary to us counting from the document root
%% The latex default is 3.  If a larger number is present here, then
%% removing this command may make some cross-references ambiguous
%% The precursor variable $numbering-maxlevel is checked for consistency in the common XSL file
\setcounter{secnumdepth}{1}
%% Environments with amsthm package
%% Theorem-like environments in "plain" style, with or without proof
\usepackage{amsthm}
\theoremstyle{plain}
%% Numbering for Theorems, Conjectures, Examples, Figures, etc
%% Controlled by  numbering.theorems.level  processing parameter
%% Always need a theorem environment to set base numbering scheme
%% even if document has no theorems (but has other environments)
\newtheorem{theorem}{Theorem}[section]
%% Only variants actually used in document appear here
%% Style is like a theorem, and for statements without proofs
%% Numbering: all theorem-like numbered consecutively
%% i.e. Corollary 4.3 follows Theorem 4.2
%% Localize LaTeX supplied names (possibly none)
\renewcommand*{\chaptername}{Chapter}
%% For improved tables
\usepackage{array}
%% Some extra height on each row is desirable, especially with horizontal rules
%% Increment determined experimentally
\setlength{\extrarowheight}{0.2ex}
%% Define variable thickness horizontal rules, full and partial
%% Thicknesses are 0.03, 0.05, 0.08 in the  booktabs  package
\makeatletter
\newcommand{\hrulethin}  {\noalign{\hrule height 0.04em}}
\newcommand{\hrulemedium}{\noalign{\hrule height 0.07em}}
\newcommand{\hrulethick} {\noalign{\hrule height 0.11em}}
%% We preserve a copy of the \setlength package before other
%% packages (extpfeil) get a chance to load packages that redefine it
\let\oldsetlength\setlength
\newlength{\Oldarrayrulewidth}
\newcommand{\crulethin}[1]%
{\noalign{\global\oldsetlength{\Oldarrayrulewidth}{\arrayrulewidth}}%
\noalign{\global\oldsetlength{\arrayrulewidth}{0.04em}}\cline{#1}%
\noalign{\global\oldsetlength{\arrayrulewidth}{\Oldarrayrulewidth}}}%
\newcommand{\crulemedium}[1]%
{\noalign{\global\oldsetlength{\Oldarrayrulewidth}{\arrayrulewidth}}%
\noalign{\global\oldsetlength{\arrayrulewidth}{0.07em}}\cline{#1}%
\noalign{\global\oldsetlength{\arrayrulewidth}{\Oldarrayrulewidth}}}
\newcommand{\crulethick}[1]%
{\noalign{\global\oldsetlength{\Oldarrayrulewidth}{\arrayrulewidth}}%
\noalign{\global\oldsetlength{\arrayrulewidth}{0.11em}}\cline{#1}%
\noalign{\global\oldsetlength{\arrayrulewidth}{\Oldarrayrulewidth}}}
%% Single letter column specifiers defined via array package
\newcolumntype{A}{!{\vrule width 0.04em}}
\newcolumntype{B}{!{\vrule width 0.07em}}
\newcolumntype{C}{!{\vrule width 0.11em}}
\makeatother
%% Figures, Tables, Listings, Floats
%% The [H]ere option of the float package fixes floats in-place,
%% in deference to web usage, where floats are totally irrelevant
%% We re/define the figure, table and listing environments, if used
%%   1) New mbxfigure and/or mbxtable environments are defined with float package
%%   2) Standard LaTeX environments redefined to use new environments
%%   3) Standard LaTeX environments redefined to step theorem counter
%%   4) Counter for new environments is set to the theorem counter before caption
%% You can remove all this figure/table setup, to restore standard LaTeX behavior
%% HOWEVER, numbering of figures/tables AND theorems/examples/remarks, etc
%% WILL ALL de-synchronize with the numbering in the HTML version
%% You can remove the [H] argument of the \newfloat command, to allow flotation and 
%% preserve numbering, BUT the numbering may then appear "out-of-order"
\usepackage{float}
\usepackage[bf]{caption} % http://tex.stackexchange.com/questions/95631/defining-a-new-type-of-floating-environment 
\usepackage{newfloat}
% Figure environment setup so that it no longer floats
\SetupFloatingEnvironment{figure}{fileext=lof,placement={H},within=section,name=Figure}
% figures have the same number as theorems: http://tex.stackexchange.com/questions/16195/how-to-make-equations-figures-and-theorems-use-the-same-numbering-scheme 
\makeatletter
\let\c@figure\c@theorem
\makeatother
% Table environment setup so that it no longer floats
\SetupFloatingEnvironment{table}{fileext=lot,placement={H},within=section,name=Table}
% tables have the same number as theorems: http://tex.stackexchange.com/questions/16195/how-to-make-equations-figures-and-theorems-use-the-same-numbering-scheme 
\makeatletter
\let\c@table\c@theorem
\makeatother
%% Raster graphics inclusion, wrapped figures in paragraphs
%% \resizebox sometimes used for images in side-by-side layout
\usepackage{graphicx}
%%
%% More flexible list management, esp. for references and exercises
%% But also for specifying labels (i.e. custom order) on nested lists
\usepackage{enumitem}
%% Lists of references in their own section, maximum depth 1
\newlist{referencelist}{description}{4}
\setlist[referencelist]{leftmargin=!,labelwidth=!,labelsep=0ex,itemsep=1.0ex,topsep=1.0ex,partopsep=0pt,parsep=0pt}
%% Support for index creation
%% imakeidx package does not require extra pass (as with makeidx)
%% Title of the "Index" section set via a keyword
%% Language support for the "see" and "see also" phrases
\usepackage{imakeidx}
\makeindex[title=Index, intoc=true]
\renewcommand{\seename}{see}
\renewcommand{\alsoname}{see also}
%% hyperref driver does not need to be specified, it will be detected
\usepackage{hyperref}
%% Hyperlinking active in PDFs, all links solid and blue
\hypersetup{colorlinks=true,linkcolor=blue,citecolor=blue,filecolor=blue,urlcolor=blue}
\hypersetup{pdftitle={Combinatorics Through Guided Discovery}}
%% If you manually remove hyperref, leave in this next command
\providecommand\phantomsection{}
%% Graphics Preamble Entries
\usepackage{tikz}
\usepackage{tkz-graph}
\usepackage{tkz-euclide}
\usetikzlibrary{patterns}
\usetikzlibrary{positioning}
\usetikzlibrary{matrix,arrows}
\usetikzlibrary{calc}
\usetikzlibrary{shapes}
\usetikzlibrary{through,intersections,decorations,shadows,fadings}

\usepackage{pgfplots}
%% If tikz has been loaded, replace ampersand with \amp macro
%% extpfeil package for certain extensible arrows,
%% as also provided by MathJax extension of the same name
%% NB: this package loads mtools, which loads calc, which redefines
%%     \setlength, so it can be removed if it seems to be in the 
%%     way and your math does not use:
%%     
%%     \xtwoheadrightarrow, \xtwoheadleftarrow, \xmapsto, \xlongequal, \xtofrom
%%     
%%     we have had to be extra careful with variable thickness
%%     lines in tables, and so also load this package late
\usepackage{extpfeil}
%% Custom Preamble Entries, late (use latex.preamble.late)
%% Begin: Author-provided packages
%% (From  docinfo/latex-preamble/package  elements)
%% End: Author-provided packages
%% Begin: Author-provided macros
%% (From  docinfo/macros  element)
%% Plus three from MBX for XML characters
\newcommand{\cycle}[1]{\arraycolsep 5 pt
\left(\begin{array}#1\end{array}\right)}
\newcommand{\iteme}{
\item[\ $\bullet$\ \theenumi.]}
\newcommand{\items}{
\item[\ \tiny$+$\ \theenumi.]}
\newcommand{\itemh}{
\item[\ {$*$}\ \theenumi.]}
\newcommand{\itemes}{
\item[\ \Large$\cdot$\ \theenumi.]}
\newcommand{\itemesi}{
 \item[\ \importantarrow\ \Large
$\cdot$\ \theenumi.]}
\newcommand{\itemm}{
\item[\ $\circ$\ \theenumi.]}
\newcommand{\importantarrow}{\psfig{figure=arrow.eps}}
\newcommand{\itemi}{
\item[\ \importantarrow\ \theenumi.]}
\newcommand{\itemitemi}{
\item[\ \importantarrow\ (\theenumii)]}
\newcommand{\itemei}{
\item[\ \importantarrow\ $\bullet$\ \theenumi.]}
\newcommand{\itemih}{
\item[\ \importantarrow\ $*$\ \theenumi.]}
\newcommand{\itemitemih}{
\item[\ \importantarrow\ $*$\ (\theenumii)]}
\newcommand{\itemitemh}{
\item[\ $*$\ (\theenumii)]}
\newcommand{\qchoose}[2]{\left[{#1\atop#2}\right]_q}
\def\neg1choose#1#2{\left[{#1\atop#2}\right]_{-1}}
\newcommand{\bp}{
\begin{enumerate}{\setcounter{enumi}{\value{problemnumber}}}}
\newcommand{\ep}{\setcounter{problemnumber}{\value{enumi}}
\end{enumerate}}
\newcommand{\ignore}[1]{}
\renewcommand{\bottomfraction}{.8}
\renewcommand{\topfraction}{.8}
\newcommand{\apple}{\mbox{\psfig{figure=apple.eps,height=10 pt}}}
\newcommand{\ap}{\apple}
\newcommand{\banana}{\mbox{\psfig{figure=Banana.eps,height=10 pt}}}
\newcommand{\ba}{\banana}
\newcommand{\pear}{\mbox{\psfig{figure=Pear.eps,height=10 pt}}}
\newcommand{\pe}{\pear}
\newcommand{\lt}{<}
\newcommand{\gt}{>}
\newcommand{\amp}{&}
%% End: Author-provided macros
%% Title page information for book
\title{Combinatorics Through Guided Discovery}
\author{Kenneth P. Bogart
}
\date{}
\begin{document}
\frontmatter
%% no half-title
%% No adcard
%% begin: title page
%% Inspired by Peter Wilson's "titleDB" in "titlepages" CTAN package
\thispagestyle{empty}
{\centering
\vspace*{0.14\textheight}
%% Target for xref to top-level element is ToC
\addtocontents{toc}{\protect\hypertarget{index}{}}
{\Huge Combinatorics Through Guided Discovery}\\[3\baselineskip]
{\Large Kenneth P. Bogart}\\}
\clearpage
%% end:   title page
%% begin: copyright-page
\thispagestyle{empty}
\vspace*{\stretch{2}}
\vspace*{\stretch{1}}
\null\clearpage
%% end:   copyright-page
%% begin: table of contents
%% Adjust Table of Contents
\setcounter{tocdepth}{2}
\renewcommand*\contentsname{Contents}
\tableofcontents
%% end:   table of contents
\mainmatter
\typeout{************************************************}
\typeout{Chapter 0 What is Combinatorics?}
\typeout{************************************************}
\chapter[{What is Combinatorics?}]{What is Combinatorics?}\label{chapter-1}
\typeout{************************************************}
\typeout{Introduction  }
\typeout{************************************************}
Combinatorial mathematics arises from studying how we can \emph{combine} objects into arrangements. For example, we might be combining sports teams into a tournament, samples of tires into plans to mount them on cars for testing, students into classes to compare approaches to teaching a subject, or members of a tennis club into pairs to play tennis. There are many questions one can ask about such arrangements of objects. Here we will focus on questions about \emph{how many ways} we may combine the objects into arrangements of the desired type. These are called \emph{counting problems}. Sometimes, though, combinatorial mathematicians ask if an arrangement is possible (if we have ten baseball teams, and each team has to play each other team once, can we schedule all the games if we only have the fields available at enough times for forty games?). Sometimes they ask if all the arrangements we might be able to make have a certain desirable property (Do all ways of testing 5 brands of tires on 5 different cars [with certain additional properties] compare each brand with each other brand on at least one common car?). Problems of these sorts come up throughout physics, biology, computer science, statistics, and many other subjects. However, to demonstrate all these relationships, we would have to take detours into all these subjects. While we will give some important applications, we will usually phrase our discussions around everyday experience and mathematical experience so that the student does not have to learn a new context before learning mathematics in context!%
\typeout{************************************************}
\typeout{Section 0.1 About These Notes}
\typeout{************************************************}
\section[{About These Notes}]{About These Notes}\label{section-1}
These notes are based on the philosophy that you learn the most about a subject when you are figuring it out directly for yourself, and learn the least when you are trying to figure out what someone else is saying about it. On the other hand, there is a subject called combinatorial mathematics, and that is what we are going to be studying, so we will have to tell you some basic facts. What we are going to try to do is to give you a chance to discover many of the interesting examples that usually appear as textbook examples and discover the principles that appear as textbook theorems. Your main activity will be solving problems designed to lead you to discover the basic principles of combinatorial mathematics. Some of the problems lead you through a new idea, some give you a chance to describe what you have learned in a sequence of problems, and some are quite challenging. When you find a problem challenging, don't give up on it, but don't let it stop you from going on with other problems. Frequently you will find an idea in a later problem that you can take back to the one you skipped over or only partly finished in order to finish it off. With that in mind, let's get started. In the problems that follow, you will see some problems marked on the left with various symbols. The preface gives a full explanation of these symbols and discusses in greater detail why the book is organized as it is! \hyperref[tab_prob-symbs]{Table~\ref{tab_prob-symbs}}, which is repeated from the preface, summarizes the meaning of the symbols.%
\begin{table}
\centering
\begin{tabular}{ll}
&\tabularnewline\hrulethin
\(\bullet\)&essential\tabularnewline[0pt]
\(\circ\)&motivational material\tabularnewline[0pt]
\(+\)&summary\tabularnewline[0pt]
\importantarrow&especially interesting\tabularnewline[0pt]
\(*\)&difficult\tabularnewline[0pt]
\(\cdot\)&essential for this section or the next\tabularnewline[0pt]
&\tabularnewline\hrulethin
\end{tabular}
\caption{The meaning of the symbols to the left of problem numbers.\label{tab_prob-symbs}}
\end{table}
\typeout{************************************************}
\typeout{Section 0.2 Supplementary Chapter Problems}
\typeout{************************************************}
\section[{Supplementary Chapter Problems}]{Supplementary Chapter Problems}\label{section-2}
\leavevmode%
\begin{enumerate}
\item\hypertarget{compositiondefinition}{}(interesting) Remember that we can write \(n\) as a sum of \(n\) ones.  How many plus signs do we use?  In how many ways may we write \(n\) as a sum of a list of \(k\) positive numbers?  Such a list is called a \emph{composition}\index{composition} of \(n\) into \(k\) parts.\index{composition!\(k\) parts}\index{composition!\(k\) parts!number of} We use \(n-1\) plus signs. Write down such a sum and choose \(k-1\) of the plus signs. Then each string of ones and plusses between two chosen plus signs, before the first chosen plus sign or after the last chosen one corresponds to a part of a composition of \(n\). Thus the number of compositions of \(n\) with \(k\) parts is the number of ways to choose the \(k-1\) places, which is \(n-1\choose k-1\).%
%
\item\hypertarget{composition_numberof}{}In \hyperlink{compositiondefinition}{Problem~1} we defined a composition of \(n\) into \(k\) parts.  What is the total number of compositions of \(n\) (into any number of parts). \index{compositions!number of} The total number of compositions is the number of ways to choose a subset of the plus signs which is \(2^{n-1}\).%
%
\item\hypertarget{GreyCode}{}(essential for this or the next section) Write down a list of all 16 0-1 sequences of length four starting with 0000 in such a way that each entry differs from the precious one by changing just one digit.  This is called a Grey Code.\index{Grey Code}  That is, a \emph{Grey Code} for 0-1 sequences of length \(n\) is a list of the sequences so that each entry differs from the previous one in exactly one place.  Can you describe how to get a Grey Code for 0-1 sequences of length five from the one you found for sequences of length 4?  Can you describe how to prove that there is a Grey code for sequences of length \(n\)? (One of many) 0000, 0001, 0011, 0010, 0110, 0111, 0101, 0100, 1100, 1101, 1111, 1110, 1010, 1011, 1001, 1000. To get a code for sequences of length 5, put a zero at the end of each of the sequences we have. Follow that revised sequence by 10001, and write the remainder of the sequence in reverse order with a 1 at the end of each term. (Don't reverse the individual length four sequences, just the sequence of sequences!) We just, in essence, described the inductive step of an inductive proof that Grey Codes exist for sequences of any length.%
%
\item\hypertarget{li-4}{}(interesting) Use the idea of a Grey Code from \hyperlink{GreyCode}{Problem~3} to prove bijectively that the number of even-sized subsets of an \(n\)-element set equals the number of odd-sized subsets of an \(n\)-element set. Each sequence in the Grey Code is the characteristic function of a set, and the number of elements of the set is the number of ones in the sequence. Since each sequence differs in just one place from the preceding one, the sequences alternate between having an even number of ones and an odd number of ones. Since the first sequence is all zeros and there are \(2^n\) sequences, the last one has an odd number of zeros. Thus the map that takes each sequence except the last to the next one, and takes the last to the first is a bijection between the characteristic functions of sets with an even number of elements and sets with an odd number of elements.%
%
\item\hypertarget{li-5}{}(interesting) A list of parentheses is said to be balanced if there are the same number of left parentheses as right, and as we count from left to right we always find at least as many left parentheses as right parentheses.  For example, (((()()))()) is balanced and ((()) and (()()))(() are not.  How many balanced lists of \(n\) left and \(n\)  right parentheses are there? The number is the Catalan number: we get a bijection between balanced lists of parentheses and Catalan paths by sending each left paren to an upstep and each right paren to a downstep. The condition that there are always as many left parentheses as right ensures we never go below the \(x\) axis.%
%
\item\hypertarget{li-6}{}(difficult) Suppose we plan to put six distinct computers in a network as shown in \hyperref[hexagonalnetwork]{Figure~\ref{hexagonalnetwork}}.  The lines show which computers can communicate directly with which others.  Consider two ways of assigning computers to the nodes of the network different if there are two computers that communicate directly in one assignment and that don't communicate directly in the other.  In how many different ways can we assign computers to the network? \begin{figure}
\centering
\includegraphics[width=0.73\linewidth]{images/}
\caption{A computer network.\label{hexagonalnetwork}}
\end{figure}
 We consider two assignments of computers to be equivalent if in both assignments, each computer communicates directly with exactly the same computers. This partitions the set of all \(6!\) computer assignments into blocks of \(48\) computers each. Thus we have \(720/48=15\) ways to assign the computers to the network.%
%
\item\hypertarget{li-7}{}(interesting) In a circular ice cream dish we are going to put four distinct scoops of ice cream chosen from among twelve flavors.  Assuming we place four scoops of the same size as if they were at the corners of a square, and recognizing that moving the dish doesn't change the way in which we have put the ice cream into the dish, in how many ways may we choose the ice cream and put it into the dish? Each ice cream arrangement is equivalent to three others, the ones we get by rotating the dish. This divides the arrangements of four flavors of ice cream into blocks of size 4. Thus we may arrange the ice cream we have chosen in the dish in \(4!/4=6\) ways. We may choose the ice cream in \({12\choose 4}=495\) ways, and so we may choose it and put it into the dish in 2970 ways.%
%
\item\hypertarget{li-8}{}(interesting) In as many ways as you can, show that \({n\choose k}{n-k\choose m} =
{n\choose m}{n-m\choose k}\). You can prove this by plugging in the formula for \(n\choose k\) on both sides and cancelling stuff until you get the same thing on both sides. However a much more interesting proof is that the right hand side counts the number of ways to choose a \(k\)-element set form an \(n\)-element set and then choose an \(m\)-element set from what remains. The left hand side counts the number of ways to first chose a \(k\)-element subset from the \(n\)-element set and then choose an \(m\)-element subset from what remains. Thus in both cases you are counting the number of ways to choose an ordered pair consisting of an \(m\)-element subset and a disjoint \(k\)-element subset from an \(n\)-element set. You can also base a proof on the observation that \((x+y+x)^n=
\sum_{k=0}^n{n\choose k}(x+y)^kz^{n-k}\) and \((x+y+z)^n=\sum_{m=0}^n{n\choose m}x^m(y+z)^{n-m}\) and asking for the coefficient of \(x^my^{n-m-k}z^k\). You do have to use the binomial theorem with an eye to the result you are looking for, however.%
%
\item\hypertarget{li-9}{}(interesting) A tennis club has \(4n\) members.  To specify a doubles match, we choose two teams of two people.  In how many ways may we arrange the members into doubles matches so that each player is in one doubles match?  In how many ways may we do it if we specify in addition who serves first on each team? We now have many methods for solving this problem. Perhaps the easiest is to list all \((4n)\) people and take them in groups of four for doubles matches, with the first two in a group of four as one team and the second two as another team. We note that interchanging the \(n\) blocks of 4 does not change the matches, nor does interchanging the two people on a team nor interchanging the two teams. Thus we have \((4n)!/n!2^{3n}\) ways to arrange the matches. If we are to say who serves first on each team, we might as well say it is the first of the two listed, so now we have \((4n)!/n!2^n\) ways to arrange the matches.%
%
\item\hypertarget{li-10}{}A town has \(n\) streetlights running along the north side of main street.  The poles on which they are mounted need to be painted so that they do not rust.  In how many ways may they be painted with red, white, blue, and green if an even number of them are to be painted green? We can think of first choosing the set of even size of poles to be painted green, and the painting the remaining poles red, white, and blue. We may do this in \(\sum_{k=0}^{\lfloor n/2\rfloor}{n\choose 2k}3^{n-2k}\) ways.%
%
\item\hypertarget{pingpongpaint}{}We have \(n\) identical ping-pong balls.  In how many ways may we paint them red, white, blue, and green? We can line up the identical ping-pong balls and break them into four groups, those of each color, by inserting dividers. If we want to paint at least one in each color, we can choose three of the spaces between the balls in which to insert dividers, so we can paint them in \(n-1\choose k\). But the problem didn't require us to use each color, so we can put two dividers adjacent to each other. Thus there are \(n+1\) places where we can put the first divider (putting it before all the balls means we use no red, and putting it after all of them means we use no green. Now there are \(n+2\) places where we can put the second divider, including before or after the first, and \(n+3\) places where we can put the third divider. However if we interchange two dividers we still paint the balls before the first divider red, those between then next two white, and so on. Thus \(3!=6\) of these arrangements of balls and dividers correspond to the same paint job, so the number of ways to paint the balls is \({(n+1)(n+2)(n+3)\over6} ={n+3\choose
3}\). This suggests that another way to think of the problem is to consider \(n+3\) slots in a row, and fill \(n\) of them with balls and \(3\) of them with dividers; since the balls are identical and the dividers might as well be identical, the number of ways to do this is the number of ways to choose the slots that get dividers.%
%
\item\hypertarget{li-12}{}We have \(n\) identical ping-pong balls.  In how many ways may we paint them red, white, blue, and green if we use green paint on an even number of them? We first decide how many balls to paint green, then paint the remainder with the other three colors as in \hyperlink{pingpongpaint}{Problem~11} This gives us%
\begin{equation*}
\sum_{k=0}^{\lfloor n/2\rfloor}{n-2k+2\choose 2}
\end{equation*}
ways to paint the balls.%
%
\end{enumerate}
%
\backmatter
%
%
%% A lineskip in table of contents as transition to appendices, backmatter
\addtocontents{toc}{\vspace{\normalbaselineskip}}
%
\typeout{************************************************}
\typeout{References  Bibliography}
\typeout{************************************************}
\chapter[{Bibliography}]{Bibliography}\label{references-1}
%
%% The index is here, setup is all in preamble
\markright{Index}
\renewcommand{\leftmark}{Index}
\printindex
%
\end{document}