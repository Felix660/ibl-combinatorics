\chapter{What is Combinatorics?} Combinatorial mathematics
arises from studying how we can {\em combine} objects into
arrangements. For example, we might be combining sports teams
into a tournament, samples of tires into plans to mount them
on cars for testing,  students into classes to compare
approaches to teaching a subject, or members of a tennis club into pairs to
play tennis.  There are many questions one can ask about such arrangements of
objects.  Here we will focus on questions about {\em how many ways} we may
combine the objects into arrangements of the desired type.  These are
called \emph{counting problems}. Sometimes,
though, combinatorial mathematicians ask if an arrangement is possible (if we have ten baseball teams,
and each team has to play each other team once, can we schedule all the games
if we only have the fields available at enough times for forty games?). 
Sometimes they ask if all the arrangements we might be able to make have a
certain desirable property (Do all ways of testing 5 brands of tires on 5
different cars [with certain additional properties] compare each brand with
each other brand on at least one common car?).    Problems of these sorts
come up throughout physics, biology, computer science, statistics, and many
other subjects.  However, to demonstrate all these relationships, we would
have to take detours into all these subjects.  While we will give some
important applications, we will usually phrase our discussions around everyday
experience and mathematical experience so that the student does not have to
learn a new context before learning mathematics in context!


\section{About These Notes}  These notes are based on the
philosophy that you learn the most about a subject when you
are figuring it out  directly for yourself, and learn the
least when you are trying to figure out what someone else is
saying about it.  On the other hand, there is a subject called
combinatorial mathematics, and that is what we are going to be
studying, so we will have to tell you some basic facts.  What
we are going to try to do is to give you a chance to discover
many of the interesting examples that usually appear as
textbook examples and discover the principles that appear as
textbook theorems.  Your main activity will be solving problems designed
to lead you to discover the basic principles of combinatorial
mathematics.  Some of the problems lead you through a new idea, some give
you a chance to describe what you have learned in a sequence of problems,
and some are quite challenging.  When you find a problem challenging,
don't give up on it, but don't let it stop you from going on with other
problems.  Frequently you will find an idea in a later problem that you
can take back to the one you skipped over or only partly finished in order
to finish it off.  With that in mind, let's get started.  In the
problems that follow, you will see some problems marked on the left
with various symbols. The preface gives a full explanation of these
symbols and discusses in greater detail why the book is organized as
it is! Table~\ref{tab:prob-symbs}, which is repeated from the preface, summarizes
the meaning of the symbols.

\begin{table}[h]
  \centering
  \caption{The meaning of the symbols to the left of problem numbers.}
  \label{tab:prob-symbs}
\ \\
  \begin{tabular}{|c|l|}
    \hline
    $\bullet$& essential\\
    $\circ$ & motivational material\\
    $+$ & summary\\
    \importantarrow & especially interesting\\
    $*$ & difficult\\
    $\cdot$ & essential for this section or the next\\\hline
  \end{tabular}
\end{table}
\section{Basic Counting Principles}

\bp
\itemm Five schools are going to send their baseball teams to a
tournament, in which each team must play each other team
exactly once.  How many games are
required?\label{fiveteamtournament}
\solution{Think of numbering the five teams.  The first team must play all
four  others, the second team will be in one of these games but must play in
three more games, with the third, fourth and fifth team.  The third team is
in two of the games we've already mentioned, and must still play the fourth
and fifth team for two more games, and the fourth team must play the fifth
team in addition to playing in three of the games already mentioned.   Thus
there are $ 4+3+2+1 = 10$ games.  Alternatively, there are five teams, each of
which must play in four games, giving us 20 pairings of two teams each. 
However each game involves two of these pairings, so there are $20/2 =10$
games.}
\iteme Now some number $n$ of schools are going to send their
baseball teams to a tournament, and each team must play each
other team exactly once.  Let us think of the teams as
numbered 1 through $n$.\label{baseball2}
\begin{enumerate}
\item How many games does  team 1 have to play in?
\solution{$n-1$}
\item How many games, other than the one with team 1, does
team two have to play in?\solution{$n-2$}
\item How many games, other than those with the first $i-1$
teams, does team $i$ have to play in?\solution{$n-i$}
\item In terms of your answers to the previous parts of this
problem, what is the total number of games that must be played?
\solution{$1+2 +\cdots+ n-1$.  Although this need not be part of the answer, a
formula that we usually use in both algebra and calculus courses tells us
this sum is $n(n-1)/2$.}
\end{enumerate}

\iteme One of the schools sending its team to the tournament
has to send its players from some distance, and so it is
making sandwiches for team members to eat along the way. 
There are three choices for the kind of bread and five choices
for the kind of filling.   How many different kinds of
sandwiches are available? \label{basicsandwiches}
\solution{$3\cdot5=15$, or $5+5+5=15$.}

\items An \index{ordered pair}\index{pair,ordered}{\em ordered pair}
$(a,b)$ consists of two things we call $a$ and
$b$.  We say $a$ is the first member of the pair and $b$ is
the second member of the pair. If $M$ is an $m$ element set
and $N$ is an
$n$-element set, how many ordered pairs are there whose first
member is in $M$ and whose second member is in $N$?  Does this problem have
anything to do with any of the previous problems?
\label{orderedpair}
\solution{$m\cdot n$. This is because if $M = \{x_1,x_2,\ldots, x_m\}$, then
we have $n$ ordered pairs starting with $x_1$, $n$ ordered pairs starting
with $x_2$, and so on, so the total number of ordered pairs is a sum of $m$
terms, all equal to $n$. In problem
\ref{basicsandwiches} we were looking at ordered pairs of bread and filling. 
Less directly, and so not required for the answer, In Problem
\ref{fiveteamtournament} we have 20 ordered pairs, and each baseball game
involved two of the ordered pairs so we had 10 baseball games.  The same
argument applies to Problem \ref{baseball2}; namely we have $n$ teams each of
which is in an ordered pair with $n-1$ other teams, so we have $n(n-1)$
ordered pairs, and each game corresponds to two ordered pairs so we have
$n(n-1)/2$ games.  This gives us one proof of the formula we mentioned in the
solution to that problem.}

\itemm  Since a sandwich by itself is pretty boring, students
from the school in Problem \ref{basicsandwiches} are offered a
choice of a drink (from among five different kinds), a
sandwich, and a fruit (from among four different kinds).   In
how many ways may a student make a choice of the three items
now?\label{completelunch}
\solution{$5\cdot15\cdot4 = 300$.  Why do we multiply?  Multiplying five by 15
is equivalent to adding 15, the number of sandwiches, once for each drink,
giving us 75 combinations of drink and sandwich.  For each such pair we have
4 choices of fruit, and we can either think of adding 75 fours or adding four
75s to get three hundred.  Thus we multiply because multiplication is
repeated addition.}

\iteme  The coach of the team in Problem \ref{basicsandwiches}
knows of an ice cream parlor along the way where she plans to
stop to buy each team member a triple decker cone.  There are
12 different flavors of ice cream, and triple decker cones are made in
homemade waffle cones.  Having chocolate ice cream as the bottom scoop is
different from having chocolate ice cream as the top scoop. How many possible
ice cream cones are going to be available to the team
members?\label{tripledeckercone}  How many cones with three different kinds
of ice cream will be available?
\solution{$12\cdot12\cdot12 =1728$ possible cones.  If the flavors must be
different, we have $12\cdot11\cdot 10 = 1320$ possible cones.  In both cases,
the reason we are multiplying is as a shortcut for repeated addition.}



\iteme The idea of a function is ubiquitous in mathematics.  A
function\index{function} $f$ from a set $S$ to a set $T$ is a relationship
between the two sets that associates exactly one member $f(x)$
of $T$ with each element $x$ in $S$.    We
will come back to the ideas of functions and relationships in more detail and
from different points of view from time to time.  However, the quick review
above should probably let you answer these questions.  If you have difficulty
with them, it would be a good idea to go now to Appendix~\ref{Relations} and
work through Section \ref{relationsandfunctions} which covers this definition
in more detail. You might also want to study Section~\ref{digraphsoffunctions} to learn
to visualize the properties of functions. We will take up the topic of
this section later in this chapter as well, but in less detail than is
in the appendix.
\begin{enumerate} 
\item Using $f$, $g$, \ldots, to stand for the various
functions, write down all the different functions you can
from the set
$\{1,2\}$ to the set $\{a,b\}$.  For example, you might start
with
$f(1)=a$, $f(2)=b$.  How many functions are there from the
set $\{1,2\}$ to the set
$\{a,b\}$?\label{countingfunctionsparta}
\solution{$f(1) =a$, $f(2) = b$. Or, $h(1) =a$,
$h(2) =a$.  Or, $g(1)=b$, $g(2) = a$.   Or $j(1) =b$, $j(2) =b$. We have
exhausted all the possibilities for functions that associate a with 1 and
all possibilities for functions that associate $b$ with 1, so we have
exhausted all possibilities.  There are four such functions.} 
\item How many functions are there from the three element set
$\{1,2,3\}$
 to the two element set $\{a,b\}$?
\solution{$2\cdot2\cdot2 = 8$}
\item How many functions are there from the two element set $\{a,b\}$ to the
three element set $\{1,2,3\}$?
\solution{$3\cdot3=9$}
\item  How many functions are there from a three element set
to a 12 element set? 
\solution{$12\cdot12\cdot12 = 1728$}
\item The function $f$ is called
\index{one-to-one}\index{function!one-to-one}{\bf one-to-one} or an
\index{injection}\index{function!injection}{\em injection} if whenever
$x$ is different from
$y$,
$f(x)$ is different from $f(y)$.  How many one-to-one
functions are there from a three element set to a  12 element
set? 
\solution{$12\cdot11\cdot10= 1320$}
\item Explain the relationship between this problem and Problem
\ref{tripledeckercone}.
\solution{When we counted the number of possible ice cream cones we were
counting functions from the three places in the cone where ice cream would
sit to the 12 flavors.  When we counted the number of possible ice cream
cones with different flavors, we were counting the number of one-to-one
functions from the three places in the cone where ice cream would sit to the
12 flavors.}
\end{enumerate}\label{countingfunctions}



\iteme A group of hungry team members in Problem
\ref{tripledeckercone} notices it would be cheaper to buy three
pints of ice cream for them to split than to buy a triple decker cone for
each of them, and that way they would get more ice cream. They ask their
coach if they can buy three pints of ice cream. \label{icecreaminpints}
\begin{enumerate}
\item In how many ways can they choose three pints of different
  flavors out of the 12 flavors? \solution{
There are $12\cdot11\cdot 10 = 1320$ ways to make a list of three
flavors.  But a choice of three flavors accounts for $3\cdot2\cdot1 = 6$ of
those lists.  Therefore there are $1320/6=220$ ways to choose the pints if
the flavors are different.}
\item In how many ways may they choose three pints if the flavors
  don't have to be different?\solution{If the flavors need not be different, we must  add in the number of ways
to choose two pints of one flavor and one of a second and also the number fo
ways to choose three pints of one flavor. the first of these is $12\cdot11
=132$ and the third is 12, so we have $220+132+12=364$ ways to choose three 
pints.  We can do a more elegant solution after we learn about multisets in
Problem \ref{multiset}.
}
\end{enumerate}

\iteme Two  sets are said to be
\index{disjoint}\index{sets!disjoint}{\em disjoint} if they have no
elements in common.  For example,
$\{1,3,12\}$ and $\{6, 4, 8, 2\}$ are disjoint, but $\{1,3,12\}$ and
$\{3,5,7\}$ are not. Three or more sets are said to be
\index{sets!mutually disjoint}{\em mutually disjoint} if no two of them
have any elements in common.  What can you say about the  size of the
union of a  finite number of finite (mutually) disjoint sets?  Does
this have anything to do with any of the previous problems?
\label{sum}
\solution{The size of a union of disjoint sets is the sum of their sizes.  We
used this principle in problems 1 and two directly, and indirectly in every
other problem when we multiplied the number of ways of doing one thing times
the number of ways of doing another.  Note that we used this principle
informally in the explanation in the solution of Problem~\ref{orderedpair}.}

\iteme Disjoint subsets are defined in Problem \ref{sum}.  What can you
say about the size of the union of $m$ (mutually) disjoint sets, each
of size
$n$?  Does this have anything to do with any of the previous problems?
\label{product}
\solution{The size of the union is $m\cdot n$.  This is because the size of a
union of disjoint sets is the sum of their sizes, and a sum of $m$ terms each
equal to $n$ is $m\cdot n$.}

\ep


\subsection{The sum and product principles} These problems
contain among them the kernels of many of the fundamental
ideas of combinatorics.   For example, with luck, you just \index{sum
principle}\index{principle!sum}\index{principle!product} \index{product
principle}stated the sum principle (illustrated in Figure
\ref{sumprinc}), and product principle (illustrated in Figure
\ref{prodprinc}) in Problems
\ref{sum} and
\ref{product}.    These are two of the most basic
principles of combinatorics.  
\begin{figure}[htb]\caption{The union of these
two disjoint sets has size 17.}\label{sumprinc}\smallskip
\begin{center}\mbox{\psfig{figure=SumPrinc.eps%,height=1.0in
}}
\end{center}  
\end{figure}
\begin{figure}[htb]\caption{The
union of four disjoint sets of size
five.}\label{prodprinc}\smallskip
\begin{center}\mbox{\psfig{figure=ProdPrinc.eps%,height=1.0in
}}
\end{center}  
\end{figure}


You may have noticed some standard mathematical words and
phrases such as {\em set}, {\em ordered pair}, {\em function}
and so on creeping into the problems.  One of our goals in
these notes is to show how most counting problems can
be recognized as counting all or some of the elements of a set
of standard mathematical objects.  For example Problem
\ref{orderedpair} is meant to suggest that the question we
asked in Problem \ref{basicsandwiches} was really a problem of
counting all the ordered pairs consisting of a bread choice
and a filling choice.  We use $A\times B$ to stand for the set
of all ordered pairs whose first element is in $A$ and whose
second element is in $B$ and we call $A\times B$ the
\index{Cartesian product}\index{product!Cartesian}{\em Cartesian
product} of
$A$ and
$B$, so you can think of Problem
\ref{orderedpair} as asking you for the size of the Cartesian
product of
$M$ and $N$, that is, asking you to count the number of elements of
this Cartesian product.

When a set $S$ is a union of disjoint sets $B_1, B_2, \ldots, B_m$ we say
that the sets $B_1, B_2, \ldots, B_m$ are a {\bf
partition}\index{partition!of a set} of the set
$S$.  Thus a partition of $S$ is a (special kind of) set of sets.  So that
we don't find ourselves getting confused between the set $S$ and the sets
$B_i$ into which we have divided it, we often call the sets $B_1, B_2,
\ldots, B_m$ the {\em blocks}\index{partition!blocks of}\index{block of a
partition} of the partition.  In this language, the
\index{principle!sum}\index{sum principle}{\bf sum principle} says that 
\begin{quote}if we have a partition of a set $S$, then the size of $S$ is the
sum of the sizes of the blocks of the partition.\end{quote}  The
\index{principle!product}\index{product principle}{\bf product principle} says
that
\begin{quote}if we have a partition of a set $S$ into $m$ blocks, each
  of size $n$, then $S$ has size $mn$.\end{quote} You'll notice that
in our formal statement of the sum and product pinciple we talked
about a partition of a finite set. We could modify our language a bit
to cover infinite sizes, but whenever we talk about sizes of sets in
what follows, we will be working with finite sets. So as to avoid
possible complications in the future, let us agree that when we refer
to the size of a set, we are implicitly assuming the set is
finite. There is another version of the product principle that applies
directly in problems like Problem \ref{completelunch} and Problem
\ref{tripledeckercone}, where we were not just taking a union of $m$
disjoint sets of size $n$, but rather $m$ disjoint sets of size $n$,
each of which was a union of $m'$ disjoint sets of size $n'$.  This is
an inconvenient way to have to think about a counting problem, so we
may rephrase the product principle in terms of a sequence of
decisions: \bp \iteme If we make a sequence of $m$ choices for
which\label{generalproductprincipleintro}
\begin{itemize} 
\item There are $k_1$ possible first choices, and
\item for each way of making the first $i-1$ choices, there are $k_i$ ways to
make the $i$th choice,
\end{itemize}
then in how many ways may we
make our sequence of choices?  (You need not prove your answer correct at
this time.)
\ep
 
The counting principle you gave in Problem
\ref{generalproductprincipleintro} is called the {\em general product
  principle}.\index{general product principle}\index{product
  principle!general}\index{principle!product!general} We will outline
a proof of the general product pinciple from the original product
principle in Problem~\ref{generalproductprincipleproof}.  Until then,
let us simply accept it as another counting principle. For now, notice
how much easier it makes it to explain why we multiplied the things we
did in Problem \ref{completelunch} and Problem \ref{tripledeckercone}.
\bp

\itemi A tennis club has $2n$ members.  We want to pair up the
members by twos for singles matches. \label{tennispairings1} 
\begin{enumerate}
\item In how many ways may we pair up all the members of the club? (Hint: consider the cases
  of 2, 4, and 6 members.) \solution{Suppose we list the people in the club in some way, and keep that
list for the remainder of the problem.  Take the first person from the list
and pair that person with any of the $2n-1$ remaining people.  Now take the
next {\em unpaired} person from the list and pair that person with any of the
remaining $2n-3$ unpaired people.  Continuing in this way, once $k$ pairs have
been selected, take the next unpaired person from the list and pair that
person with any of the remaining $2n-2k-1$ unpaired people.  Every pairing
can arise in this way, and no pairing can arise twice in this process.  Thus
the number of outcomes is $\prod_{i=0}^{n-1} 2n-2i-1$.}

\item Suppose that in addition to specifying who plays whom, for each
  pairing we say who serves first.  Now in how many ways may we
  specify our pairs?
\end{enumerate}


\items  Let us now return to Problem \ref{countingfunctions} and
justify---or perhaps finish---our answer to the question about the
number of functions from a three-element set to a 12-element
set.\label{countingfunctions2}
\begin{enumerate}
\item How can you justify your answer in
  Problem~\ref{countingfunctions} to the question ``How many functions
  are there from a three element set (say $[3]=\{1,2,3\}$) to a twelve
  element set (say [12])?''  \solution{The number of functions from $[3]$ to $[12]$ with $f(3) =1$ is the
number of functions from $[2]$ to $[12]$, namely 144.  The same is true for
the number of functions with $f(3)=2$ and $f(3)=3$, and the size of the set
$S_i$.  The union of the sets $S_i$ is the set of all functions from $[3]$ to
$[12]$, and since there are twelve sets $S_i$, this union has size
 $12\cdot144= 1728$.}
\item Based on the examples you've seen so far, make a conjecture
  about how many functions there are from the set
  \[[m] = \{1,2,3,\dots,m\}\]
  to $[n]=\{1,2,3,\dots,n\}$ and prove it.
\label{numberoffunctionsconjecture}
 \solution{$n^m$.}
\item A common notation for the set of all functions from a set $M$ to a
set $N$ is $N^M$.  Why is this a good notation? 
\solution{Because there are $n^m$ such functions, at least according to our
conjecture.}
\end{enumerate}

\items Now suppose we are thinking about a set $S$ of functions $f$
from $[m]$ to some set $X$.  (For example, in
Problem~\ref{tripledeckercone} we were thinking of the set of
functions from the three possible places for scoops in an ice-cream
cone to $12$ flavors of ice cream.) Suppose there are $k_1$ choices
for $f(1)$.  (In Problem~\ref{tripledeckercone}, $k_1$ was $12$,
because there were $12$ ways to choose the first scoop.) Suppose that
for each choice of $f(1)$ there are $k_2$ choices for $f(2)$.  (For
example, in Problem~\ref{tripledeckercone} $k_2$ was $12$ if the
second flavor could be the same as the first, but $k_2$ was $11$ if
the flavors had to be different.) In general, suppose that for each
choice of $f(1)$, $f(2)$, \ldots $f(i-1)$, there are $k_i$ choices for
$f(i)$. (For example, in Problem~\ref{tripledeckercone}, if the
flavors have to be different, then for each coice of $f(1)$ and
$f(2)$, there are $10$ choices for $f(3)$.)

What we have assumed so far about the functions in $S$ may be
summarized as
\begin{itemize}
\item There are $k_1$ choices for $f(1)$.
\item For each choice of $f(1)$, $f(2)$, \ldots, $f(i-1)$, there are $k_i$
choices for~$f(i)$.
\end{itemize}  How many functions are in the set
$S$? Is there any practical difference between the result of this
problem and the general product principle?\label{generalproductprinciple}      
%  This is called the 
% {\em product principle for functions}\index{product
% principle!for functions} or the 
% {\bf general product principle}\index{product principle!general}. 
 \solution{$\prod_{i=1}^m k_i$. }
\ep

The point of Problem~\ref{generalproductprinciple} is that the general
product principle can be stated informally, as we did originally, or
as a statement about counting sets of standard concrete mathematical
objects, namely functions.

\bp
\itemi  A roller coaster car has $n$ rows of seats, each of which
has room for two people.  If $n$ men and $n$ women get into the car with a
man and a woman in each row,  in how many ways may they choose their seats?
\solution{$(n!)^22^n$}

% \iteme
%  How does the general product principle relate to counting  functions
% from
% $[m]$ to
% $[n]$?  How does it relate to counting    one-to-one functions from
% $[m]$ to
% $[n]$?  \index{functions!one-to-one!number of}\index{functions!number of}
%  \solution{It tells us there are $\prod_{i=1}^m n =n^m$ functions from $[m]$
% to
% $[n]$ and that there are $\prod_{i=1}^m n-i+1 $ one-to-one
% functions.}
% \items  Prove 
% the conjecture (about the number of functions in $S$) in Problem
% \ref{numberoffunctionsconjecture} when
% $m=2$ and  when $m=3$.  Prove the conjecture for an
% arbitrary positive integer
% $m$. \label{provenumberoffunctionsconjecture}\index{functions!number of}
%  \solution{When $m=2$, we have $n$ functions $f$ for each value of $f(1)$,
% and thus the set of all functions is a union of $n$ sets of size $n$, which
% by the product principle has size $n^2$. (Alternatively, we could apply the
% general product principle with $k_i=n$ for $i=1,2$.)  When
% $m=3$,  there is a bijection between the functions from $[m]$ to $[2]$ and
% functions
% $f$ from
% $[m]$ to
% $[3]$ with $f(3) =i$ for any particular $i$ between 1 and $n$.  Thus the
% set of all functions is a union of $n$ sets of size $n^2$ and so by the
% product principle has size $n^3$.  (Alternatively, we could apply the
% general product principle with $k_i=n$ for $i=1,2,3$.)  To prove the
% conjecture in general we apply the general product principle to get
% $\prod_{i=1}^n m =n^m$ functions from
% $[m]$ to
% $[n]$. 

%   As another option, we also prove by induction on
% $m$ that there are $n^m$ functions from $[m]$ to $[n]$.  When $m=0$ there is
% one function (the so-called "empty function" from $[m]$ to $[n]$).  Now assume
% inductively that when $m=k-1$ there are $n^{k-1}$ functions from $[m]$ to
% $[n]$.  For each value $i$ between 1 and $n$ there is a bijection between the
% functions from $[k-1]$ to $[n]$ and the functions $f$ from $k$ to $n$ with
% $f(k)=i$.  Thus the set of all functions from $[k]$ to $[n]$ is a union of
% $n$ sets of size $n^{k-1}$ and so by the ordinary product principle this set
% has size $n^k$.  Thus by the principle of mathematical induction, then number
% of functions from $m$ to $n$ is $m^n$ for all nonnegative integers $m$.}


\items How does the general product principle apply to Problem
\ref{tripledeckercone}?
\solution{By the general product principle, there are $12\cdot 11\cdot 10$
triple decker cones.}

\iteme In how many ways can we pass out $k$ distinct pieces of
fruit to
$n$ children (with no restriction on how many pieces of fruit a child may
get)?  
 \solution{Either by the formula for the number of functions from an
$m$-element set to an $n$-element set or the general product principle, there
are $k^n$ ways.  (Each distribution is a function from the set of fruit to
the set of children, because each piece of fruit goes to one and only one
child.)}

\iteme \label{SubsetsFirstTime}How many subsets does a set $S$ with $n$ elements have?
\solution{For each of the $n$ elements of $S$, we have two options:
  either we put the element into the subset or we do not. Thus, the
  general product principle tells us that there are $2^n$ subsets of $S$.}

\itemm  Assuming $k\le n$, in how many ways can we pass out  $k$
distinct pieces of fruit to
$n$ children if each child may get at most one?  What is the number if
$k>n$?  Assume for both questions that we pass out all the fruit.
\solution{We are asking for the number of $k$-element permutations of $n$
children, which is $\prod_{i=1}^k n-i+1$, and is zero if $k>n$.}


\iteme Another name for a list, in a specific order, of $k$ distinct things
chosen from a set $S$ is a  $\mathbf k$-{\bf element permutation of}
$\mathbf S$.\index{permutation!
$k$-element}  We can also think of a $k$-element permutation of $S$ as a
one-to-one function (or, in other words, injection) from
$[k]=\{1,2,\ldots, k\}$ to $S$.  How many $k$-element permutations does an
$n$-element set have?\label{kelementpermutation}  (For this problem it is
natural to assume $k\le n$.  However the question makes sense even if $k>n$. 
What is the number of $k$-element permutations of an $n$-element set if $k>n$?
 \solution{By the general product principle, the number is $$\prod_{i=1}^k
n-i+1 $$.  In the case that $k>n$, there are no such lists
with distinct entries, and that is what the formula gives us, because
$n-(n+1)+1=0$.}
\ep


There are a number of different notations for the number of $k$-element
permutations of an $n$-element set.  The one we shall use was introduced by
Don Knuth; namely $n^{\underline{k}}$, read ``$n$ to the $k$ falling'' or
``$n$ to the $k$ down". In Problem \ref{kelementpermutation} you may have
shown that 
\begin{equation}n^{\underline{k}} =n(n-1)\cdots (n-k+1)= \prod_{i=1}^k(
n-i+1).\label{productnotation}
\end{equation}
  It is standard to call $n^{\underline{k}}$\index{$n^{\underline{k}}$}  the
{\bf
$k$-th falling factorial power of $n$}\index{falling factorial
power}\index{factorial power!falling}, which explains why we use exponential
notation.  Of course we call it a {\em factorial} power since
$n^{\underline{n}} = n(n-1)\cdots 1$ which we call {\em $n$-factorial} and
denote by
$n!$.\index{factorial}\index{$n"!$}  If you are unfamiliar with the Pi
notation, or {\em product notation}\index{product notation}\index{Pi
notation} we introduced for products in Equation
\ref{productnotation}, it works just like the Sigma notation works for
summations.

\bp
\iteme Express $n^{\underline{k}}$ as a quotient of factorials.
\solution{$n^{\underline{k}}=n!/(n-k)!$}
\itemi How should we define $n^{\underline{0}}$?\solution{We define
  $n^{\underline{0}}$ to be $1$.}
\ep


\subsection{Functions and directed graphs} 

As another example how standard mathematical language relates to
counting problems, Problem \ref{countingfunctions} explicitly asked
you to relate the idea of counting functions to the question of
Problem \ref{tripledeckercone}.  You have probably learned in algebra
or calculus how to draw graphs in the Cartesian plane of functions
from a set of numbers to a set of numbers.  You may recall how we can
determine whether a graph is a graph of a function by examining
whether each vertical straight line crosses the graph at most one
time.  You might also recall how we can determine whether such a
function is one-to-one by examining whether each horizontal straight
line crosses the graph at most one time.  The functions we deal with
will often involve objects which are not numbers, and will often be
functions from one finite set to another.  Thus graphs in the
cartesian plane will often not be available to us for visualizing
functions.  

However, there is another kind of graph called a {\em directed
  graph}\index{graph! directed}\index{directed graph} or {\em
  digraph}\index{digraph}\index{function!digraph of} that is
especially useful when dealing with functions between finite sets. We
take up this topic in more detail in Appendix~\ref{Relations},
particularly Section~\ref{relationdigraph} and Section~\ref{digraphsoffunctions}. In
Figure \ref{functiondigraphs} we show several examples of digraphs of functions.
\begin{figure}[hbt]\caption{What is a digraph of a
function?}\label{functiondigraphs}\smallskip
\begin{center}\psfig{figure=functiondigraph.eps,height=4.5in
}\end{center}
\end{figure}
If we have a function $f$ from a set $S$ to a set $T$, we draw a line
of dots or circles, called \emph{vertices} to represent the elements of $S$ and another
(usually parallel) line of circles or dots to represent the elements
of $T$.  We then draw an arrow from the circle for $x$ to the circle
for $y$ if $f(x) = y$. Sometimes, as in part (e) of the figure, if we
have a function from a set $S$ to itself, we draw only one set of
vertices representing the elements of $S$, in which case we can have
arrows both entering and leaving a given vertex. As you see, the
digraph can be more enlightening in this case if we experiment with
the function to find a nice placement of the vertices rather than
putting them in a row.

\textbf{The figure needs to have part (e) added to it.}



Notice that there is a simple test for whether a digraph whose vertices
represent the elements of the sets $S$ and $T$ is the digraph of a function
from $S$ to $T$.  There must be one and only one arrow leaving each vertex of
the digraph representing an element of $S$.  The fact that there is one arrow
means that $f(x)$ is defined for each $x$ in $S$.  The fact that there is
only one arrow means that each $x$ in $S$ is related to exactly one element of
$T$. (Note that these remarks hold as well if we have a function from
$S$ to $S$ and draw only one set of vertices representing the elements
of $S$.) For further discussion of functions and digraphs see Sections
\ref{functionrelation} and \ref{relationdigraph} of {Appendix~\ref{Relations}}.

\bp
\itemm Draw the digraph of the function from the set $\{$Alice, Bob, Dawn,
Bill$\}$ to the set $\{$A, B, C, D, E$\}$ given by $$f(X) = \mbox{the first
letter of the name $X$}.$$ 
\solution{
\mbox{\psfig{figure=AliceBobdigraph.eps, height=.75 in }}
}
\iteme A function
$f:S\rightarrow T$ is called an
\index{onto function}\index{function!onto}{\em onto function} or
\index{surjection}\index{function!surjection}{\em surjection} if each
element of
$T$ is $f(x)$ for some $x\in S$.  Choose a set $S$ and a set $T$ so that you
can draw the digraph of a function from $S$ to $T$ that is one-to-one but not
onto, and draw the digraph of such a function.
\solution{The digraph of one such function  follows\\
\mbox{\psfig{figure=OnetoOneNotOnto.eps}}
}
\itemm Choose a set $S$ and a set $T$ so that you can draw the digraph of a
function from $S$ to $T$ that is onto but not one-to-one, and draw the
digraph of such a function.
\solution{The digraph of one such function follows\\
\mbox{\psfig{figure=OntoNotOnetoOne.eps}}
}
\iteme Digraphs of functions help us visualize the ideas of one-to-one
functions and onto functions.
\begin{enumerate}
\item What does the digraph of a one-to-one function (injection) from
  a finite set $X$ to a finite set $Y$ look like? (Look for a test
  somewhat similar to the one we described for when a digraph is the
  digraph of a function.)
\item   What does the digraph of an onto function
  look like?
\item   What does the digraph of a one-to-one and onto function
  from a finite set $S$ to a set $T$ look
  like? \label{bijectiondigraph}
\end{enumerate}

\solution{One-to-one: One arrow comes out of each vertex representing a member
of
$X$ and at most one arrow goes into each vertex representing a member of $Y$. 
Onto:  : One arrow comes out of each vertex representing a
member of $X$ and at least one arrow goes into each vertex representing a
member of
$Y$.  One-to-one and onto: One arrow comes out of each vertex representing a
member of
$X$ and at exactly one arrow goes into each vertex representing a member of $Y$.
(The first half of each sentence is optional.)}

\iteme The word {\em permutation} is actually used in two different ways in
mathematics.  A {\bf permutation}\index{permutation!as a bijection} of a set
$S$ is a bijection from $S$ to
$S$.  How many permutations does an $n$-element set have?
\label{permutationasbijection}
\solution{$n!$.}
\ep 

Notice that there is a great deal of consistency between the use of
the word permutation in Problem~\ref{permutationasbijection} and the
use in Problem~\ref{kelementpermutation}.  If we have
some way $a_1,a_2,\ldots,a_n$ of listing our set, then any other list
$b_1,b_2,\ldots,b_n$ gives us the bijection whose rule is
$f(a_i) =b_i$, and any bijection, say the one given by $g(a_i)=c_i$
gives us a list $c_1,c_2,\ldots,c_n$ of $S$.  Thus there is really
very little difference between the idea of a permutation of $S$ and an
$n$-element permutation of $S$ when $n$ is the size of $S$.


\subsection{The bijection principle} Another name for a one-to-one
and onto function is \index{bijection}\index{function!bijection}{\bf
bijection}.  The first two digraphs in Figure \ref{functiondigraphs} are
digraphs of bijections.  The description in Problem
\ref{bijectiondigraph} of the digraph of a bijection  from
$X$ to
$Y$ illustrates one of the fundamental principles of combinatorial
mathematics, the
\index{principle!bijection}\index{bijection principle}{\bf bijection
principle};
\begin{quote}  Two sets have the same size if and only if
there is a bijection between them.
\end{quote} It is surprising how this innocent sounding
principle guides us into finding insight into some otherwise
very complicated proofs.

\subsection{Counting subsets of a set}
\bp 
\item The {\em binary} representation of a number $m$ is a list, or string,
$a_1a_2\ldots a_k$ of zeros and ones such that $m=a_12^{k-1} + a_2 2^{k-2}
+\cdots+ a_k 2^0.$  Describe a bijection between the binary representations of
the integers between 0 and $2^n-1$ and the subsets of an $n$-element set. 
What does this tell you about the number of subsets of an $n$-element
set?\label{SubsetsBinaryRepresentation}\solution{The sequence $a_1a_2\ldots
a_k$ corresponds to the set of $i$ such that $a_i = 1$.  This is a bijection
because each sequence gives a set, and each set can be the set of places
where a sequence is 1.  Since there are $2^n$ integers which are between 0
and $2^n-1$, and they correspond to sequences of length $n$ (notice, we have
another bijection, the one between a number and its binary representation),
there are $2^n$ subsets of the $n$-element set $[n]$.}
\ep

Notice that the first question in Problem \ref{icecreaminpints} asked
you for the number of ways to choose a three element subset from a 12
element subset.  You may have seen a notation like
$n\choose k$, $C(n,k)$, or $_nC_k$ which stands for the number of ways to
choose a
$k$-element subset from an
$n$-element set.  The number $n\choose k$ is read as ``$n$ choose $k$'' and is
called a
\index{binomial coefficient}{\bf binomial coefficient}  for
reasons we will see later on.  Another frequently used  way to read the
binomial coefficient notation  is ``the number of combinations
\index{combinations} of
$n$ things taken $k$ at a time." You are going to be asked to construct
two bijections that relate to these numbers and figure out what famous
formula they prove.  We are going to think about subsets of the
$n$-element set $[n] =
\{1,2,3,\ldots, n\}$. As an example, the set of two-element subsets of
$[4]$ is
$$\{\{1,2\}, \{1,3\}, \{1,4\}, \{2,3\}, \{2,4\}, \{3,4\}\}.$$  This
example tells us that ${4\choose 2} = 6$. 

\bp
\iteme Let
$C$ be the set of $k$-element subsets of $[n]$ that contain
the number
$n$, and let
$D$ be the set of $k$-element subsets of $[n]$ that don't
contain $n$.  
\begin{enumerate} 
\item Let $C'$ be the set of $(k-1)$-element subsets of
$[n-1]$.  Describe a bijection from $C$ to $C'$.  (A verbal
description is fine.)
\solution{Let $f(X) = X-\{n\}$, the set $X$ with $n$ removed.  This is a
bijection because two different sets containing $n$ must yield different sets
when $n$ is removed (one-to-one), and each $(k-1)$-element subset $X$ of
$[n-1]$ may be obtained from the $k$-element subset $X\cup \{n\}$ of $[n]$ by
removing
$n$ (onto).}
\item Let $D'$ be the set of $k$-element subsets of
$[n-1]=\{1,2,\ldots n-1\}$.  Describe a bijection from $D$ to
$D'$. (A verbal description is fine.)
\solution{Simply let $f(X) =X$.  This is one-to-one by definition, and onto
because the subsets of $[n-1]$ are identical with the subsets of $[n]$ not
containing $n$.}
\item Based on the two previous parts, express the sizes of
$C$ and $D$ in terms of binomial coefficients involving $n-1$
instead of $n$.\solution{$|C|= {n-1\choose k-1}$; $|D| = {n-1\choose k}$}
\item Apply the sum principle to $C$ and $D$ and obtain a
formula that expresses $n\choose k$ in terms of two binomial
coefficients involving
$n-1$.  You have just derived the Pascal Equation that is the
basis for the famous Pascal's Triangle.
\solution{${n\choose k} = {n-1\choose k-1} +{n-1\choose k}$.}
\end{enumerate}\label{Pascal} 

\ep

\subsection{Pascal's Triangle}

The Pascal Equation that you derived in Problem \ref{Pascal}
gives us the triangle in Figure \ref{Pascaltriangle}.  This
figure has the number of $k$-element subsets of an $n$-element
set as the $k$th number over in the
$n$th row (we call the top row the zeroth row and the
beginning entry of a row the zeroth number over).  You'll see
that your formula doesn't say anything about
$n\choose k$ if
$k=0$ or
$k=n$, but otherwise it says that each entry is the sum of
the two that are above it and just to the left or right.  

\begin{figure}[h]\caption{Pascal's
Triangle}\label{Pascaltriangle}
$$\begin{matrix}
&&&&&&&1&&&&&&&\\
&&&&&&1&&1&&&&&&\\
&&&&&1&&2&&1&&&&&\\
&&&&1&&3&&3&&1&&&&\\
&&&1&&4&&6&&4&&1&&\\
&&1&&5&&10&&10&&5&&1&\\
&1&&6&&15&&20&&15&&6&&1\\
1&&7&&21&&35&&35&&21&&7&&1
\end{matrix}$$
\end{figure}\index{Pascal's Triangle}


\bp
\item Just for practice, what is the next row of Pascal's triangle?
\solution{1,8,28,56,70,56,28,8,1}

\itemi Without writing out the rows completely, write out enough of Pascal's
triangle to get a numerical answer for the first question  in Problem
\ref{icecreaminpints}.
\solution{Starting with row 9, we get
$$\begin{matrix}&&&1&&9&&36&&84 \cr&&1&&10&&45&&120\cr&1&&11&&55&&165\cr
1&&12&&66&&220\end{matrix}$$ so the answer is 220.}
\ep

It is less common to see Pascal's triangle as a right
triangle, but it actually makes your formula easier to
interpret.  In Pascal's Right Triangle, the element in row $n$
and column $k$ (with the convention that the first row is row
zero and the first column is column zero) is $n\choose k$.  In
this case your formula says each entry in a row is the sum of
the one above and the one above and to the left, except for
the leftmost and right most entries of a row, for which that
doesn't make sense.  Since the leftmost entry is $n\choose 0$
and the rightmost entry is
$n\choose n$, these entries are both one (to see why, ask yourself how
many $0$-element subsets and how many $n$-element subsets an
$n$-element set has), and
your formula then tells how to fill in the rest of the table.

\begin{figure}[ht]\caption{Pascal's Right
Triangle}\label{Pascalrighttriangle}
\begin{center}
  \begin{tabular}{c|cccccccc}
    &$k=0$ & 1 & 2 & 3 & 4 & 5 & 6 & 7\\\hline
    $n=0$  &1&&&&&&\\
    1& 1&1&&&&&\\
    2& 1&2&1&&&&&\\
    3&1&3&3&1&&&&\\
    4& 1&4&6&4&1&&&\\
    5& 1&5&10&10&5&1&&\\
    6&1&6&15&20&15&6&1&\\
    7& 1&7&21&35&35&21&7&1
  \end{tabular}
\end{center}
\end{figure}

Seeing this right triangle leads us to ask whether there is
some natural way to extend the right triangle to a rectangle. 
If we did have a rectangular table of binomial coefficients, counting
the first row as row zero (i.e., $n=0$) and the first column as column
zero (i.e., $k=0$), the entries we don't yet have are values of
$n\choose k$ for $k>n$.  But how many $k$-element subsets does
an
$n$-element set have if $k>n$?  The answer, of course, is
zero, so all the other entries we would fill in would be zero,
giving us the rectangular array in Figure \ref{Pascal'sRectangle}.   
It is straightforward to check that Pascal's equation now works for all
the entries in the rectangle that have an entry above them and an
entry above and to the left.

\begin{figure}[ht]\caption{Pascal's
Rectangle}\label{Pascal'sRectangle}
\begin{center}
  \begin{tabular}{c|cccccccc}
    &$k=0$ & 1 & 2 & 3 & 4 & 5 & 6 & 7\\\hline
    $n=0$  &1&0&0&0&0&0&0&0\\
    1& 1&1&0&0&0&0&0&0\\
    2& 1&2&1&0&0&0&0&0\\
    3&1&3&3&1&0&0&0&0\\
    4& 1&4&6&4&1&0&0&0\\
    5& 1&5&10&10&5&1&0&0\\
    6&1&6&15&20&15&6&1&0\\
    7& 1&7&21&35&35&21&7&1
  \end{tabular}
\end{center}
\end{figure}

\bp
\itemi Because our definition told us that $\binom{n}{k}$ is 0 when
$k>n$, we got a rectangular table of numbers that satisfies the Pascal
Equation.  
\begin{enumerate}
\item Is there any other way to define $n \choose k$ when $k>n$
in order to get a rectangular table that agrees with Pascal's Right
Triangle for $k\le n$ and satisfies the Pascal Equation?  
\solution{No, because there must be a zero above each one not in column zero. 
Then above each zero not in column zero or one, there must be yet another zero
and so on.}
\item Suppose we
want to extend Pascal's Rectangle to the left and define $n\choose -k$
for $n\ge 0$ and $k>0$ so that $-k<0$. What should we put into row
$n$ and column
$-k$ of Pascal's Rectangle in order for the Pascal Equation to
hold true?  
  \solution{To the left of all the ones in column zero, we must have zeros
for the Pascal Equation to hold.  To the left of those zeros, we must again
have zeros, and so on. }
\itemitemh What should we put into row $-n$ and column $k$ or
column $-k$ in order for the Pascal Equation to continue to
hold?  Do we have any freedom of choice?
\solution{
 Above row zero, we have some freedom. The
-1,-1 and the -1,0 entry must add to one, so they can be $-x$ and $x+1$ for
any number $x$.  To the right of the -1,0 entry they must alternate between
$(-x-1)$ and $x+1$ while to the left of the -1,1 entry they must alternate
between $-x$ and $x$, ending with the $-x$ in position -1,1.  Now if we know
the entry in row -2 and column zero, we can use the Pascal equation (in the
form ${n-1\choose k-1} = {n\choose k} - {n-1\choose k}$) to compute all the
entries to the left of it, and (in a different form) to compute all the
entries to the right of it.  Thus we may be arbitrary about the entries in
column 0 (or, in fact, one entry in each row) and then the Pascal Equation
tells us how to fill in the rest of each row.  We shall see later on that
there is one very natural choice for how to fill in all the rows above row
zero.} 
\end{enumerate}



% \iteme There are  variants of  the bijections we used to prove the
% \label{SubsetsByInduction} Pascal Equation that can be used to give another
% proof of a formula (as in Problem \ref{SubsetsBinaryRepresentation}) for the
% number of subsets of an
% $n$-element set using the Principle of Mathematical Induction.  If you are
% familiar with Mathematical Induction, try to find the proof.  If not, now is
% the time to visit the Appendix on Mathematical Induction (and work through
% the problems there).  From this point forward, we shall assume that you are
% familiar with the principle of mathematical induction.
% \solution{We prove inductively that the number of subsets of an the
% $n$-element set $[n]$ is $2^n$.  The bijection between $[n]$ and an arbitrary
% $n$-element set extends to a bijection between their sets of subsets, so it
% suffices to consider the $n$-element set $[n]$.  In particular, when
% $n=0$, we have the empty set, and it has just $1=2^0$ subset, the empty set
% itself.  Now assume that $[n-1]$  has
% $2^{n-1}$ subsets.  The subsets of $[n]$ may be divided into those that
% contain
% $n$ and those that do not.  There is a bijection between the subsets that do
% contain $n$ and the subsets of the set $[n-1]$.  Thus there are $2^{n-1}$
% subsets that do contain $n$.  There is also a bijection between the subsets
% that do not contain $n$ and the subsets of the set $[n-1]$.  Thus there are
% $2^{n-1}$ subsets of $[n]$ that do not contain $n$.  By the sum principle the
% total number of subsets of $[n]$ is $2^{n-1}+2^{n-1} = 2\cdot2^{n-1} = 2^n$. 
% Therefore by the principle of mathematical induction, for all $n$ the number
% of subsets of $[n]$, and thus of any $n$-element set, is $2^n$.}

 



\item  There is yet another bijection that lets us prove  that a set of
size $n$ has $2^n$ subsets.  Namely, for each subset $S$ of
$[n]=\{1,2,\ldots, n\}$, define a function (traditionally
denoted by $\Chi_S$) as follows.\footnote{The symbol
$\Chi$ is the Greek letter chi that is pronounced Ki, with the
$i$ sounding like ``eye."} 
$$\Chi_S(i) = \begin{cases}1 &\mbox{ if } i\in S \\ 0 &\mbox{ if } i\not\in
S\end{cases}$$ The function $\Chi_S$ is called the \index{characteristic
function}\index{function!characteristic}{\em
characteristic function} of
$S$.  Notice that the characteristic function is a function
from $[n]$ to
$\{0,1\}$.\label{charfunction}
\begin{enumerate}
\item For practice, consider the function $\Chi_{\{1,3\}}$ for
the subset
$\{1,3\}$ of the set $\{1,2,3,4\}$.  What are
\begin{enumerate}
\item $\Chi_{\{1,3\}}(1)$?
\solution{1}
\item $\Chi_{\{1,3\}}(2)$?
\solution{0}
\item $\Chi_{\{1,3\}}(3)$?
\solution{1}
\item $\Chi_{\{1,3\}}(4)$?
\solution{0}
\end{enumerate}
\item We define a function $f$ from the set of subsets of
$[n]=\{1,2,\ldots, n\}$ to the set of functions from $[n]$ to
$\{0,1\}$ by $f(S)=\Chi_S$.  Explain why $f$ is a bijection. 
\solution{
If $i\in S$ but $i\not\in T$, then $\Chi_S(i)=1$ but $\Chi_T(i)=0$. 
Thus if $S\not= T$, the $\Chi_S\not=\Chi_T$.  Therefore $f$ is one-to-one. 
Given a function $g$ from $[n]$ to $\{0,1\}$, let $S=\{i|g(i)=1\}$.  Then by
definition, $g=\Chi_S=f(S)$.  Therefore  $f$ is onto, so it is a bijection.}
\item Why does the fact that $f$ is a bijection prove that
$[n]$ has $2^n$ subsets?
\solution{We have seen that there are $2^n$ functions from $[n]$ to a
two-element set, and we have a bijection between the set of all such
functions and the subsets of $[n]$.}
\end{enumerate}


\ep

In Problems \ref{SubsetsFirstTime}, \ref{SubsetsBinaryRepresentation}, and
\ref{charfunction} you gave three proofs of the following
theorem.

\begin{theorem} The number of subsets of an $n$-element set is
$2^n$.
\end{theorem}


The proofs in Problem \ref{SubsetsBinaryRepresentation} and
\ref{charfunction} use essentially the same bijection, but they
interpret sequences of zeros and ones differently, and so end up being
different proofs. We will give yet another proof, using bijections
similar to those we used in proving the Pascal Equation, at the
beginning of Chapter~\ref{InductionRecursion}.


\subsection{The quotient principle}

\bp
\iteme As we noted in Problem \ref{Pascal}, the first question in
Problem \ref{icecreaminpints} asked us for the number of three-element
subsets of a twelve-element set.  We were able to use the Pascal Equation to
get a numerical answer to that question.  Had we had twenty or thirty flavors
of ice cream to choose from, using the Pascal Equation to get our answer
would have entailed a good bit more work. We have seen how the general
product principle gives us an answer to Problem \ref{tripledeckercone}.  Thus
we might think that the number of ways to choose a three element set from 12
elements is the number of ways to choose the first element times the number
of ways to choose the second element times the number of ways to choose the
third element, which is $12\cdot11\cdot10=1320$.  However, our result in
Problem \ref{Pascal} shows that this is wrong.  What is it that is different
between the number of ways to stack ice cream in  a triple decker cone with
three different flavors of ice cream and the number of ways to simply choose
three different flavors of ice cream?  In particular, how many different
triple decker cones use the same three flavors?  Using this, compute the
number of ways to choose three different flavors of ice cream (from 12
flavors) from the number of ways to choose a triple decker cone with three
different flavors (from 12 flavors).\label{twelvechoosethree}
\solution{What is different is that the order in which we put the scoops into
the cone matters, but for simply choosing three flavors, the order of the
choices doesn't matter.  Six different triple decker cones have the same
three flavors.  Thus we have $1320/6=220$ different ways to choose three
flavors of ice cream from 12 flavors.}

\iteme  Based on what you observed in Problem
\ref{twelvechoosethree}, how many $k$-element subsets does an
$n$-element set have?\label{nchoosek}
\solution{Following the reasoning of Problem \ref{twelvechoosethree}, there
are $n^{\underline{k}}$ $k$-element permutations of an $n$-element set, and
$k!$ of these permutations list the same set of $k$ elements, so the number of
$k$-element sets is ${n^{\underline{k}}\over k!}= {n!\over k!(n-k)!}$.}

\iteme The formula you proved in Problem \ref{nchoosek} is symmetric in $k$
and $n-k$; that is, it gives the same number for $n\choose k$ as it gives
for $n\choose n-k$.  Whenever two quantities are counted by the same formula
it is good for our insight to find a bijection that demonstrates the two sets
being counted have the same size.  In fact this is a guiding principle of
research in combinatorial mathematics.  Find a bijection that proves that
$n\choose k$ equals $n\choose n-k$.
\solution{For each $k$-element subset $K$ of the $n$-element set $N$, define
$f(K)$ to be the set of all elements of $N$ {\em not} in $K$.  Then $f$ is
the desired bijection.}

\iteme In how many ways can we pass out $k$ (identical) ping-pong balls to $n$
children if each child may get at most one?\label{ping-pong}
\solution{$n\choose k$, because we choose the $k$ children to whom we give
ping-pong balls.}

\itemi While the formula you proved in Problem \ref{nchoosek} is very useful,
it doesn't give us a sense of how big the binomial coefficients are.  We can
get a very rough idea, for example, of the size of $2n\choose n$ by
recognizing that we can write $(2n)^{\underline{n}}/n!$ as ${2n\over n}\cdot
{2n-1\over n-1}\cdots {n+1\over 1}$, and each quotient is at least $2$, so the
product is at least $2^n$.  If this were an accurate estimate, it would mean
the fraction of $n$-element subsets of a $2n$-element set would be
about
$2^n/2^{2n}=1/2^n$, which is becomes very small as $n$ becomes large. 
However it is pretty clear the approximation will not be a very good one,
because some of the terms in that product are much larger than 2.  In fact,
if $2n\choose k$ were the same for every $k$, then each would be the fraction
$1\over 2n+1$, so we know this is a bad approximation.  For estimates like
this, James Stirling developed a formula to approximate
$n!$  when $n$ is large, namely $n!$ is about $ \left(\sqrt{2\pi
n}\right){n^n/ e^n}$.\index{Stirling's formula for
$n"!$}\index{$n"!$!Stirling's formula for}  In fact the ratio of
$n!$ to this expression approaches 1 as
$n$ becomes infinite.\footnote{Proving this takes more of a detour than is
advisable here; however there is an elementary proof which you can work
through in the problems of the end of Section 1 of Chapter 1 of {\em
Introductory Combinatorics} by Kenneth P. Bogart, Harcourt Academic Press,
(2000).}  We write this as $$n!\sim \sqrt{2\pi
n}{n^n\over e^n}.$$  Use Stirling's formula to show that the fraction of 
subsets of size
$n$ in an
$2n$-element set is approximately $1/\sqrt{\pi n}$.  This is a much bigger
fraction than $1\over 2^n$!\label{Stirling'sapproximation}



\iteme In how many ways may $n$ people sit around a round
table?  (Assume
 that when people are sitting around a round table, all that
really matters is who is to each person's right.  For example, if we can get
one arrangement of people around the table from another by having everyone
get up and move to the right one place and sit back down, we get an
equivalent arrangement of people.  Notice that you can get a
list from a seating arrangement by marking a place at the
table, and then listing the people at the table, starting at
that place and moving around to the right.)  There are at least two
different ways of doing this problem.   Try to find them both.
\label{roundtable}
\solution{The total number of ways to list how the $n$ people sit around the
table is $n!$.  However, two lists are the same if we get one from the other
by shifting everyone right the same number of places.  This divides the set
of lists up into sets  of  $n$ mutually equivalent lists.  The number $m$ of
such sets is the number of seating arrangements.  However by the product
principle, $mn=n!$, because we have partitioned up the set of $n!$ lists into
$m$ sets of size $n$.  Therefore $m=(n-1)!$  A second solution may be
obtained by choosing one of the $n$ people and letting this person sit
anywhere.  Since all that matters is who is to the right of each person, it
doesn't matter where this person sits.  Once this person is seated, let
everybody else  sit down. If they sit down first in one order clockwise around
the table and then in some other order, the person to the right of somebody
has changed.    Thus there are
$(n-1)!$ ways (the number of ways to seat everybody else) to seat the people
around the table.}


\iteme A given $k$-element subset
can be listed as a $k$-element permutation in $k!$ ways.  We can partition the
set of all $k$-element permutations of $S$ up into blocks by letting $B_K$ be
the set of all $k$-element permutations of $K$ for each $k$-element subset $K$
of
$S$.  How many permutations are there in a block?  If $S$ has $n$ elements,
what does problem \ref{kelementpermutation} tell you about the total number
of $k$-element permutations of $S$?  Describe a bijection between the
set of blocks of the partition and the set of $k$-element subsets of $S$. 
What formula does this give you for the number $n\choose k$ of $k$-element
subsets of an
$n$-element set? \label{formulanchoosek}
\solution{The number of permutations in a block is $k!$.  Problem
\ref{kelementpermutation} tells us that the total number of $k$-element
permutations is $n^{\underline{k}}={n!\over (n-k)!}$.  Each $k$-element set
corresponds to the block of all permutations of that set.  It is immediate
that this is a bijection.  Assuming there are $s$ subsets, we have
$k!s$ permutations in total, so $k!s={n!\over (n-k)!}$ or $s= {n!\over
k!(n-k)!}$.}

\itemi A basketball team has 12 players.  However, only five players play at
any given time during a game.  In how may ways may the coach choose the five
players?  To be more realistic, the five players playing a game normally
consist of two guards, two forwards, and one center.  If there are five
guards, four forwards, and three centers on the team, in how many ways can the
coach choose two guards, two forwards, and one center?  What if one of the
centers is equally skilled at playing forward?
\solution{$12\choose 5$. In the more realistic version,
${5\choose2}{4\choose2}{3\choose1}=180$. Finally,  either the versatile
player is playing center or not, and in the second case is available to play
forward.  This gives us
${5\choose2}{4\choose2}{1\choose1}+{5\choose2}{5\choose2}{2\choose1}=260$ ways
to choose the players.}



\iteme In Problem \ref{roundtable}, describe a way to partition the
$n$-element permutations of the $n$ people into blocks so that there is a
bijection between the set of blocks of the partition and the set of
arrangements of the
$n$ people around a round table.  What method of solution for Problem
\ref{roundtable} does this correspond to? \label{roundtablepartition}
\solution{Put two permutations in the same block if we can get one from the
other by moving everyone (circularly) some number $r$ places to the right. 
This corresponds to the method that gives $n!/n$ as the answer.  Many
students should be able to answer this question by saying ``See the answer to
Problem \ref{roundtable}."}

\iteme In Problems \ref{formulanchoosek} and \ref{roundtablepartition}, you
have been using the product principle in a new way.  One of the ways in which
we previously stated the product principle was ``If we partition a set into
$m$ blocks each of size $n$, then the set has size $m\cdot n$.''  In problems
\ref{formulanchoosek} and \ref{roundtablepartition} we knew the size $p$ of a
set
$P$ of permutations of a set, and we knew we had partitioned $P$ into some
unknown number of blocks, each of a certain known size $r$.  If we let
$q$ stand for the number of blocks, what does the product principle tell us
about $p$, $q$, and
$r$?  What do we get when we solve for $q$?\label{quotientprinciple}
\solution{$p=qr$, so that $q=p/r$.}
\ep

The formula you found in the Problem \ref{quotientprinciple} is so useful that
we are going to single it out as another principle.  The {\bf quotient
principle}\index{quotient principle} says: 
\begin{quote}If we partition a set $P$ 
into $q$ blocks, each of size $r$, then $q=p/r.$
\end{quote}
The quotient principle is really just a restatement of the product principle,
but thinking about it as a principle in its own right often leads us to find
solutions to problems.  Notice that it does not always give us a formula for
the number of blocks of a partition; it only works when all the blocks have
the same size.

In Section \ref{equivalencerelations} of Appendix \ref{Relations} we
introduce the idea of an equivalence relation, see what equivalence relations
have to do with partitions, and discuss the quotient principle from that
point of view.  While that appendix is not required for what we are doing
here, if you want a more thorough discussion of the quotient principle, this
would be a good time to work through that appendix.

\bp
\itemei In how many ways may we string $n$ distinct beads on a necklace
without a clasp? (Assume someone can pick up the necklace,
move it around in space and put it back down, giving an
apparently different way of stringing the beads that is
equivalent to the first.  How could we get a list of beads
from a necklace?)\label{necklace}
\solution{We can obtain a permutation of the beads by cutting the necklace and
stretching it out in a straight line.  We can partition the permutations
according to which necklace they come from in this process. Two permutations
are in the same block if we get one either by circularly permuting the other
or by reversing the other (this corresponds to flipping the necklace over in
space).  Thus each necklace corresponds to $2n$ permutations so by the
quotient principle we have $n!/2n=(n-1)!/2$ ways to string $n$ distinct beads
on a necklace.}

\itemi We first gave this problem as Problem
\ref{tennispairings1} Now we have several ways to approach the problem.  A
tennis club has
$2n$ members.  We want to pair up the members by twos for singles matches. 
In how many ways may we pair up all the members of the club?  Suppose that in
addition to specifying who plays whom, for each pairing we say who serves
first.  Now in how many ways may we specify our pairs?\label{tennispairings2}
\solution{Suppose we list the people in the club in some way, and keep that
list for the remainder of the problem.  Take the first person from the list
and pair that person with any of the $2n-1$ remaining people.  Now take the
next {\em unpaired} person from the list and pair that person with any of the
remaining $2n-3$ unpaired people.  Continuing in this way, once $k$ pairs have
been selected, take the next unpaired person from the list and pair that
person with any of the remaining $2n-2k-1$ unpaired people.  Every pairing
can arise in this way, and no pairing can arise twice in this process.  Thus
the number of outcomes is $\prod_{i=0}^{n-1} 2n-2i-1$.  

For another solution,
choose people in pairs.  There are $2n\choose 2$ ways to choose one pair,
$2n-2\choose 2$ ways to choose a second pair, and once $k$ pairs have been
chosen, there are $2n-2k\choose 2$ ways to choose the next pair.  The number
of {\em lists} of pairs we get in this way is $\prod_{i=0}^{n-1}
{2n-2i\choose 2}= {(2n)!\over 2^i}$.  However each way of pairing people gets
listed $n!$ times since we see all possible length $n$ lists of pairs. 
Therefore the number of actual pairings is ${(2n)!\over 2^n n!}= ={2n!\over
2n\cdot2n-2\cdot2n-4\cdot \cdots\cdot 2} =  \prod_{i=0}^{n-1} 2n-2i-1$.

For yet another solution, we can list the $2n$ members in $(2n)!$ ways.  Then
we can take the first two as a tennis pair, the next two, and so on.  There
are $n!$ ways that a given set of tennis pairings could be arranged, and
each of the $n$ pairs could appear in 2 ways, so the tennis pairings
partition the set of all permutations of the $2n$ members into blocks of size
$n!2^n$.  Thus we have $(2n)!\over n!2^n$ tennis pairings once again. }

\iteme In how many ways may we attach two identical red beads
and two identical blue beads to the corners of a square free
to move around in (three-dimensional) space? \label{twocolorsofbeads}
\solution{Two ways; either the red beads are side-by-side or diagonally
opposite.  If we think about partitioning lists of 2 $R$s and 2 $B$s so that
two are in the same block if we get one from the other by moving the square,
we get two blocks, $\{RRBB, BRRB, BBRR, RBBR\}$ and $\{RBRB, BRBR\}$.}
\ep

\section{Some Applications of the Basic Principles}

\subsection{Lattice paths and Catalan Numbers} 
\bp
\itemm In a part of a city, all streets run either north-south or east-west,
and there are no dead ends.  Suppose we are standing on a street corner.  In
how many ways may we walk to a corner that is four blocks north and six blocks
east, using as few blocks as possible?\label{blockwalking}
\solution{The shortest possible walk is going to be ten blocks.  To plan a
walk, we must choose which four of those ten blocks go north; the other six
blocks we will have to go east.  There are $10\choose 4$ ways to make this
selection.}

\itemes Problem \ref{blockwalking} has a geometric interpretation in a
coordinate plane.  A {\em lattice path}\index{lattice
path}\index{path!lattice} in the plane is a ``curve" made up of line segments
that either go from a point
$(i,j)$ to the point
$(i+1,j)$ or from a point $(i,j)$ to the point $(i,j+1)$, where $i$ and $j$
are integers. (Thus lattice paths always move either up or to the right.) The
length of the path is the number of such line segments.  What is the length
of a  lattice path from
$(0,0)$ to
$(m,n)$?    How many such lattice paths of that length are
there? How many lattice paths are there from $(i,j)$ to $(m,n)$,
assuming $i$,
$j$,
$m$, and
$n$ are integers?\label{latticepaths}
\solution{The length of a lattice path from $(0,0)$ to $(m,n)$ is $m+n$.  The
number of such paths is $m+n\choose n$.  Since lattice paths move up and to
the right, there are no paths from $(i,j)$ to $(m,n)$ unless $i\le m$ and
$j\le n$.  In that case, the number of paths is $m+n-i-j\choose n-j$ which is
the same as $m+n-i-j\choose m-i$.}


\itemes Another kind of geometric path in the plane is a {\em diagonal lattice
path}\index{lattice path!diagonal}\index{path!lattice!diagonal}.  Such a path
is a path made up of line segments that go from a point $(i,j)$ to
$(i+1,j+1)$ (this is often called an {\em upstep}) or
$(i+1,j-1)$ (this is often called a {\em downstep}), again where
$i$ and
$j$ are integers. (Thus diagonal lattice paths always move towards the right
but may move up or down.) Describe which points are connected to
$(0,0)$ by diagonal lattice paths.  What is the  length of a diagonal
lattice path from $(0,0)$ to $(m,n)$?  Assuming that
$(m,n)$ is such a point, how many  diagonal lattice paths are there
from
$(0,0)$ to
$(m,n)$?\label{diagonallattice}
\solution{The points $(m,n)$ connected to $(0,0)$ by diagonal lattice paths
will have $m+n$ is even, because each upstep adds two
to the sum of $i$ and $j$ while each downstep does not change the sum. 
Further, since we go one step to the right each time we go up or down, we
cannot get above the line $y=x$ or below the line $y=-x$.  However for any
point $(m,n)$ with $m$ and $n$ nonnegative integers such that $m+n$ even and
$-m\le n\le m$, we can get to $m,n$ by making $m-n\over 2$ downsteps and
$m+n\over2$ upsteps. In this way we will make a total of $m$ steps, and our
total motion parallel to the $y$ axis will be ${m+n\over2}-{m-n\over2} = n$. 
The length of such a path is $m\sqrt{2}$; we might informally just call it
$m$ steps.  The number of possible paths is the number of ways we can choose
which of the $m$steps are upsteps (or equivalently downsteps) this number is
$m\choose{m+n\over2}$.}

\itemm A school play requires a ten dollar donation per person; the donation
goes into the student activity fund.  Assume that each person who comes to
the play pays with a ten dollar bill or a twenty dollar bill.  The teacher
who is collecting the money forgot to get change before the event.  If there
are  always at least as many people who have paid with a ten as a twenty as
they arrive the teacher won't  have to give anyone an IOU for change.  Suppose
$2n$ people come to the play, and exactly half of them pay with ten dollar
bills.  
\begin{enumerate}
\item Describe a bijection between the set of sequences of tens and
twenties people give the teacher and the set of lattice paths from $(0,0)$ to
$(n,n)$.  
\solution{For each ten dollar bill take a rightstep and for each twenty dollar
bill take an upstep (where rightstep and upstep have the hopefully natural
meaning).  The assumption that there are an equal number of ten and twenty
dollar bills means that the path will end up at $(n,n)$.  Each sequence of
tens and twenties gives a lattice path and each lattice path corresponds to
such a sequence, so we have a bijection.}
\item Describe a bijection between the set of sequences of tens and
twenties that people give the teacher and the set of diagonal lattice paths
from $(0,0)$ and $(0,2n)$.  
\solution{For each ten dollar bill take an upstep and for each twenty dollar
bill take a downstep.  Each sequence of tens and twenties will give us a
diagonal lattice path from $(0,0)$, and each diagonal lattice path
from$(0,0)$ to $(0,2n)$ will give us a sequence of tens and twenties with
an equal number of tens and twenties, so we have a bijection.}
\item In each case, what is the geometric
interpretation of a sequence that does not require the teacher to give any
IOUs?
\solution{In the first case a sequence that does not require the teacher to
give any IOUs will correspond to a lattice path that stays on or below the
line $y=x$, and in the second case such a sequence will correspond to a
diagonal lattice path that stays on or above the $x$-axis.}
\end{enumerate}

\itemesi Notice that a  lattice path from $(0,0)$ to $(n,n)$ stays
inside (or on the edges of) the square whose sides are the $x$-axis, the
$y$-axis, the line
$x=n$ and the line $y=n$.  In this problem we will compute the number of
lattice paths from (0,0) to
$(n,n)$ that stay inside (or on the edges of) the triangle whose sides are the
$x$-axis, the line $x=n$ and the line $y=x$.  For example, in Figure
\ref{CatalanPaths} we show the grid of points with integer coordinates for
the triangle whose sides are the $x$-axis, the line $x=4$ and the line $y=x$.
\begin{figure}[htb]\caption{The lattice paths
from $(0,0)$ to $(i,i)$ for $i=0,1,2,3,4$. 
The number of paths to the point $(i,i)$ is
shown just above that point.}\label{CatalanPaths}
\begin{center}\mbox{\psfig{figure=CatalanPaths.eps%,height=1.0in
}}
\end{center}
\end{figure}
\begin{enumerate}
\item Explain why the number of lattice paths from $(0,0)$ to
$(n,n)$ that go outside the triangle is the number of lattice paths from
$(0,0)$ to
$(n,n)$ that either touch or cross the line $y=x+1$.
\solution{If a lattice path between $(0,0)$ and $(n,n)$ goes outside the
triangle, it can only do so on an upstep. (A step from $(i,j)$ to
$(i,j+1)$.)  And an upstep must originate at a point with integer
coordinates.  If $j<i$ an upstep from $(i,j))$ cannot leave the triangle. 
Thus to leave the triangle, the upstep must leave from a point of the form
$(i,i)$, and go to $(i,i+1)$, which is on the line $y=x+1$.
\item Find a bijection between  lattice paths from $(0,0)$ to $(n,n)$
that touch (or cross) the line $y=x+1$ and lattice paths from
$(-1,1)$ to $(n,n)$.
}
\solution{Suppose we have a lattice path form $(0,0)$ to $(n,n)$ which
touches or crosses the line $y=x+1$.  Let $(k,k+1)$ be the first point on the
line $y=x+1$ that the lattice path touches.  From that point, work backwards,
replacing every upstep with a step one unit to the left and every rightstep
with a step one unit down.  The segment of the path you just changed will
have moved left $k+1$ times, so its leftmost $x$ coordinate will be $-1$, and
it will have moved down $k$ times, so its lowest $y$ coordinate will be 1. 
Thus we now have a lattice path from $(-1,1)$ to $(n,n)$.  Further, given a
lattice path from $(-1,1)$ to $(n,n)$, it must cross the line $y=x+1$ at least
once, because it starts above the line and ends below it.  At the first point
where such a path touches the line $y=x+1$, say $(k',k'+1)$, work backwards
replacing every upstep with a step to the left and every rightstep with a
step downwards.  The leftmost point on this path will have $x$ coordinate 0,
and the lowest point will have $y$ coordinate 0, so the new path will be a
lattice path from $(0,0)$ to $(n,n)$ that touches the line $y=x+1$.  Clearly
these two processes reverse each other, and so they give us a bijection
between paths form $(0,0)$ to $(n,n)$ that touch the line $y=x+1$ and lattice
lattice paths from $(-1,1)$ to $(n,n)$.  Notice that geometrically what we
are doing to get the bijection is to take the portion of a lattice path that
goes from the initial point till the first touch of the line $y=x+1$ and
reflecting it around that line.  This idea of reflection was introduced by
Feller, and is called Feller's reflection principle.}

\item Find a formula for the number of lattice paths from $(0,0)$ to $(n,n)$
that do not cross the line $y=x$.  The number of such paths is called a {\em
Catalan Number}\index{Catalan Number} and is usually denoted by $C_n$.
\solution{$C_n={2n\choose n} - {2n \choose n+1}={1\over n+1}{2n\choose n}.$}
\end{enumerate}

\itemi Your formula for the Catalan Number can be expressed as a binomial
coefficient divided by an integer.  Whenever we have a formula that calls
for division by an integer, an ideal combinatorial explanation of the
formula is one that uses the quotient principle.  The purpose of this
problem is to find such an explanation using diagonal lattice
paths.\footnote{The result we will derive is called the Chung-Feller
Theorem\index{Chung-Feller Theorem}; this approach is based of a paper of
Wen-jin Woan ``Uniform Partitions of Lattice Paths and Chung-Feller
Generalizations," {\bf American Mathematics Monthly 58} June/July 2001,
p556.}  A diagonal lattice path that never goes below the
$y$-coordinate of its first point is called a {\em Dyck Path}\index{Dyck
path}.  We will call a Dyck Path from
$(0,0)$ to
$(2n,0)$ a {\em Catalan Path}\index{Catalan Path} of length $2n$.  Thus the
number of Catalan Paths of length $2n$ is the Catalan Number $C_n$.
\begin{enumerate} 
\item If a Dyck Path has $n$ steps (each an upstep or
downstep), why do the first $k$ steps form a Dyck Path for each nonnegative
$k\le n$?\solution{If no points on the path are lower than the first point,
then no points among the first $k$ steps are lower than the first point.}
\item Thought of as a curve in the plane, a diagonal lattice path can have many
local maxima and minima, and can have several absolute maxima and minima, that is,
several highest points and several lowest points. What is the $y$-coordinate
of an absolute minimum point of a Dyck Path starting at $(0,0)$?  Explain why
a Dyck Path whose rightmost absolute minimum point is its last point is a
Catalan Path.
\solution{Since the path starts at $(0,0)$ and can't go below it, the $y$
coordinate of an absolute minimum must be zero.  If the last point is an
absolute minimum, then (because it ends with the same $y$ coordinate with
which it starts) the path has an even number
$2k$ of steps and ends at
$(2k,0)$.}
\item Let $D$ be the set of all diagonal lattice paths from $(0,0)$ to
$(2n,0)$.  (Thus these paths can go below the $x$-axis.) Suppose we partition
$D$ by letting
$B_i$ be the set of lattice paths in $D$ that have $i$ upsteps (perhaps mixed
with some downsteps) following the last absolute minimum.  How many blocks
does this partition have?  Give a succinct description of the block $B_0$.
\solution{The path must have $n$ upsteps total, and so can have any number
between 0 and $n$ upsteps after the rightmost absolute minimum.  Thus the
partition has $n+1$ blocks.  Block $B_0$ consists of the Catalan Paths.}
\item How many upsteps are in a Catalan Path? \solution{$n$.} 
\itemitemh We are going to give a bijection between the set of Catalan Paths and
the block $B_i$ for each $i$ between $1$ and $n$.  For now, suppose the
value of $i$, while unknown, is fixed.  We take a Catalan path and break it
into three pieces.  The piece $F$ (for ``front") consists of all steps before
the
$i$th upstep in the Catalan path.  The piece $U$ (for ``up") consists of the
$i$th upstep.  The piece
$B$ (for ``back") is the portion of the path that follows the $i$th upstep. 
Thus we can think of the path as $FUB$.  Show that the function that takes
$FUB$ to
$BUF$ is a bijection from the set of Catalan Paths onto the block $B_i$ of the
partition.  (Notice that $BUF$ can go below the $x$ axis.)
\solution{Since we are starting with a Catalan path, the point on the path at
the beginning of the $i$th upstep must have $y$ coordinate greater or equal
to  than zero.  Thus wherever we start the sequence $F$ of upsteps and
downsteps, a path constructed by this sequence never goes lower than its
starting point. Thus  in $BUF$ the last absolute minimum is either right
before the
$U$ or earlier.  But $B$ is the final segment of a Catalan Path, so its final
point is at least as low as its starting point.  Thus the point at the
beginning of the $U$ in $BUF$ is an absolute minimum, and there are $i$
upsteps after that local minimum.  If we take two different sequences and
rearrange them in the same way, we get two different sequences, so the
function we just described is a one-to-one function.  If we take an arbitrary
diagonal lattice path from $(0,0)$ to $(2n,0)$, let $U'$ be the first upstep
after the last absolute minimum, $F'$ be the portion of the path that follows
$U'$, and $B'$ be the portion that precedes $U'$, then $F'U'B'$ is a Catalan
Path, and $U'$ is its $i$th upstep if and only if in $B'U'F'$ there are $i$
upsteps after the last absolute minimum.  Thus the mapping from $FUB$ to
$BUF$ is a bijection.}
\item Explain how you have just given another proof of the formula for the
Catalan Numbers.
\solution{We have taken the set of all $2n\choose n$ diagonal lattice paths
of length $2n$ from $(0,0)$ to $(2n,0)$ and partitioned it into $n+1$ blocks
all of size $C_n$.  Thus by the quotient principle, $C_n={1\over
n+1}{2n
\choose n}$.}
\end{enumerate}

\subsection{The Binomial Theorem}
\itemm We know that $(x+y)^2 = x^2+2xy+y^2$.  Multiply both sides by $(x+y)$
to get a formula for $(x+y)^3$ and repeat to get a formula for $(x+y)^4$.  Do
you see a pattern?  If so, what is it?  If not, repeat the process to get a
formula for $(x+y)^5$ and look back at Figure \ref{Pascaltriangle} to see the
pattern.  Conjecture a formula for $(x+y)^n$. \label{Conjecturebinomthm}
\solution{$(x+y)^3=x^3+2x^2y +xy^2+x^2y+ +2xy^2 +y^3=x^3+3x^2y++3xy^2+y^3$.\\
Similarly, $(x+4)^4=x^4+4x^3y+6x^2y^2+4xy^3+y^4$,\\
and $(x+y)^5=x^5+5x^4y+10x^3y^2+10x^2y^3+5xy^4+y^5.$
The pattern is that the coefficient of $x^iy^j$ is $i+j\choose i$ which is
the same as $i+j\choose j$.  Said differently, the coefficient of
$x^{n-i}y^i$ is $n\choose i$ or the coefficient of $x^iy^{n-i}$ is $n\choose
i$.  We conjecture that $$(x+y)^n=\sum_{i=0}^n {n\choose i}x^{n-i}y^i.$$
(The reason for putting $x^{n-i}y^i$ into the sum is so that as $i$ goes from
0 to $n$, the powers of $x$ decrease from $n$ to 0.)}

\iteme  When we apply the distributive law $n$ times to $(x+y)^n$, we get a
sum of terms of the form $x^iy^{n-i}$ for various values of the integer $i$.  
\begin{enumerate}
\item If it is clear to you that each term of the form $x^iy^{n-i}$ that we
get comes from choosing an $x$ from $i$ of the $(x+y)$ factors and a $y$ from
the remaining
$n-i$ of the factors and multiplying these choices together, then answer this
part of the problem and skip the next part.  Otherwise, do the next part
instead of this one.  In how many ways can we choose an $x$ from $i$ terms
and a $y$ from
$n-i$ terms?\solution{The number of ways to choose an $x$ from $i$ of the
factors and a $y$ from the remaining ones is the way to choose the $i$
factors from the $n$ factors; that is, $n\choose i$.}
\item  Expand the product $(x_1 +y_1)(x_2 +y_2)(x_3+y_3)$.  What do you get
when you substitute $x$ for each $x_i$ and $y$ for each $y_i$?  Now imagine
expanding $$(x_1+y_1)(x_2+y_2)\cdots (x_n+y_n).$$  Once you apply the
commutative law to the individual terms you get, you will have a sum of terms
of the form $$x_{k_1}x_{k_2}\cdots x_{k_i}\cdot y_{j_1}y_{j_2}\cdots
y_{j_{n-i}}.$$
What is the set $\{k_1,k_2,\ldots, k_i\}\cup \{j_1,j_2,\ldots, j_{n-i}\}$?  In
how many ways can you choose the set  $\{k_1,k_2,\ldots, k_i\}$?  Once you
have chosen this set, how many choices do you have for $\{j_1,j_2,\ldots,
j_{n-i}\}$?  If you substitute $x$ for each $x_i$ and $y$ for each $y_i$, how
many terms of the form $x^iy^{n-i}$ will you have in the expanded product
$$(x_1+y_1)(x_2+y_2)\cdots (x_n+y_n)=(x+y)^n?$$  How many terms of the form
$x^{n-i}y^i$ will you have?
\solution{\begin{eqnarray*}
(x_1+y_1)(x_2+y_2)(x_3+y_3)=x_1x_2x_3 \!\!&+&\!\!x_1x_2y_3+x_1y_2x_3+\\
y_1x_2x_3+x_1y_2y_3+y_1x_2y_3 \!\!&+&\!\! y_1y_2x_3+y_1y_2y_3.\end{eqnarray*} 
When you substitute $x$ for each $x_i$ and
$y$ for each
$y_i$, you get $(x+y)^3=x^3+3x^2y+3xy^2+y^3$.

$\{k_1,k_2,\ldots, k_i\}\cup
\{j_1,j_2,\ldots, j_{n-i}\}=\{1,2,\ldots, n\}$.  You can choose the set
$\{k_1,k_2,\ldots k_i\}$ in $n\choose i$ ways.  Then there is just one way to
choose the set of $j$s.  If you substitute $x$ for $x_i$ and substitute $y$
for $y_i$, you will get $n\choose i$ terms of the form $x^iy^{n-i}$.  You
will also get $n\choose i$ terms of the form $x^{n-i}y^i$.}
\item Explain how you have just proved your conjecture from Problem
\ref{Conjecturebinomthm}.  The theorem you have proved is called the
{\bf Binomial Theorem}.\index{Binomial Theorem}
\solution{We have proved that the coefficient of $x^iy^{n-i}$ in $(x+y)^n$ is
$n\choose i$, or equivalently that the coefficient of $x^{n-i}y^i$ in
$(x+y)^n$ is $n\choose i$.}
\end{enumerate}
\item What is $\sum_{i=1}^n {10\choose i}3^i$?
\solution{\hspace*{-6 pt}$\sum_{i=1}^n {10\choose i}3^i=\sum_{i=0}^n
{10\choose i}3^i-{10\choose 0}3^0 =(1+3)^{10}-1=4^{10}-1$}
\item What is ${n\choose 0}-{n\choose 1}+{n\choose 2}-\cdots \pm
{n\choose n}$ if $n$ is an integer bigger than 0?
\itemei Explain why $$\sum_{i=0}^m{m\choose i}{n\choose k-i} = {m+n\choose
k}.$$  Find two different explanations.
%%%%Changed upper limit of sum from n to m.
\solution{When we expand both sides of $(x+y)^m(x+y)^n=(x+y)^{m+n}$ by the
binomial theorem we get $\sum_{i=0}^m{m\choose i}{n\choose  k-i}$ as the
coefficient of $x^{m+n-k}y^k$ on the left hand side and $m+n\choose k$ on the
right hand side.  For a second explanation, to choose $k$ elements out of the
union of an $m$-element set and a disjoint $n$-element set, chose some number
$i\le m$ of them from the $m$-element set and the remaining $k-i$ of them
from the
$n$-element set.  The sum on the left hand side of the equation simply sums
the number of such choices over all possible $i$, and the binomial
coefficient on the right hand side of the equation says we will end up
choosing $k$ elements from among our $m+n$ elements.}
\itemi  From the symmetry of the binomial coefficients, it is not too hard to
see that when
$n$ is an odd number, the number of subsets of $\{1,2,\ldots,n\}$ of odd size
equals the number of subsets of $\{1,2,\ldots,n\}$  of even size.  Is it true
that when $n$ is even the number of subsets of $\{1,2,\ldots,n\}$ of even
size equals the number of subsets of odd size?  Why or why not?
\solution{It is true, because if $n>0$, when you expand $(1-1)^n$ by the
binomial theorem, you get an alternating sum of binomial coefficients equal
to 0, and so the sum of the binomial coefficients $n\choose i$ with $i$ even
must equal the sum of the binomial coefficients $n\choose i$ with $i$ odd.}

\itemi What is $\sum_{i=0}^n i{n\choose i}$?   (Hint: think about how you
might use calculus.)
\solution{$\sum_{i=0}^n({n\choose i}x^i = (1+x)^n$.  Taking derivatives of
both sides gives us $\sum_{i=0}^ni{n\choose i}x^{i-1} = n(1+x)^{n-1}.$.  Now
substitute 1 for $x$ and you get $\sum_{i=0}^n i{n\choose i} = n2^{n-1}$.}


\ep

\subsection{The pigeonhole principle}
\bp 
\itemm American coins are all marked with the year in which they were made. 
How many coins do you need to have in your hand to guarantee that on two (at
least) of them, the date has the same last digit?\label{elevencoins}
\solution{Since there are ten possible last digits, you need at least 11
coins, and with 11 coins, at least two last digits must be the same.}
\ep
 
There are many ways in which you might explain your answer to Problem
\ref{elevencoins}.  For example, you can partition the coins according to the
last digit of their date; that is, you put all the coins with a given last
digit in a block together, and put no other coins in that block; repeating
until all coins are in some block.  Then you have a partition of your set of
coins.  If no two coins have the same last digit, then each block has exactly
one coin.  Since there are only ten digits, there are at most ten blocks and
so by the sum principle there are at most ten coins. In fact with ten coins
it is possible to have no two with the same last digit, but with 11 coins
some block must have at least two coins in order for the sum of the sizes of
at most ten blocks to be 11.  This is one explanation of why we need 11
coins in Problem \ref{elevencoins}.  This kind of situation arises often in
combinatorial situations, and so rather than always using the sum principle
to explain our reasoning , we enunciate another principle which we can think
of as yet another variant of the sum principle.  The {\bf pigeonhole
principle}\index{pigeonhole principle} states that
\begin{quote} If we partition a set with more than $n$ elements into 
$n$ parts, then at least one part has more than one element.\end{quote}  The
pigeonhole principle gets its name from the idea of a grid of little boxes
that might be used, for example, to sort mail, or as mailboxes for a group of
people in an office.  The boxes in such grids are sometimes called
pigeonholes in analogy with stacks of boxes used to house homing pigeons when
homing pigeons were used to carry messages.  People will sometimes state the
principle in a more colorful way as ``if we put more than $n$ pigeons into
$n$ pigeonholes, then some pigeonhole has more than one pigeon.''

\bp 
\item Show that if we have a function from a set of size $n$ to a set of
size less than $n$, then $f$ is not one-to-one.
\solution{Let $T$ be the set of size less than $n$, and $S$ be the set of
size $n$.  Let $B_j=\{i|f(i)=j\}$ for each $j$ in $T$.  Then the nonempty
sets among the $B_j$s form a partition of $S$ and the number of blocks is
less than the size of $S$.  Therefore by the pigeonhole principle, there is
at least one block with at least two elements, so there are two elements
$i_1$ and $i_2$ such that $f(i_1)=f(i_2)$.}

\iteme Show that if $S$ and $T$ are finite sets of the same size, then a
function
$f$ from $S$ to $T$ is one-to-one if and only if it is onto.
\solution{First suppose that $f$ is a one-to-one function from $S$ to $T$,
sets which have the same size. Let $B_j=\{i|f(i)=j\}$ for each $j$ in $T$.. 
If
$f$ is not onto, then the number of nonempty sets $B_j$ is smaller than the
number of elements of
$T$ and thus is smaller than the size of $S$. The nonempty sets $B_j$ are a
partition of
$S$. But then by the pigeonhole principle, some nonempty
$B_j$ has two or more elements, contradicting the assumption that $f$ is
one-to-one.  Therefore if $f$ is one-to-one, then it is onto.  Now suppose
that $f$ is an onto function from $S$ to $T$, sets of the same size.  Again
let $B_j =\{i|f(i)=j\}$ for each $j$ in $T$.  The size of the union of the
sets $B_j$ is, by the sum principle, the sum of their sizes. Since $f$ is
onto, each $B_j$ has at least one element. Since the number of sets $B_j$ is
the number of elements of
$S$, if one of those sets has more than one element, the the size of their
union is more than the size of $S$, which is a contradiction since they are
subsets of $S$.  Therefore each set $B_j$ has exactly one element and
therefore $f$ is one-to-one.}

\itemes There is a {\em generalized pigeonhole principle}\index{pigeonhole
principle! generalized} which says that if we partition a set with more than
$kn$ elements into $n$ blocks, then at least one block has at least $k+1$
elements.  Prove the generalized pigeonhole principle.
\solution{Suppose we partition a set $S$ of more than $kn$ elements into $n$
blocks.  If each block has at most $k$ elements, the by the sum principle
the size of  $S$ is at most $kn$.  But this is a contradiction, so some block
has at least $k+1$ elements.}

\item All the powers of five end in a five, and all the powers of two are
even.  Show that for for some integer $n$, if you take the first $n$ powers of
a prime other than two or five, one must have ``01'' as the last two digits.
\solution{If we take  40 powers  of such a
prime, either one will end in ``01'' or some two, say
$p^i$ and $p^j$ with $i>j$ must have the same last two digits by the pigeon
hole principle.  Then $p^i-p^j=100k$ for some integer $k$.  Thus $p^j(p^{i-j}
-1)$ must be a multiple of 100, and since neither 2 nor 5 divide $p$, $p^{i-j}
-1 = 100k'$ for some integer $k'$ then $p^{i-j} = 100k'+1$, so the last two
digits of $p^{i-j}$ must be ``01.''}

\itemesi Show that in a set of six people, there is a set of at least three
people who all know each other, or a set of at least three people none of
whom know each other.  (We assume that if person A knows person B, then person
B knows person A.)\label{R(3,3)}
\solution{By the generalized pigeonhole principle, person 1 either knows at
least three people or doesn't know at least three people.  Suppose person 1
knows three people.  Then either two of these people know each other, giving
us, with person 1, three mutual acquaintances, or no two of these people know
each other, giving us three mutual strangers.  On the other hand if there are
three people person 1 does not know, then either two of these people don't
know each other, giving us, with person 1, three mutual strangers, or all
three of these people know each other, giving us three mutual acquaintances.}

\itemes Draw five circles labeled Al, Sue, Don, Pam, and Jo.  Find a way to
draw red and green lines between people so that every pair of people is
joined by a line and there is neither a triangle consisting entirely of red
lines or a triangle consisting of green lines.  What does Problem
\ref{R(3,3)} tell you about the possibility of doing this with six people's
names?  What does this problem say about the conclusion of Problem
\ref{R(3,3)} holding when there are five people in our set rather than
six?\label{notR(3,3)}
\solution{\begin{center}
\mbox{\psfig{figure=NonRamsey5.eps%,height=1.0in
}}\vspace*{-.5in}\end{center}Problem \ref{R(3,3)} says you can't do this with
six people's names.  This problem says that the conclusion of Problem
\ref{R(3,3)} does not hold when you have five people.}

\ep

\subsection{Ramsey Numbers}\label{Ramseysection}
Problems \ref{R(3,3)} and \ref{notR(3,3)}  together show that six is the
smallest number $R$ with the property that if we have $R$ people in a room,
then there is either a set of (at least) three mutual acquaintances or a
set of (at least) three mutual strangers.  Another way to say the same
thing is to say that six is the smallest number so that no matter how we
connect 6 points in the plane (no three on a line) with red and green lines,
we can find either a red triangle or a green triangle.  There is a name for
this property.  The {\bf Ramsey Number} $R(m,n)$ is the smallest number $R$
so that if we have $R$ people in a room, then there is a set of at least $m$
mutual acquaintances or at least $n$ mutual strangers.  There is also a
geometric description of Ramsey Numbers; it uses the idea of a {\em complete
graph}  on $R$ vertices.  A {\bf complete graph}\index{graph!complete} on $R$
vertices consists of $R$ points in the plane together with line segments (or
curves) connecting each two of the $R$ vertices.\footnote{As you may have
guessed, a complete graph is a special case of something called a graph. 
THe word graph will be defined in Section \ref{graphsection}.} The points are
called {\em vertices}\index{vertex!of a complete graph}\index{vertex} and the line
segments are called {\em edges}\index{edge!of a complete graph}\index{edge}. In
Figure
\ref{completegraph} we show three different ways to draw a complete graph on
four vertices.  We use $K_n$ to stand for a complete graph on $n$ vertices.

\begin{figure}[htb]\caption{Three ways to draw a
complete graph on four vertices}\label{completegraph}\vglue-1in
\begin{center}\vglue-.5in\mbox{\psfig{figure=threeK4s.eps%,height=1.0in
}}
\end{center}
\end{figure}

Our geometric description of $R(3,3)$ may be translated into the language
of graph theory (which is the subject that includes complete graphs) by
saying $R(3,3)$ is the smallest number $R$ so that if we color the edges of a
$K_R$ with two colors, then we can find in our picture a
$K_3$  all of whose edges have the same color.  
The graph theory description of $R(m,n)$ is that $R(m,n)$ is the smallest
number $R$ so that if we color the edges of a $K_R$ with red and
green, then we can find in our picture either a $K_m$ all of whose edges are
red or a
$K_n$ all of whose edges are green.  Because we could have said our colors in
the opposite order, we may conclude that $R(m,n) = R(n,m)$.  In particular
$R(n,n)$ is the smallest number $R$ such that if we color the edges of a
$K_R$ with two colors, then our picture contains a $K_n$ all of whose edges
have the same color. 

\bp 
\itemm Since $R(3,3)=6$, an uneducated guess might be that $R(4,4)=8$.   Show
that this is not the case. 
\solution{In the graph
\begin{center}\vglue-.75in\mbox{\psfig{figure=NonRamsey8.eps%,height=1.0in
}}
\end{center}
each vertex has three dashed lines emanating from it, and there are no dashed
lines connecting any of the three vertices adjacent to it by dashed lines. 
Each vertex has four solid lines emanating from it, and no three of the four
vertices adjacent to it by solid lines are all adjacent by solid lines.  Thus
there is no solid line $K_4$ and there is no dashed line $K_4$.}

\itemes Show that among ten people, there are either four mutual acquaintances
or three mutual strangers.  What does this say about $R(4,3)$?
\solution{Take a person, say person 1.  If person has six acquaintances, then
by Problem \ref{R(3,3)} among them there are either three mutual strangers,
in which case we are done, or three mutual acquaintances.  These three
acquaintances together with person 1 form a set of 4 mutual acquaintances in
which case we are again done.  Thus we may assume Person 1 has at most 5
acquaintances, and so has four non-acquaintances.  Now either all four of
these people are acquainted, in which case we are done, or else two of them
are not acquainted.  Then these two people, together with person 1 make three
mutual nonacquaintances.  Therefore in every possible case, we have either
four mutual acquaintances or three mutual strangers.  This means that $R(4,3)
\le 10$.}

\itemes Show that among an odd number of people there is at least one person
who is an acquaintance of an even number of people and therefore also a
stranger to an even number of people.\label{OddNoPeople}
\solution{Suppose we add, for each person, the number of people with whom he
or she is acquainted.  Then we get twice the number of acquaintance edges in
the graph of acquaintance and non-acquaintance relationships.  Thus the sum
must be even.  But if each person among an odd number of people were
acquainted with an odd number of people, then the sum would be odd.  Since
this is a contradiction, among an odd number of people, there must be at
least one who is acquainted with an even number of people.  Since the number
of people different from this person is even, the number of people with whom
this person is not acquainted is also even.}

\itemes Find a way to color the edges of a $K_8$ with red and green so that
there is no red $K_4$ and no green $K_3$.\label{R(4,3)not8}
\solution{In the graph
\begin{center}\vglue-.75in\mbox{\psfig{figure=NonRamsey8.eps%,height=1.0in
}}
\end{center}
there is no $K_3$ whose edges are dashed, and no $K_4$ whose edges are
solid.  By symmetry, to verify this you need only look at vertex 1 and
vertices connected to it by either dashed lines or by solid lines.}

\itemesi Find $R(4,3)$.
\solution{$R(4,3)=9$. In Problem \ref{R(4,3)not8} we showed that $R(4,3)$ is
more than 8.  So we must show that if we have nine people, we either have 4
mutual acquaintances or three mutual strangers.  By Problem \ref{OddNoPeople}
there is at least one person (say person A) who is acquainted with an even
number of people.  If person A is acquainted with six or more people, then
among these six people, there are either three mutual acquaintances or three
mutual strangers.  If there are three mutual strangers, we are done; if there
are three mutual acquaintances, they, together with Person A
are four mutual acquaintances.  Thus we may assume Person A is acquainted
with at most four people.  Thus person A is a stranger to at least four
people.  If two of these people are strangers, then they, together with
person A form three mutual strangers and we are done.  Otherwise all of these
people know each other and we have at least four mutual acquaintances, and so
in every possible situation, we have either four mutual acquaintances or
three mutual strangers.}
\ep

As of this writing, relatively few Ramsey Numbers are known.  $R(3,n)$ is
known for $n<10$, $R(4,4) = 18$, and $R(5,4)=R(4,5)=25$.


\section{Supplementary Chapter Problems}
\begin{enumerate}
\itemi Remember that we can write $n$ as a sum of $n$ ones.  How many plus
signs do we use?  In how many ways may we write $n$ as a sum of a list of $k$
positive numbers?  Such a list is called a {\em
composition}\index{composition} of $n$ into $k$ parts.\index{composition!$k$
parts}\index{composition!$k$ parts!number of}
\label{compositiondefinition}
\solution{We use $ n-1$ plus signs.  Write down such a sum and choose $k-1$
of the plus signs.  Then each string of ones and plusses between two chosen
plus signs, before the first chosen plus sign or after the last chosen one
corresponds to a part of a composition of
$n$.  Thus the number of compositions of $n$ with $k$ parts is the number of
ways to choose the $k-1$ places, which is
$n-1\choose k-1$.}

\item In Problem \ref{compositiondefinition} we defined a composition of $n$
into $k$ parts.  What is the total number of compositions of $n$ (into any
number of parts).\label{composition:numberof}\index{compositions!number of}
\solution{The total number of compositions is the number of ways to choose a
subset of the plus signs which is $2^{n-1}$.}

\itemes Write down a list of all 16 0-1 sequences of length four starting
with 0000 in such a way that each entry differs from the precious one by
changing just one digit.  This is called a Grey Code.\index{Grey Code}  That
is, a {\em Grey Code} for 0-1 sequences of length $n$ is a list of the
sequences so that each entry differs from the previous one in exactly one
place.  Can you describe how to get a Grey Code for 0-1 sequences of length
five from the one you found for sequences of length 4?  Can you describe how
to prove that there is a Grey code for sequences of length
$n$?\label{GreyCode}
\solution{(One of many) 0000, 
0001, 
0011, 
0010, 
0110, 
0111, 
0101, 
0100, 
1100, 
1101, 
1111, 
1110, 
1010, 
1011, 
1001, 
1000.  To get a code for sequences of length 5, put a zero at the end of each
of the sequences we have.  Follow that revised sequence by 10001, and write
the remainder of the sequence in reverse order with a 1 at the end of each
term.  (Don't reverse the individual length four sequences, just the sequence
of sequences!) We just, in essence, described the inductive step of an
inductive proof that Grey Codes exist for sequences of any length. } 

\itemi Use the idea of a Grey Code from Problem \ref{GreyCode} to prove
bijectively that the number of even-sized subsets of an $n$-element set
equals the number of odd-sized subsets of an $n$-element set.
\solution{Each sequence in the Grey Code is the characteristic function of a
set, and the number of elements of the set is the number of ones in the
sequence.  Since each sequence differs in just one place from the preceding
one, the sequences alternate between having an even  number of ones and an
odd number of ones.  Since the first sequence is all zeros and there are
$2^n$ sequences, the last one has an odd number of zeros.  Thus the map that
takes each sequence except the last to the next one, and takes the last to the
first is a bijection between the characteristic functions of sets with an
even number of elements and sets with an odd number of elements.}

\itemi A list of parentheses is said to be balanced if there are the same
number of left parentheses as right, and as we count from left to right we
always find at least as many left parentheses as right parentheses.  For
example, (((()()))()) is balanced and ((()) and (()()))(() are not.  How
many balanced lists of $n$ left and $n$  right parentheses are there?
\solution{The number is the Catalan number: we get a bijection between
balanced lists of parentheses and Catalan paths by sending each left paren to
an upstep and each right paren to a downstep.  The condition that there are
always as many left parentheses as right ensures we never go below the $x$
axis.}

\itemh Suppose we plan to put six distinct computers in a network as shown in
Figure
\ref{hexagonalnetwork}.  The lines show which computers can communicate
directly with which others.  Consider two ways of assigning computers to the
nodes of the network different if there are two computers that communicate
directly in one assignment and that don't communicate directly in the
other.  In how many different ways can we assign computers to the network? 
\begin{figure}[htb]\caption{A computer
network.}\label{hexagonalnetwork}\smallskip
\begin{center}\mbox{\psfig{figure=HexagonalNetwork.eps%,height=1.0in
}}
\end{center}  
\end{figure}
\solution{We consider two assignments of computers to be equivalent if in
both assignments, each computer communicates directly with exactly the same
computers.  This partitions the set of all $6!$ computer assignments into
blocks of $48$ computers each.  Thus we have $720/48=15$ ways to assign the
computers to the network.}

\itemi In a circular ice cream dish we are going to put four distinct scoops
of ice cream chosen from among twelve flavors.  Assuming we place four scoops
of the same size as if they were at the corners of a square, and recognizing
that moving the dish doesn't change the way in which we have put the ice
cream into the dish, in how many ways may we choose the ice cream and put it
into the dish?
\solution{Each ice cream arrangement is equivalent to three others, the ones
we get by rotating the dish.  This divides the arrangements of four flavors
of ice cream into blocks of size 4.  Thus we may arrange the ice cream we have
chosen in the dish in
$4!/4=6$ ways.  We may choose the ice cream in ${12\choose 4}=495$ ways, and
so we may choose it and put it into the dish in 2970 ways.}

\itemi In as many ways as you can, show that ${n\choose k}{n-k\choose m} =
{n\choose m}{n-m\choose k}$.
\solution{You can prove this by plugging in the formula for $n\choose k$ on
both sides and cancelling stuff until you get the same thing on both sides. 
However a much more interesting proof is that the right hand side counts the
number of ways to choose a $k$-element set form an $n$-element set and then
choose an $m$-element set from what remains.  The left hand side counts the
number of ways to first chose a $k$-element subset from the $n$-element set
and then choose an $m$-element subset from what remains.  Thus in both cases
you are counting the number of ways to choose an ordered pair consisting of an
$m$-element subset and a disjoint $k$-element subset from an $n$-element set.
You can also base a proof on the observation that $(x+y+x)^n=
\sum_{k=0}^n{n\choose k}(x+y)^kz^{n-k}$ and
$(x+y+z)^n=\sum_{m=0}^n{n\choose m}x^m(y+z)^{n-m}$ and asking for the
coefficient of $x^my^{n-m-k}z^k$.  You do have to use the binomial theorem
with an eye to the result you are looking for, however.}

\itemi A tennis club has $4n$ members.  To specify a doubles match, we choose
two teams of two people.  In how many ways may we arrange the members into
doubles matches so that each player is in one doubles match?  In how many
ways may we do it if we specify in addition who serves first on each team?
\solution{We now have many methods for solving this problem.  Perhaps the
easiest is to list all $(4n)$ people and take them in groups of four for
doubles matches, with the first two in a group of four as one team and the
second two as another team.  We note that interchanging the $n$ blocks of 4
does not change the matches, nor does interchanging the two people on a
team nor interchanging the two teams.  Thus we have $(4n)!/n!2^{3n}$ ways to
arrange the matches.  If we are to say who serves first on each team, we
might as well say it is the first of the two listed, so now we have
$(4n)!/n!2^n$ ways to arrange the matches.}


\item A town has $n$ streetlights running along the north side of main
street.  The poles on which they are mounted need to be painted so that they
do not rust.  In how many ways may they be painted with red, white, blue, and
green if an even number of them are to be painted green?
\solution{We can think of first choosing the set of even size of poles to
be painted green, and the painting the remaining poles red, white, and blue. 
We may do this in $\sum_{k=0}^{\lfloor n/2\rfloor}{n\choose 2k}3^{n-2k}$
ways.}

\item We have $n$ identical ping-pong balls.  In how many ways may we
paint them red, white, blue, and green?\label{pingpongpaint}
\solution{We can line up the identical ping-pong balls and break them into
four groups, those of each color, by inserting dividers.  If we want to paint
at least one in each color, we can choose three of the spaces between the
balls in which to insert dividers, so we can paint them in $n-1\choose k$.
But the problem didn't require us to use each color, so we can put two
dividers adjacent to each other.  Thus there are $n+1$ places where we can
put the first divider (putting it before all the balls means we use no red,
and putting it after all of them means we use no green.  Now there are $n+2$
places where we can put the second divider, including before or after the
first, and $n+3$ places where we can put the third divider.  However if we
interchange two dividers we still paint the balls before the first divider
red, those between then next two white, and so on.  Thus $3!=6$ of these
arrangements of balls and dividers correspond to the same paint job, so the
number of ways to paint the balls is ${(n+1)(n+2)(n+3)\over6} ={n+3\choose
3}$.  This suggests that another way to think of the problem is to consider
$n+3$ slots in a row, and fill $n$ of them with balls and $3$ of them with
dividers; since the balls are identical and the dividers might as well be
identical, the number of ways to do this is the number of ways to choose the
slots that get dividers.}

\item We have $n$ identical ping-pong balls.  In how many ways may we
paint them red, white, blue, and green if we use green paint on an even
number of them?
\solution{We first decide how many balls to paint green, then paint the
remainder with the other three colors as in Problem \ref{pingpongpaint} This
gives us
$$\sum_{k=0}^{\lfloor n/2\rfloor}{n-2k+2\choose 2}$$ ways to paint the balls.}
\end{enumerate}
%%% Local Variables:
%%% mode: latex
%%% TeX-master: "MainProbsNotesCombinatorics"
%%% End:
