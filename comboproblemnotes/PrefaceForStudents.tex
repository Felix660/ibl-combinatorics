\chapter{Preface}

This book is an introduction to combinatorial mathematics, also
known as combinatorics. The book focuses especially but not
exclusively on the part of combinatorics that mathematicians refer
to as ``counting."  The book consist almost entirely of problems. 
Some of the problems are designed to lead you to think about a
concept, others are designed to help you figure out a concept and
state a theorem about it, while still others ask you to prove the
theorem. Other problems give you a chance to use a theorem you have
proved. From time to time there is a discussion that pulls together
some of the things you have learned or introduces a new idea for you
to work with.  Many of the problems are designed to build up your
intuition for how combinatorial mathematics works.  There are
problems that some people will solve quickly, and there are problems
that will take days of thought for everyone.  Probably the best way
to use this book is to work on a problem until you feel you are not
making progress and then go on to the next one.  Think about the
problem you couldn't get as you do other things.  The next chance you
get, discuss the problem you are stymied on with other members of the
class.  Often you will all feel you've hit dead ends, but when you
begin comparing notes and listening {\em carefully} to each other,
you will see more than one approach to the problem and be able to
make some progress. In fact, after comparing notes you may realize
that there is more than one way to interpret the problem.  In this
case your first step should be to think together about what the
problem is actually asking you to do.  You may have learned in
school that for every problem you are given, there is a method that
has already been taught to you, and you are supposed to figure out
which method applies and apply it.  That is not the case here. 
Based on some simplified examples, you will discover the method for
yourself.  Later on, you may recognize a pattern that suggests you
should try to use this method again.  

The
point of learning from this book is that you are learning how to
discover ideas and methods for yourself, not that you are learning
to apply methods that someone else has told you about. The problems
in this book are designed to lead you to discover for yourself and
prove for yourself the main ideas of combinatorial mathematics. 
There is considerable evidence that this leads to deeper learning
and more understanding.



You will see
that some of the problems are marked with bullets.  Those are the
problems that I feel are essential to having an understanding of
what comes later, whether or not it is marked by a bullet.  The
problems with bullets are the problems in which the main ideas of
the book are developed.  Many of the most interesting problems, in
fact entire sections, are not marked in this way, because they use an
important idea rather than developing one.  Other problems are not
marked with bullets because they are designed to provide motivation
for the important concepts, motivation with which some students may
already be familiar.  If you are taking a course, your instructor
will choose other problems (especially among the motivational ones) for you to
work on based on the prerequisites for and goals of the course.  If you are
reading the book on your own, I recommend that you try all the problems in a
section you want to cover.  Make sure you can do the problems with
bullets, but by all means don't restrict yourself to them. 
Sometimes a bulleted problem makes more sense if you have done some
of the easier motivational problems that come before it. Problems that are
motivational in nature or are introductory to the topic at hand are marked
with a small circle. If, after you've tried it, you want to skip over a problem
without a bullet or circle, you
should not miss out on much by not doing that problem.  Also, if you don't
find the problems in a section with no bullets interesting, you can skip them,
understanding that you may be skipping an entire branch of combinatorial
mathematics!  And no matter what, read the textual material that comes before,
between, and immediately after problems you are working on!

You will also see that some problems are marked with arrows.  These
point to problems that I think are particularly interesting.  Some
of them are also difficult, but not all are.  A few problems that summarize
ideas that have come before but aren't really essential are marked with a
plus, and problems that are essential if you want to cover the section they
are in or, perhaps, the next section.  If a problem is relevant to a much
later section in an essential way, I've marked it with a dot and a
parenthetical not e that explains where it will be essential.  Finally,
problems that seem unusually hard to me are marked with an asterisk. Some
I've marked as hard only because I think they are difficult in light of what
has come before, not because they are intrinsically difficult.  In particular,
some of the problems marked as hard will not seem so hard if you come back to
them after you have finished more of the problems.  

One of the downsides of how we learn math in
high school is that many of us come to believe that if we can't solve
a problem in ten or twenty minutes, then we can't solve it at all. 
There will be problems in this book that take hours of hard thought. 
Many of these problems were first conceived and solved by
professional mathematicians, and {\em they} spent days or weeks on
them.  How can you be expected to solve them at all then?  You have
a context in which to work, and even though some of the problems are
so open ended that you go into them without any idea of the answer,
the context and the leading examples that preceded them give you a
structure to work with.  That doesn't mean you'll get them right
away, but you will find a real sense of satisfaction when you see
what you can figure out with concentrated thought.  Besides, you can
get hints!

Some of the questions will appear to be trick questions, especially when you
get the answer.  They are not intended as trick questions at all.  Instead
they are designed so that they don't tell you the answer in advance.  For
example the answer to a question that begins ``How many..." might be ``none." 
Or there might be just one example (or even no examples) for a problem that
asks you to find all examples of something.  So when you read a question,
unless it directly tells you what the answer is and asks you to show it is
true, don't expect the wording of the problem to suggest the answer.  The book
isn't designed this way to be cruel.  Rather, there is evidence that the more
open-ended a question is, the more deeply you learn from working on it.  If
you do go on to do mathematics later in life, the problems that come to you
from the real world or from exploring a mathematical topic are going to be
open-ended problems because nobody will have done them before.  Thus working
on open-ended problems now should help to prepare you to do mathematics later
on. 

You should try to write up answers to all the problems that you work
on.  If you claim something is true, you explain why it is true;
that is you should prove it.  In some cases an idea is introduced
before you have the tools to prove it, or the proof of something
will add nothing to your understanding.  In such problems there is a
remark telling you not to bother with a proof.
\iffalse Then you should ask other members of the class who have
written up their solutions to the same problems to read yours and
really try to understand it.  When you are reading someone else's
solution, your goal should be to help them make their ideas clear to
you.  You do this by asking about things that you are having trouble
reading.  The point is not whether you can see how to do the 
problem after reading what your classmate has written, but whether
you understand how your classmate has solved the problem, and how
his or her words are explaining the solution to you.  Everyone in
the class is trying to learn to do combinatorial mathematics, and on
tests and on problems your instructor chooses to grade, your
explanations have to be clear.  This means\fi When you write up a
problem, remember that the instructor has to be able to ``get'' your
ideas and understand exactly what you are saying.  Your instructor is
going to choose some of your solutions to read carefully and give
you detailed feedback on.  When you get this feedback, you should
think it over carefully and then write the solution again!   You may
be asked not to have someone else read your solutions to some of
these problems until your instructor has.   This is so that the
instructor can offer help which is aimed at your needs.  On other
problems it is a good idea to seek feedback from other students.     One of
the best ways of learning to write clearly is to have someone who is as
easily confused as you are point out to you where it is hard to
figure out what you mean.  

As you work on a problem, think about why you are doing what you are
doing.  Is it helping you?  If your current approach doesn't feel right,
try to see why.  Is this a problem you can decompose into simpler
problems?  Can you see a way to make up a simple example, even a silly
one, of what the problem is asking you to do?  If a problem is asking you
to do something for every value of an integer $n$, then what happens with
simple values of $n$ like 0, 1, and 2?  Don't worry about making
mistakes; it is often finding mistakes that leads mathematicians to their
best insights.  Above all, don't worry if you can't do a problem.  Some
problems are given as soon as there is one technique you've learned that
might help do that problem.  Later on there may be other techniques that
you can bring back to that problem to try again.  The notes have been
designed this way on purpose.  If you happen to get a hard problem with
the bare minimum of tools, you will have accomplished much.  As you
go along, you will see your ideas appearing again later in other
problems.  On the other hand, if you don't get the problem the first
time through, it will be nagging at you as you work on other things,
and when you see the idea for an old problem in new work, you will
know you are learning. 

There are quite a few concepts that are developed in this book. 
Since most of the intellectual content is in the problems, it is
natural that definitions of concepts will often be within problems. 
When you come across an unfamiliar term in a problem, it is likely
it was defined earlier.  Look it up in the index, and with luck
(hopefully no luck will really be needed!) you will be able to find
the definition.

Above all, this book is dedicated to the principle that doing
mathematics is fun.  As long as you know that some of the problems are
going to require more than one attempt before you hit on the main idea,
you can relax and enjoy your successes, knowing that as you work more and
more problems and share more and more ideas, problems that seemed
intractable at first become a source of satisfaction later on.

The development of this book is supported by the National Science
Foundation.  An essential part of this support is an advisory board
of faculty members from a wide variety of institutions who have made
valuable contributions.  They are Karen Collins, Wesleyan University,
Marc Lipman, Indiana University/Purdue University, Fort Wayne,
Elizabeth MacMahon, Lafayette College, Fred McMorris, Illinois
Institute of Technology, Mark Miller, Marietta College, Rosa
Orellana, Dartmouth College, Vic Reiner, University of Minnesota,
and Lou Shapiro, Howard University.  Professors Reiner and Shapiro
are responsible for the overall design and most of the problems in
the appendix on exponential generating functions.  I believe the
board has managed both to make the book more accessible and more
interesting.