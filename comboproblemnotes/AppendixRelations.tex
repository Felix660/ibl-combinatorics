\appendix 
\chapter{Relations}
\label{Relations}

\section{Relations as sets of Ordered Pairs}
\subsection{The relation of a function}\label{functionrelation}
%\begin{enumerate}
\label{relationsandfunctions}
\bp

\item Consider the functions from $S=\{-2,-1,0,1,2\}$ to
$T=\{1,2,3,4,5\}$ defined by
$f(x) = x+3$, and $g(x) = x^5-5x^3+5x +3$.  Write down the set
of ordered pairs $(x,f(x))$ for $x \in S$  and the set of
ordered pairs
$(x,g(x))$ for $x \in S$.  Are the two functions the same or
different?
\solution{We get $\{(-2,1),(-1,2),(0,3),(1,4),(2,5)\}$ in both
cases, so the functions are the same.}
\label{functionsasorderedpairs}

\ep

Problem
\ref{functionsasorderedpairs} points out how two functions
which appear to be different are actually the same on some
domain of interest to us.   Most of the time when we are
thinking about functions it is fine to think of a function
casually as a relationship between two sets.  In Problem
\ref{functionsasorderedpairs} the set of ordered pairs you
wrote down for each function is called the \index{relation!of a
function}\index{function!relation of}{\em relation} of the function. 
When we want to distinguish between the casual and the careful in
talking about relationships, our casual term will be ``relationship''
and our careful term will be ``relation.''  So {\em relation}
 is a technical word in mathematics, and as such it has
a technical definition.  A \index{relation}{\em relation} from a set
$S$ to a set $T$ is a set of ordered pairs whose first elements are in
$S$ and whose second elements are in $T$.  Another way to say this is
that a {\em relation} from $S$ to $T$ is a subset of $S\times T$.

A typical way to define a {\em function}\index{function} $f$ from a
set
$S$ to a set
$T$ is that $f$ is a relationship between $S$ to $T$ that relates
one and only one member of $T$ to each element of $X$. We use
$f(x)$ to stand for the element of $T$ that is related to the
element $x$ of $S$.  If we wanted to make our definition more
precise, we could substitute the word ``relation'' for the word
``relationship'' and we would have a more precise definition.  For
our purposes, you can choose whichever definition you prefer. 
However, in any case, there is a relation associated with each
function.  As we said above, the relation of a function $f:
S\rightarrow T$ (which is the standard shorthand for ``$f$ is a
function from $S$ to $T$'' and is usually read as $f$ {\em maps}
$S$ to $T$) is the set of all ordered pairs
$(x,f(x))$ such that $x$ is in $S$.  


\bp
\item Here are some questions that will help you get used to
the formal idea of a relation and the related formal idea of a
function.  $S$ will stand for a set of size
$s$ and
$T$ will stand for a set of size $t$.\label{formalrelations} 

\begin{enumerate}
\item What is the size of the largest relation from $S$ to $T$?
\solution{$st$ because that is the size of the relation that has
all the ordered pairs $(x,y)$ with $x\in S$ and $y\in T$.}
\item What is the size of the smallest relation from $S$ to
$T$?\solution{0, because the empty set of ordered pairs is a
relation.}
\iffalse
\item  How many relations are there from the set
$S$ to the set
$T$?
\fi
\item The relation of a function $f:S\rightarrow T$ is the set
of all ordered pairs $(x,f(x))$ with $x\in S$.  What is the
size of the relation of a function from $S$ to $T$?  That is, how many
ordered pairs are in the relation of a function from $S$ to $T$?
\solution{The size of the relation of $f:S\rightarrow T$ is $s$.}

\item We say $f$ is a {\em 
one-to-one}\index{function!one-to-one}\index{one-to-one
function}\index{injection} function or {\em injection} from
$S$ to
$T$ if each member of $S$ is related to a {\em different} element
of $T$. How many different elements must appear as second elements
of the ordered pairs in the relation of a one-to-one function
(injection) from
$S$ to
$T$?  
\solution{$s$ different elements must appear, one for each element
of $S$.}
\item  A function $f:S\rightarrow T$ is called an \index{onto
function}\index{function!onto}{\em onto function} or
\index{surjection}\index{function!surjection}{\em surjection} if each
element of
$T$ is $f(x)$ for some $x\in S$ What is the minimum size that
$S$ can have if there is a  surjection from
$S$ to
$T$?\label{onto}
\solution{In order to have a surjection from $S$ to $T$, the size
of $S$ must be at least $t$.}
\end{enumerate}

\item When $f$ is  a function from $S$ to $T$, the sets $S$ and
$T$ play a big role in determining whether a function is
one-to-one or onto (as defined in Problem \ref{formalrelations}). 
For example, if $S$ and $T$ are both the nonnegative real numbers,
and $f:S\rightarrow T$ is given by $f(x) =x^2$, is $f$
one-to-one?  Is $f$ onto?  Now assume $S'$ is the set of all real
numbers and $g:S'\rightarrow T$ is given by $g(x) = x^2$.  Is $g$
one-to-one?  Is $g$ onto?  Assume that $T'$ is the set of all
real numbers and $h:S\rightarrow T'$ is given by $h(x) = x^2$.  Is
$h$ one-to-one?  Is $h$ onto?  And if the function
$j:S'\rightarrow T'$ is given by $j(x)=x^2$, is $j$ one-to-one? 
Is $j$ onto?
\solution{With $S$ as domain, $f$ is both one-to-one and onto.
The function $g$ is not one-to-one, but it is onto.  The function
$h$ is  one-to-one but not onto.  The function $j$ is neither
one-to-one nor onto.}

\item  If $f:S\rightarrow T$ is a function, we say that $f${\em 
maps}
$x$ to $y$ as another way to say that $f(x)=y$.  Suppose
$S=T=\{1,2,3\}$.  Give a function from
$S$ to
$T$ that is not onto.  Notice that two different members of $S$
have mapped to the same element of $T$.  Thus when we say that $f$
associates one and only one element of $T$ to each element of $S$,
it is quite possible that the one and only one element $f(1)$ that
$f$ maps 1 to is exactly the same as the one and only one element
$f(2)$ that $f$ maps 2 to.
\solution{The function given by $f(x)=1$ for all $x$ in $S$ is
not onto.}  
\ep

\subsection{Directed graphs}\label{relationdigraph} We visualize
numerical functions like $f(x)=x^2$ with their graphs in
Cartesian coordinate systems.  We will call these kinds of graphs
\index{graph!coordinate}{\em coordinate graphs} to distinguish them
from other kinds of graphs used to visualize relations that are
non-numerical. 
\begin{figure}[htb]\caption{The alphabet
digraph.}\label{alphabetdigraph}\smallskip
\begin{center}\mbox{\psfig{figure=alphabetdigraph.eps%,height=1.0in
}}
\end{center}  
\end{figure}In Figure
\ref{alphabetdigraph} we illustrate another kind of graph, a
``directed graph'' or ``digraph'' of the ``comes before in
alphabetical order" relation on the letters $a$, $b$, $c$, and
$d$.  To draw a \index{graph!directed}\index{directed
graph}\index{digraph}{\em directed graph} of a relation on a set
$S$, we draw a circle (or dot, if we prefer), which we call a
\index{vertex}{\em vertex}\index{vertex!of a complete graph}, for
each element of the set, we usually label the vertex with the set
element it corresponds to, and we draw an arrow from the vertex
for $a$
 to that for  $b$ if $a$ is related to $b$, that is, if
the ordered pair
$(a,b)$ is in our relation.  We call such an arrow an \index{edge}{\em
edge}\index{edge! in a digraph} or a {\em directed edge}.  We draw
the arrow from
$a$ to
$b$, for example, because $a$ comes before $b$ in alphabetical
order.  We try to choose the locations where we draw our
vertices so that the arrows capture what we are trying to
illustrate as well as possible.  Sometimes this entails
redrawing our directed graph several times until we think the
arrows capture the relationship well.



We also draw digraphs for relations from a set $S$ to a  set
$T$; we simply draw vertices for the elements of $S$ (usually
in a row) and vertices for the elements of $T$ (usually in a
parallel row) draw an arrow from
$x$ in $S$ to $y$ in $T$ if $x$ is related to $y$.  Notice
that instead of referring to the vertex representing $x$, we
simply referred to $x$.  This is a common shorthand.  Here are
some exercises just to practice drawing digraphs.

\bp
\item Draw the digraph of the ``is a proper subset of" relation
on the set of subsets of a two element set.  How many arrows
would you have had to draw if this problem asked you to draw
the digraph for the subsets of a three-element set?
\solution{
\begin{center}\mbox{\psfig{figure=subsetsof2set.eps%,height=1.0in
}}
\end{center} 
We would need to draw 19 arrows, some of them curved,
for the subsets of a three-element set.}

\item Draw the digraph of the relation from the set \{A, M, P,
S\} to the set \{Sam, Mary, Pat, Ann, Polly, Sarah\}  given by
``is the first letter of."
\solution{\begin{center}
\mbox{\psfig{figure=initialdigraph.eps%,height=1.0in }}
}}
\end{center} }
\ep

\subsection{Equivalence relations}\label{equivalencerelations} So
far we've used relations primarily to talk about functions. 
There is another kind of relation, called an equivalence
relation, that comes up in the counting problems with which we
began.
 In Problem
\ref{icecreaminpints} with three distinct flavors, it was
probably tempting to say there are 12 flavors for the first
pint, 11 for the second, and 10 for the third, so there are
$12\cdot 11\cdot 10$ ways to choose the pints of ice cream. 
However, once the pints have been chosen, bought, and put into a
bag, there is no way to tell which is first, which is second and
which is third.  What we just counted is lists of three
distinct flavors---one to one functions from the set
$\{1,2,3\}$ in to the set of ice cream flavors.  Two of those
lists become equivalent once the ice cream purchase is
made if they list the same ice cream.  In other words, two of
those lists become equivalent (are related) if they list same
subset of the set of ice cream flavors.  To visualize this
relation with a digraph, we would need one vertex for each of
the 
$12\cdot 11\cdot 10$ lists.  Even with
 five flavors of ice cream, we would need one vertex for each
of
$5\cdot4\cdot3=60$ lists.  So for now we will work with the easier
to draw question of choosing three pints of ice cream of different
flavors from four flavors of ice cream.

\bp
\item Suppose we have four flavors of ice cream, V(anilla),
C(hocolate), S(trawberry) and P(each).  Draw the directed
graph whose vertices consist of all lists of three distinct
flavors of the ice cream, and whose edges connect two lists if
they list the same three flavors.  This graph makes it
pretty clear in how many ways we may choose 3 flavors
out of four.  How many is it?\label{fourchoosethree}
\solution{\begin{center}
\mbox{\psfig{figure=flavordigraph.eps,height=2.5in }}
\end{center}
We used double-headed arrows in place of one arrow going in each
direction to reduce the clutter in the picture. Note that there is
an arrow from each vertex to itself. We may choose 3 flavors in
four ways.}

\itemi Now suppose again we are choosing three distinct flavors of
ice cream out of four, but instead of putting scoops in a cone or
choosing pints, we are going to have the three scoops arranged
symmetrically in a circular dish.  Similarly to choosing three
pints, we can describe a selection of ice cream in terms of which
one goes in the dish first, which one goes in second (say to the
right of the first), and which one goes in third (say to the right
of the second scoop, which makes it to the left of the first
scoop).  But again, two of these lists will sometimes be
equivalent.  Once they are in the dish, we can't tell which one
went in first.  However, there is a subtle difference between
putting each flavor in its own small dish and putting all three
flavors in a circle in a larger dish.  Think about what makes the
lists of flavors equivalent, and draw the directed graph whose
vertices consist of all lists of     three of the flavors of ice
cream and whose edges connect two lists that we cannot tell the
difference between as dishes of ice cream.  How many dishes of ice
cream can we distinguish from one another?\label{icecreaminadish}
\solution{\begin{center}
\mbox{\psfig{figure=icecreamindish.eps,%height=2.5in }}
}}\end{center}We can distinguish eight different dishes of ice
cream.}


\item Draw the digraph for Problem \ref{roundtable} in the
special case where we have four people sitting around the table.
\label{roundtablefour} 
\solution{\begin{center}
\mbox{\psfig{figure=4AROUNDATABLE.eps,%height=2.5in }}
}}\end{center}}
\ep

In Problems \ref{fourchoosethree}, \ref{icecreaminadish}, and
\ref{roundtablefour} (as well as Problems
\ref{twelvechoosethree}, \ref{roundtable}, and
\ref{formulanchoosek}) we can begin with a
set of lists, and say when two lists are equivalent as
representations of the objects we are trying to count.  In
particular, in Problems
\ref{fourchoosethree}, \ref{icecreaminadish}, and
\ref{roundtablefour} you drew the directed graph for this
relation of equivalence.  Technically, you should have had an
arrow from each vertex (list) to itself.  This is what we mean
when we say a relation is {\em reflexive}.  Whenever you had an
arrow from one vertex to a second, you had an arrow back to the
first.  This is what we mean when we say a relation is {\em
symmetric}.    

When people sit
around a round table, each list is equivalent to itself:
if List1 and List 2 are identical, then everyone has the same
person to the right in both lists (including the first person in
the list being to the right of the last person).  To see the
symmetric property of the equivalence of seating arrangements, if
List1 and List2 are different, but everyone has the same person to
the right when they sit according to List2 as when they sit
according to List1, then everybody better have the same person to
the right when they sit according to List1 as when they sit
according to List2.  



 In Problems
\ref{fourchoosethree}, \ref{icecreaminadish} and
\ref{roundtablefour} there is another property of those relations
you may have noticed from the directed graph.  Whenever you had
an arrow from
$L_1$ to
$L_2$ and an arrow from
$L_2$ to $L_3$, then there was an arrow from $L_1$ to $L_3$. 
This is what we mean when we say a relation is {\em
transitive}.  You also undoubtedly noticed how the directed
graph divides up into clumps of mutually connected vertices. 
This is what equivalence relations are all about.  Let's be a
bit more precise in our description of what it means for a
relation to be reflexive, symmetric or transitive.  

\begin{itemize} 
\item If $R$ is a relation on a set $X$, we say $R$ is
\index{reflexive}\index{relation!reflexive}{\em reflexive}  if $(x,x)\in
R$ for every
$x\in X$.  
\item If $R$ is a relation on a set $X$, we say $R$ is
\index{symmetric}\index{relation!reflexive}{\em symmetric} if $(x,y)$ is
in
$R$ whenever
$(y,x)$ is in $R$.
\item If $R$ is a relation on a set $X$, we say $R$ is
\index{transitive}\index{relation!transitive}{\em transitive} if
whenever
$(x,y)$ is in
$R$ and
$(y,z)$ is in
$R$, then $(x,z)$ is in
$R$ as well.
\end{itemize}

Each of the relations of equivalence you worked with in the
Problem \ref{fourchoosethree}, \ref{icecreaminadish} and
\ref{roundtablefour}  had these three properties.  Can you
visualize the same three properties in the relations of
equivalence that you would use in problems
\ref{twelvechoosethree}, \ref{roundtable}, and
\ref{formulanchoosek}? We call a relation an {\bf equivalence
relation}\index{equivalence
relation}\index{relation!equivalence} if it is reflexive,
symmetric and transitive. 

 
After some more examples, we will see how to show that
equivalence relations have the kind of clumping property you
saw in the directed graphs.  In our first example, using the
notation $(a,b) \in R$ to say that $a$ is related to $B$ is
going to get in the way.  It is really more common to write $a
R b$ to mean that $a$ is related to $b$. For example, if our
relation is the less than relation on $\{1,2,3\}$, you are
much more likely to use $x<y$ than you are $(x,y)\in \ <$,
aren't you? The reflexive law then says
$xRx$ for every
$x$ in
$X$, the symmetric law says that if $xRy$, then $yRx$, and the
transitive law says that if
$xRy$ and $yRz$, then $xRz$.

\bp 
\item For the necklace problem, Problem \ref{necklace}, our
lists are lists of beads.  What makes two lists equivalent for
the purpose of describing a necklace?  Verify explicitly that
this relationship of equivalence is reflexive, symmetric, and
transitive.
\solution{Two lists are equivalent if I can get one from the other
by some combination of cyclic permutations (putting the last thing
in the list at the front and moving everything else one place
right) and reversals.  The combination could include no
operation at all.  Since it is possible to have no operation, the
relation is reflexive. (Even without the opportunity to do no
operation, if we do two reversals to a list we get the original
list back so it is equivalent to itself.) Suppose we have $n$
beads.  Then if I can get from list $A$ to list $B$ with a cyclic
permutations, then $n-1$ more cyclic permutations give us the
original list.  Also if I get from a list $A$ to a list $B$ by a
reversal, then another reversal takes $B$ to $A$.  Thus any
sequence of cyclic permutations and reversals can be undone. 
Therefore if list $A$ is equivalent to list $B$, then list $B$ is
equivalent to list $A$.  Following one combination of operations
with another one still gives a combination of operations, so our
relation is transitive.}

\item  Which of the reflexive, symmetric and transitive
properties does the $<$ relation on the integers have?
\solution{It is transitive, but not reflexive or symmetric.}

\item A relation $R$ on the set of ordered pairs of positive
integers that you learned about in grade school in another
notation is the relation that says $(m,n)$ is related to
$(h,k)$ if $mk =hn$.  Show that this relation is an
equivalence relation.  In what context did you learn about
this relation in grade school?
\solution{$mn=mn$ so the relation is reflexive.  If $mk=hn$, then
$hn=mk$, so if $(m,n)$ is related to $(h,k)$, then $(h,k)$ is
related to $(m,n)$. If $(m,n)$ is related to $(h,k)$ and $(h,k)$
is related to $(p,q)$, then $mk=hn$ and $hq=pk$, which gives us
$mkhq=hnpk$, and cancelling $h$ and $k$ gives us $mq=np=pn$, so
$(m,n)$ is related to $(p,q)$.  Therefore, the relation is
transitive.  This is the relation of equality of the fractions
$m\over n$ and $h\over k$.}

\item  Another relation that you may have learned about in school,
perhaps in the guise of ``clock arithmetic," is the
relation of equivalence modulo
$n$.  For integers (positive, negative, or zero) $a$ and $b$,
we write $a
\equiv b \pmod{n}$ to mean that $a-b$ is an integer multiple
of $n$, and in this case, we say that $a$ is \index{congruence
modulo $n$}{\em congruent to
$b$ modulo $n$} and write $a\equiv b \pmod{n}$..  Show that the
relation of congruence modulo
$n$ is an equivalence relation.
\solution{$a-a=0=0\cdot n$, so $a\equiv a\pmod{n}$.  Thus the
relation is reflexive.  If $a-b=kn$ for some integer $k$, then
$b-a=-kn$, and -k is an integer, so if $a\equiv b \pmod{n}$, then
$b\equiv a \pmod{n}$.  If $a-b=kn$ and $b-c= jn$, then
$a-b+b-c=kn+jn$, so $a-c=(k+j)n$ and since $k+j$ is an integer
this means that $a\equiv c\pmod{n}$.  Therefore the relation of
congruence mod $n$ is an equivalence relation.}

\item  Define a relation on the set of all lists of $n$ distinct
integers chosen from $\{1,2,\ldots, n\}$, by saying two lists are
related if they have the same elements (though perhaps in a
different order) in the first
$k$ places, and the same elements (though perhaps in a
different order) in the last $n-k$ places.  Show this
relation is an equivalence relation.\label{nchoosekanotherway}
\solution{The relation is reflexive, for a list $L$ has the same
elements as the list $L$ in the first $k$ places and the last
$n-k$ places.  If $L_1$ and $L_2$ have the same elements in the
first $k$ places and have the same elements in the last $k$
places, then $L_2$ and $L_1$ have the same elements in the first
$k$ places and have the same elements in the last $n-k$ places,
so our relation is symmetric.  If $L_1$ and $L_2$ have the same
elements in the first
$k$ places and $L_2$ and $L_3$ have the same elements in the first
$k$ places, then $L_1$ and $L_3$ have the same elements in the
first
$k$ places.  Similarly with the last $n-k$ places.  Therefore our
relation is transitive, and so it is an equivalence relation.}

\item Suppose that $R$ is an equivalence relation on a set $X$
and for each $x\in X$, let $C_x = \{y| y\in X \mbox{ and }
yRx\}$.  If $C_x$ and
$C_z$ have an element $y$ in common, what can you conclude
about $C_x$ and $C_z$ (besides the fact that they have an
element in common!)?  Be explicit about what property(ies) of
equivalence relations justify your answer.  Why is every
element of
$X$ in some set
$C_x$?  Be explicit about what property(ies) of equivalence
relations you are using to answer this question.  Notice that
we might simultaneously denote a set by $C_x$ and $C_y$. 
Explain why the union of the sets $C_x$ is $X$.  Explain why
two distinct sets $C_x$ and $C_z$ are disjoint.  What do these
sets have to do with the ``clumping'' you saw in the digraph
of Problem \ref{fourchoosethree} and
\ref{icecreaminadish}?\label{equivalenceclasses}
\solution{If $C_x$ and $C_y$ have the  element $z$ in common,
then by symmetry and transitivity, all elements in $C_x$ are
related to
$z$ and by symmetry and  transitivity, all elements in $C_y$ are
related to
$z$.  Then by symmetry and transitivity again, all elements of
$C_y$ are related to $x$, so $C_y\subseteq C_x$.  By the same kind
of reasoning, $C_x\subseteq C_y$.  Therefore, $C_x=C_y$. Every
element $x$ is in the set $C_x$ by reflexivity Thus the union of
the sets $C_x$ is $X$.  The sets
$C_x$ and $C_y$ are disjoint if they are different, because if
they have a common element $z$ then they are equal. By definition,
the sets $C_x$ form a partition of $X$.  The clumps that we saw in
those problems are the blocks of the partition.}
\ep

In Problem \ref{equivalenceclasses} the sets $C_x$ are called
\index{equivalence class}{\em equivalence classes} of the equivalence
relation
$R$.  You have just proved that if
$R$ is an equivalence relation of the set $X$, then each
element of $X$ is in exactly one equivalence class of $R$.  
Recall that a \index{partition (of a set)}{\em partition} of a set
$X$ is a set of disjoint sets whose union is $X$.  For example,
$\{1,3\}$,
$\{2,4,6\}$, $\{5\}$ is a partition of the set
$\{1,2,3,4,5,6\}$.  Thus another way to describe what you
proved in Problem \ref{equivalenceclasses} is the following:
\begin{theorem} If $R$
is an equivalence relation on $X$, then the set of equivalence
classes of $R$ is a partition of $X$.
\end{theorem}
  Since a partition of
$S$ is a set of subsets of $S$, it is common to call the
subsets into which we partition $S$ the \index{block of a
partition}{\em blocks} of the partition so that we don't find ourselves
in the uncomfortable position of referring to a set and not being
sure whether it is the set being partitioned or one of the
blocks of the partition.


\bp 
\item In each of Problems \ref{roundtable}, \ref{formulanchoosek},
\ref{necklace},
\ref{fourchoosethree},
and
\ref{icecreaminadish},
 what does an equivalence
class correspond to? (Five answers are expected here.)
\solution{In Problem \ref{roundtable} the equivalence classes
correspond to seating arrangements.  In Problem
\ref{formulanchoosek} the equivalence classes correspond to the
$k$-element subsets of our $n$-element set $S$.  In Problem
\ref{necklace}, the equivalence classes correspond to necklaces. 
In Problem \ref{fourchoosethree} the equivalence classes
correspond to choices of three flavors of ice cream out of a
possible four flavors.  In Problem \ref{icecreaminadish} the
equivalence classes correspond to the ways we can choose 
scoops of ice cream of three different flavors out of four and put
them into a dish in a symmetric fashion.}

\item  Given the partition $\{1,3\}$, $\{2,4,6\}$, $\{5\}$ 
of the set
$\{1,2,3,4,5,6\}$, define two elements of $\{1,2,3,4,5,6\}$ to
be related if they are in the same part of the partition. 
That is, define 1 to be related to 3 (and 1 and 3 each related
to itself), define 2 and 4, 2 and 6, and 4 and 6 to be related
(and each of 2, 4, and 6 to be related to itself), and define
5 to be related to itself.  Show that this relation is an
equivalence relation.
\solution{We have said for each element of our set that it is
related to itself, so the relation is reflexive.  If $x$ and $y$
are in a given one of those sets, then $y$ and $x$  are in that
same given set.  If $x$ and $y$ are in the same set, and if $y$
and $z$ are in the same set, then $x$ and $z$ must be in the same
set because there is one and only one set that $y$ is in.  Thus
the relation is an equivalence relation.}

\item Suppose $P = \{S_1, S_2, S_3, \ldots, S_k\}$ is a
partition of
$S$.  Define two elements of $S$ to be related if they are in
the same set $S_i$, and otherwise not to be related.  Show
that this relation is an equivalence relation.  Show that the
equivalence classes of the equivalence relation are the sets
$S_i$.\label{partitiontoequivalence}
\solution{Each element is in a set $S_i$ with itself, so the
relation is reflexive.   If $x$ and $y$
are in a given one of those sets $S_i$, then $y$ and $x$  are in
that same set $S_i$.  If $x$ and $y$ are in the same set $S_i$, and
if
$y$ and $z$ are in the same set $S_j$, then $S_i$ must equal
$S_j$ because $y$ is in one and only one block of the partition.
 Therefore, $x$ and
$z$ must be in the same set $S_i$.  Thus the relation is an
equivalence relation. If $x\in S_i$, then by definition $S_i$
consists of all elements related to $x$, so it is the equivalence
class containing $x$.}
\ep

In Problem \ref{partitiontoequivalence} you just proved that
each partition of a set gives rise to an equivalence relation
whose classes are just the parts of the partition.  Thus in
Problem
\ref{equivalenceclasses} and Problem
\ref{partitiontoequivalence} you proved the following Theorem.

\begin{theorem} A relation $R$ is an equivalence relation on a
set $S$ if and only if
$S$ may be partitioned into sets $S_1$, $S_2$, \ldots, $S_n$
in such a way that $x$ and $y$ are related by $R$ if and only
if they are in the same block $S_i$ of the partition.
\end{theorem}\index{equivalence
relation}\index{relation!equivalence}


In Problems \ref{fourchoosethree}, \ref{icecreaminadish},
\ref{roundtable} and \ref{necklace} what we were doing in
each case was counting equivalence classes of an equivalence
relation.  There was a special structure to the problems that
made this somewhat easier to do.  For example, in
\ref{fourchoosethree}, we had $4\cdot3\cdot2 =24 $ lists of
three distinct flavors chosen from V, C, S, and P.  Each list
was equivalent to  $3\cdot2\cdot1=3!=6$ lists, including
itself, from the point of view of serving 3 small dishes of
ice cream.  The order in which we selected the three flavors
was unimportant.  Thus the set of all
$4\cdot3\cdot2$ lists was a union of some number $n$ of
equivalence classes, each of size 6.  By the product
principle, if we have a union of $n$ disjoint sets, each of
size 6, the union has $6n$ elements.  But we already knew that
the union was the set of all 24 lists of three distinct
letters chosen from our four letters.  Thus we have $6n=24$,
or
$n=4$ equivalence classes.

In Problem \ref{icecreaminadish} there is a subtle change.  In
the language we adopted for seating people around a round
table, if we choose the flavors V, C, and S, and arrange them
in the dish with C to the right of V and S to the right of C,
then the scoops are in different relative positions than if we
arrange them instead with S to the right of V and C to the
right of S.  Thus the order in which the scoops go into the
dish is somewhat important---somewhat, because putting in V
first, then C to its right and S to its right is the same as
putting in S first, then V to its right and C to its right. 
In this case, each list of three flavors is equivalent to
only three lists, including itself, and so if there are
$n$ equivalence classes, we have $3n=24$, so there are $24/3=8$
equivalence classes.

\bp
\item If we have an equivalence relation that divides a set
with $k$ elements up into  equivalence classes each of size $m$, what
is the number $n$ of equivalence classes? 
Explain why.\label{EquivPrincipleProblem}
\solution{The number of equivalence classes is $k/m$, because by
the product principle, $mn=k$.}
\item In Problem \ref{nchoosekanotherway}, what is the number
of equivalence classes?  Explain in words the relationship
between this problem and the Problem
\ref{formulanchoosek}.\label{twowaynchoosek}
\solution{There are $n!$ lists, and each is in an equivalence
class of size $k!(n-k)!$, so the number of equivalence classes is
$n!\over k!(n-k)!$ by Problem \ref{EquivPrincipleProblem}.  This
is a way of computing the number of $k$-element subsets that shows
why the final answer we got in Problem \ref{formulanchoosek} is
symmetric in $k$ and $n-k$.}

\item Describe explicitly what makes two lists of beads
equivalent in Problem \ref{necklace} and how Problem
\ref{EquivPrincipleProblem} can be used to compute the number
of different necklaces.
\solution{Two lists are equivalent if I can get one from the other
by some combination of cyclic shifts and reversals.  A cyclic
shift on the list $a_1,a_2,\ldots,a_{n-1},a_n$ gives either the
list $a_n,a_1,a_2,\ldots ,a_{n-1}$ or the list
$a_2,\ldots,a_{n-1},a_n,a_1$.  There are $n$ possible results of
repeated cyclic shifts, and each of them may be reversed to give a
new list if $n\ge 3$.  Further, these are the only lists we can
get from shifts and reversals.  ($a_1$ must go to one of $n$
places, and that leaves two choices for where $a_2$ goes.  Then
the rest of the list is determined.)  Thus we can get exactly $2n$
lists from combinations of cyclic shifts and reversals.  We define
two lists to be equivalent if they give the same necklace; we've
seen that this is an equivalence relation and that it has $2n$
elements per equivalence class.  Since there are $n!$ lists, this
gives us $(n-1)!/2$ equivalence classes, or necklaces.}

\item What are the equivalence classes (write them out as sets
of lists) in Problem \ref{twocolorsofbeads}, and why can't we
use Problem
\ref{EquivPrincipleProblem} to compute the number of
equivalence classes?
\solution{The equivalence classes are $$\{RRBB, BRRB, BBRR, RBBR\}
\mbox{~and~}
\{RBRB, BRBR\}.$$  We can't use Problem
\ref{EquivPrincipleProblem} to compute the number of
equivalence classes because the equivalence classes don't have the
same size.}

\ep In Problem \ref{EquivPrincipleProblem} you proved our next
theorem.  In Chapter 1 (Problem \ref{quotientprinciple}) we
discovered and stated this theorem in the context of partitions
and called it the
\index{quotient principle}\index{principle!quotient}{\em
Quotient Principle}
\begin{theorem} If an equivalence relation on a set $S$ size
$k$ has $n$ equivalence classes each of size $m$, then the
number of equivalence classes is $k/m$.\end{theorem}
