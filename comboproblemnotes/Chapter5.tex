\chapter{Groups acting on sets}

\label{groupsonsets}
\section{Permutation Groups}
Until now we have thought of permutations mostly as ways of listing the
elements of a set.  In this chapter we will find it very useful to think
of permutations as functions.  This will help us in using permutations to
solve enumeration problems that cannot be solved by the quotient
principle because they involve counting the blocks of a partition in
which the  blocks don't have the same size.  We begin by studying the
kinds of permutations that arise in situations where we have used the quotient
principle in the past.

\subsection{The rotations of a square}
\begin{figure}[htb]\caption{The four
possible results of rotating a square
and maintaining its position.}\label{RotationsOfSquare}\smallskip
\begin{center}\mbox{\psfig{figure=RotationsOfSquare.eps,width = 5.4in
}}
\end{center}  
\end{figure} 

In Figure \ref{RotationsOfSquare} we show a square with its four vertices
labelled 1, 2, 3, and 4.  We have also labeled the spot in the plane
where each of these vertices falls with the same label.  Then we have
shown the effect of rotating the square clockwise through 90, 180, 270, and 360
degrees (which is the same as rotating through 0 degrees).  Underneath
each of the rotated squares we have named the function that carries
out the rotation.  We use $\rho$, the Greek letter pronounced ``row," to
stand for a 90 degree clockwise rotation.  We use $\rho^2$ to stand for two 90
degree rotations, and so on.  We can think of the function $\rho$ as a
function on the four element set\footnote{What we are doing is
restricting the rotation $\rho$ to the set $\{1,2,3,4\}$.}
$\{1,2,3,4\}$.  In particular, for any function $\varphi$ (the Greek letter
phi, usually pronounced ``fee," but sometimes ``fie") from the plane back
to itself that may move the square around but otherwise leaves it in the
same place, we let $\varphi(i)$ be the label of the place where vertex 
previously in
position
$i$ is now.  Thus
$\rho(1) =2$, $\rho(2)=3$, $\rho(3)=4$ and $\rho(4) =1$.  Notice that
$\rho$ is a permutation on the set $\{1,2,3,4\}$.
\bp
\itemm Find $\rho^2$ of 1, 2, 3, and 4.  Find $\rho^3$ of 1, 2, 3, and 4.
Are $\rho^2$ and $\rho^3$ permutations of $\{1,2,3,4\}$?
\solution{$\rho^2(1)=3$, $\rho^2(2)=4$, $\rho^2(3)=1$, $\rho^2(4) =2$.  $\rho^3(1)=4$,
$\rho^3(2)=1$, $\rho^3(3)=2$, $\rho^3(4)=3$.  Yes, $\rho^2$ and $\rho^3$ are
permutations of $\{1,2,3,4\}$.
}

\iteme The composition\index{composition of
functions}\index{functions!composition of}
$f\circ g$ of two functions
$f$ and
$g$ is defined by $f\circ g(x) = f(g(x))$.  Is $\rho^3$ the composition of
$\rho$ and
$\rho^2$?  Does the answer depend on the order in which we write $\rho$
and
$\rho^2$? How is
$\rho^2$ related to
$\rho$? \label{composition1}
\solution{Yes, $\rho^3$ is the composition of $\rho$ and $\rho^2$, and also of $\rho^2$
and $\rho$, so it doesn't matter in which order we write them.  $\rho^2=\rho\circ\rho$.
}

\iteme Is the composition of two permutations always a permutation?
\solution{Yes, because the composition of one-to-one functions is one-to-one and the
composition of onto functions is onto.
}

\ep 

In Problem \ref{composition1} you see that we can think of
$\rho^2\circ\rho$ as the result of first rotating by 90 degrees and then
by another 180 degrees.  In other words, the composition of two rotations
is the same thing as first doing one and then doing the other. If we
first rotate by 90 degrees and then by 270 degrees then we have rotated
by 360 degrees, which does nothing visible to the square.  Thus we say
that $\rho^4$ is the ``identity function.''  In general the {\bf identity
function}\index{function!identity}\index{identity function} on a set $S$,
denoted by
$\iota$ (the Greek letter iota, pronounced eye-oh-ta) is the function
that takes each element of the set to itself.  In symbols, $\iota(x) =x$
for every
$x$ in
$S$.  Of course the identity function on a set is a permutation of that
set. 

\subsection{Groups of Permutations}
\bp
\iteme For any function $\varphi$ from a set $S$ to itself, we define
$\varphi^n$ (for nonnegative integers $n$) inductively by $\varphi^0 =
\iota$ and $\varphi^n = \varphi^{n-1}\circ\varphi$ for every positive
integer $n$.  If $\varphi$ is a permutation, is $\varphi^n$ a permutation?
Based on your experience with previous inductive proofs, what do you
expect $\varphi^n\circ \varphi^m$ to be?  What do you expect
$(\varphi^m)^n$ to be?  There is no need to prove these last two answers
are correct, for you have, in effect, already done so in Chapter 2.
\solution{It is a permutation because the computation of permutations is a
permutation.  (You could be more precise and use the inductive definition
of $\varphi^n$ to prove inductively that $\varphi^n$ is a permutation.)  We
expect
$\varphi^m\circ\varphi^n=\varphi^{m+n}$, and we expect
$(\varphi^m)n=\varphi^{mn}$, and we would prove this by induction.}

\iteme If we perform the composition $\iota\circ \varphi$ for any function
$\varphi$ from $S$ to $S$, what function do we get?  What if we perform the
composition $\varphi\circ\iota$?\label{identityproperty}
\solution{$\iota\circ\varphi=\varphi\circ\iota=\varphi$}
\ep
What you have observed about iota in Problem \ref{identityproperty} is
called the {\em identity property}\index{identity property (for
permutations)} of iota.  In the context of permutations, people usually
call the function
$\iota$ ``the identity'' rather than calling it ``iota.''

Since rotating first by 90 degrees and then by 270 degrees has the same
effect as doing nothing, we can think of the 270 degree rotation as
undoing what the 90 degree rotation does.  For this reason we say that in
the rotations of the square, $\rho^3$ is the ``inverse'' of $\rho$.  In
general, a function $\varphi:T\rightarrow S$ is called an {\bf
inverse}\index{inverse function}\index{function!inverse} of a function $\sigma:S
\rightarrow T$ (the lower case Greek letter sigma) if $\varphi\circ \sigma= \sigma
\circ\varphi =
\iota$. 

\bp
\iteme Does every permutation have an inverse?   If so, why, and is it
unique, i.e. could a permutation have two distinct inverse functions?  If
not, give an example of a permutation without an inverse or a permutation with two
distinct inverses.
\solution{Let $\varphi$ be a permutation of $S$.  Then for each $S$ in
$S$ there is one and only one $x$ in $S$ such that $\varphi(x) = s$
because
$\varphi$ is one-to-one and onto.  Thus if we define $\sigma(s)$ to be
the unique $x$ such that $\varphi(x) = s$, we have defined a function,
and $$\varphi\circ\sigma (s) = \varphi(\sigma(s)) = \varphi(x) = s.$$
Thus we have that $\varphi\circ\sigma = \iota$.  To show that $\sigma$ is
an inverse to $\varphi$, we must show that $\sigma\circ\varphi = \iota$
as well.  However $\sigma(\varphi(y))$ is the unique $x$ such that
$\varphi(x) = \varphi(y)$, that is $\sigma(\varphi(y)) = y$.  Thus
$\sigma\circ\varphi = \iota$, and $\sigma$ is an inverse to $\varphi$.}

 \iteme  Show that if $\sigma$ is an inverse of the permutation
$\varphi$  of
$S$, then $\sigma$ is a permutation of $S$ also. 
\solution{The function $\sigma$ must be one-to-one because if $\sigma(x)=\sigma(y)$,
then
$x=\varphi\circ\sigma(x)=\varphi\circ\sigma(y)=y$.  The function $\sigma$ must be onto
because for each $y$ in its range, $\sigma(\varphi(y)) = y$}

\ep

We use $\varphi^{-1}$ to denote the inverse of the permutation $\varphi$.
We've seen that the rotations of the square are functions that return the
square to its original position but may move the vertices to different
places.  In this way we create permutations of the vertices of the
square.  We've observed three important properties of these permutations.

\begin{itemize}\item (Identity Property)\index{identity property} These
permutations include the identity permutation.  
\item (Inverse Property)\index{inverse property} Whenever these
permutations include
$\varphi$, they also include $\varphi^{-1}$.
\item (Closure Property)\index{closure property}  Whenever these
permutations include
$\varphi$ and $\sigma$, they also include $\varphi\circ\sigma$.
\end{itemize}

A set of permutations with these three properties is called a {\bf
permutation group}\footnote{The concept of a permutation group is a
special case of the concept of a {\em group} that one studies in abstract
algebra.  When we refer to a group in what follows, if you  know what
groups are in the more abstract sense, you may use the word in this way. 
If you do not know about groups in this more abstract sense, then you may
assume we mean permutation group when we say group.} or a group of
permutations.\index{group of permutations}\index{permutation group}  We
will denote the group of permutations corresponding to rotations of the
square  by $R_4$ and call it the {\em rotation group}\index{rotation
group} of the square.  There is a similar rotation group with $n$
elements for any regular $n$-gon.

\bp \iteme If a finite set of permutations  satisfies the closure
property is it a permutation group?
\solution{Yes, because the permutations $\sigma$, $\sigma^2$, \ldots, $\sigma^n$,
\ldots must eventually start repeating.  But if $\sigma^i=\sigma^j$, and $i<j$, then
$\sigma^{j-i} =1$, so $\sigma^{j-i-1}$ is the inverse of $\sigma$.  This means that
every member of the set of permutations has an inverse in that set.  But then since
$\sigma\circ\sigma^{-1} = \iota$, we have that the identity is in the set as well. 
Thus it is a permutation group.}

\iteme \begin{enumerate}
\item How should we define $\varphi^{-n}$ for an element
$\varphi$ of a permutation group? 
\solution{We define $\varphi^{-n}=\left(\varphi^{-1}\right)^n$.}
\item  Will the two standard rules for exponents
$$a^ma^n=a^{m+n} \mbox{\ and\ } (a^m)^n = a^{mn}$$ still hold if one or more of the
exponents may be negative? 
\solution{Yes.}  
\item What would
we have to prove to show that the rules still hold? 
\solution{ To prove this we need to show that for nonnegative $m$ and $n$
we have that
$\varphi^m\circ\varphi^{-n} =
\varphi^{m-n}$, that $\varphi^{-m}\varphi^{-n} = \varphi^{-m-n}$, that
$(\varphi^m)^{-n} = \varphi^{-mn}$, that $(\varphi^{-m})^n = \varphi^{-mn}$ and that
$(\varphi^{-m})^{-n}= \varphi^{mn}$. }
\item If the rules hold, give
enough of the proof to show that you know how to do it; otherwise give a
counterexample.
\solution{
 We give the first and last proof. We will induct
on $n$ in the first proof.  If $n=0$ the formula automatically holds.  Assume inductively
that
$\varphi^m\circ\varphi^{-(n-1)} = \varphi^{m-(n-1)}$.  Compose both sides of this
equation on the right by $\sigma^{-1}$ and use the associative law for composition of
functions to get, in the case that $m-n$ is nonnegative
\begin{eqnarray*}
\varphi^m\circ(\varphi^{-(n-1)}\circ\varphi^{-1}) &=&
\varphi^{m-(n-1)}\circ\varphi^{-1}\\
\varphi^m\circ(\varphi^{-1})^{n-1}\varphi^{-1}&=&
(\varphi^{m-n}\circ\varphi)\circ\varphi^{-1}\\
\varphi^m\circ(\varphi^{-1})^n &=& \varphi^{m-n}.
\end{eqnarray*}
We do the same in the case that $m-n$ is negative, getting
\begin{eqnarray*}
\varphi^m\circ(\varphi^{-(n-1)}\circ\varphi^{-1}) &=&
\varphi^{m-(n-1)}\circ\varphi^{-1}\\
\varphi^m\circ(\varphi^{-1})^{n-1}\varphi^{-1}&=&
(\varphi^{-1})^{n-m-1}\circ\varphi^{-1}\\
\varphi^m\circ(\varphi^{-1})^n &=& (\varphi^{-1})^{n-m}\\
\varphi^m\circ\varphi^{-n}&=&\varphi^{m-n}.
\end{eqnarray*}
To prove the last statement we need to prove, we write the following, in which the line
marked \ref{crucialline} follows from the first equation we proved and the line marked
\ref{modestlyimportant} follows from the fact that $(\varphi^{-1})^{-1} = \varphi$.
\begin{eqnarray}
(\varphi^{-m})^{-n} &=& ([\varphi^{-1})^m]^{-1})^n\nonumber\\
&=&[(\varphi^{-1})^{-m}]^n\label{crucialline}\\
&=&([(\varphi^{-1})^{-1}]^m)^n\nonumber\\
&=&(\varphi^m)^n\label{modestlyimportant}\\
&=&\varphi^{mn}\nonumber
\end{eqnarray}
}\end{enumerate}

\iteme There are three dimensional geometric motions of the square that
return it to its original position but move some of the vertices to other
positions. For example, if we flip the square around a diagonal, most of it moves out of
the plane during the flip, but  the square ends up in the same place.  Draw a figure
like Figure
\ref{RotationsOfSquare} that shows all the possible results of such motions, including
the ones shown in Figure \ref{RotationsOfSquare}.  Do the corresponding permutations
form a group?\label{dihedral1}

\iteme If $f:S\rightarrow T$, $g:T\rightarrow X$, and
$h:X \rightarrow Y$, is $h\circ(g\circ f) = (h\circ g)\circ f$? 
What does this say about the status of the {\em associative
law}\index{associative law}
$$\rho\circ(\sigma\circ \varphi) = (\rho\circ \sigma)\circ\varphi$$ in a
group of permutations?
\solution{Since $(h\circ(g\circ f)) (x) = h((g\circ f)(x)) = h(g(f(x)))$ and $((h\circ
g)\circ f) (x) = (h\circ g)(f(x)) = h(g(f(x)))$, we have that $$h\circ (g\circ f) =
(h\circ g)\circ f.$$  This says that the associative law holds for the composition
operation in a group of permutations.}

\itemei If $\sigma$ and $\varphi$ are permutations, why must
$\sigma\circ\varphi$ have an inverse?  Is $(\sigma\circ\varphi)^{-1}=
\sigma^{-1}\varphi^{-1}$?  (Prove or give a counter-example.)  Is 
$(\sigma\circ\varphi)^{-1}=
\varphi^{-1}\sigma^{-1}$? (Prove or give a counter-example.)
\solution{
One could either say that the composition of one-to-one and onto functions
is one-to-one and onto,  or note that $(\sigma\circ\varphi)\circ (\varphi^{-1}\circ
\sigma^{-1}) =\iota$ by the associative law.  Note that this says that $(\sigma \circ
\varphi)^{-1}= \varphi^{-1}\circ \sigma^{-1}$.  In the dihedral group $D_4$,
$\rho^{-1}= \rho^3$ and $\varphi_{1/3}^{-1} = \varphi{1/3}$.  Thus
\begin{eqnarray*} (\rho\circ \varphi_{1/3})\circ (\rho^{-1}\circ
\varphi_{1/3}^{-1})&=&\rho\circ
\varphi_{1/3})\circ(\rho^3\varphi_{1/3})\\
&=&\varphi_{12/34}\circ \varphi_{14/23}\\
&=&\rho^2.\end{eqnarray*}  This shows that $(\sigma\circ\varphi)^{-1}\not=
\sigma^{-1}\varphi^{-1}$.
}

\iteme Explain why the set of all permutations of four elements is a
permutation group.  How many elements does this group have?  This group
is called the {\em symmetric group on four letters}\index{symmetric
group} and is denoted by $S_4$. 
\solution{It is a finite set of permutations that satisfies the closure property.  It
has $4!= 24$ elements.}
\ep

\subsection{The symmetric group}
In general, the set of all permutations of an $n$-element set is a
group.  It is called the {\bf symmetric group on $\mathbf n$
letters}\index{symmetric group}.  We don't have nice geometric
descriptions (like rotations) for all its elements, and it would be
inconvenient to have to write down something like ``Let $\sigma(1) =3$,
$\sigma(2) =1$, $\sigma(3)=4$, and $\sigma(4)=1$'' each time we need to
introduce a new permutation.  We introduce a new notation for
permutations that allows us to denote them {\em reasonably} compactly and
compose them reasonably quickly.  If
$\sigma$ is the permutation of
$\{1,2,3,4\}$ given by
$\sigma(1)=3$, $\sigma(2)=1$,
$\sigma(3) =4$ and $\sigma(4) =2$,
we write 
$$\sigma =\left( \matrix{1&2&3&4\cr3&1&4&2}\right).$$

We call this notation the {\em two row notation}\index{permutation!two
row notation}\index{two row notation} for permutations.  In the two row
notation for a permutation of $\{a_1,a_2,\ldots, a_n\}$, we write the
numbers $a_1$ through
$a_n$ in a one row and we write $\sigma(a_1)$ through $\sigma(a_n)$ in a
row right below, enclosing both rows in parentheses.  Notice that 
 $$\left( \matrix{1&2&3&4\cr3&1&4&2}\right) = \left(
\matrix{2&1&4&3\cr1&3&2&4}\right),$$
 although the second ordering of the columns is rarely used.

If $\varphi$ is given by
$$\varphi = \pmatrix{1&2&3&4\cr4&1&2&3},$$
then, by applying the definition of composition of functions, we may compute
$\sigma\circ
\varphi$ as shown in Figure
\ref{permutationproduct}. 


\begin{figure}[htb]\caption{How to
multiply permutations in two-row
notation.}\label{permutationproduct}\vspace*{-1in}
%\begin{center}
\mbox{\psfig{figure=productofpermutations.eps
}}
%\end{center}  
\end{figure}

 We don't normally put the circle between two
permutations in two row notation when we are composing them, and refer to
the operation as multiplying the permutations, or as the product of the
permutations. To see how  Figure
\ref{permutationproduct} illustrates composition, notice that the arrow
starting at 1 in
$\varphi$ goes to 4.  Then from the 4 in $\varphi$ it goes to the 4 in
$\sigma$ and then to 2.  This illustrates that $\varphi(1)=4$ and
$\sigma(4) =2$, so that
$\sigma(\varphi(1))=2$.  

\bp
\item For practice, compute $\pmatrix{1&2&3&4&5\cr3&4&1&5&2}
\pmatrix{1&2&3&4&5\cr 4&3&5&1&2}$.
\solution{$\pmatrix{1&2&3&4&5\cr5&1&2&3&4}$}
\ep

\subsection{The dihedral group}
We found four permutations that correspond to rotations of the square. 
In Problem \ref{dihedral1} you found four permutations that correspond to
flips of the square in space.  One flip fixes the vertices in positions 1
and 3 and interchanges those in positions 2 and 4.  (Notice we did not say
it fixes the vertices labelled 1 and 3.)\footnote{There is room for tremendous confusion
here, and professional mathematicians sometimes suffer from it.  Just try to remember
that when we describe the axis of a flip, we describe it relative to points given in
space, not points on the square or other object we are flipping.}  Let us denote it by
$\varphi_{1/3}$.  One flip fixes the vertices in positions 2 and 4 and
interchanges those in positions 1 and 3.  Let us denote it by
$\varphi_{2/4}$.  One flip interchanges the vertices in positions 
1 and 2 and also interchanges those in positions 3 and 4.  Let us denote
it by
$\varphi_{12/34}$. The fourth flip interchanges the vertices in positions
1 and 4 and interchanges those in positions 2 and 3.  Let us denote it by
$\varphi_{14/23}$. Notice that $\varphi_{1/3}$ is a permutation that takes vertex
1 to vertex 1 and vertex 3 to vertex 3, while $\varphi_{12/34}$ is a permutation
that takes the edge from 1 to 2 to the edge from 1 to 2 and takes the edge from 3
to 4 to the edge from 3 to 4.

\bp
\iteme Write down the two-row notation for $\rho^3$, $\varphi_{2/4}$,
$\varphi_{12/34}$ and $\varphi_{2/4}\circ \varphi_{12/34}$.  Remember that $\sigma(i)$
stands for the position where the vertex that originated in position $i$ is after we
apply $\sigma$.
\solution{\par\noindent$\pmatrix{1&2&3&4\cr4&1&2&3}$,
$\pmatrix{1&2&3&4\cr3&2&1&4}$,
$\pmatrix{1&2&3&4\cr2&1&4&3}$, $\pmatrix{1&2&3&4\cr2&3&4&1}$.}
\item (You may have already done this problem in Problem
\ref{dihedral1}, in which case you need not do it again!)  In Problem
\ref{dihedral1}, if a rigid motion of three-dimensional space returns the
square to its original position, in how many places can vertex number one
land?  Once the location of vertex number one is decided, how many
possible locations are there for vertex two?  Once the locations of
vertex one and vertex two are decided, how many locations are there for
vertex three?  Answer the same  question for vertex four.  What does this
say about the relationship between the four rotations and four flips
described above and the permutations you described in Problem
\ref{dihedral1}?\label{dihedral2}
\solution{Vertex number one can go in four places.  After the place for vertex 1 is
chosen, there are two choices for vertex 2.  After vertices 1 and 2 are placed, one
side of the square has been placed, and there is only one way the rest of the square
can fill in the same place the square used to be, so there is only one choice for
where vertex 3 goes and one choice for where vertex 4 goes.  Thus the permutations
described in Problem \ref{dihedral1} are exactly the rotations and flips.}
\ep

The four rotations and four flips of the square described before Problem
\ref{dihedral2} form a group called the dihedral group of the square. 
Sometimes the group is denoted $D_8$ because it has eight elements, and
sometimes the group is denoted by $D_4$ because it deals with four
vertices!  Let us agree to use the notation $D_4$ for the dihedral group
of the square.  There is a similar dihedral group, denoted by $D_{n}$, of
all the rigid motions of three-dimensional space that return a regular
$n$-gon to its original position (but might put the vertices in different
places.) 

\bp
\itemei How many elements does the group $D_n$ have? Prove that you are
correct.
  \solution{$D_n$ has $2n$ elements, because once you have chosen one of the
$n$ places for vertex one, there are two choices for vertex two, and the
remaining vertices can go in only one place each.}
\ep

\subsection{Group tables}
We can always figure out the composition of two permutations of the same
set by using the definition of composition, but if we are going to work
with a given permutation group again and again, it is worth making the
computations once and recording them in a table.  For example the group
of rotations of the square may be represented as in Table
\ref{rotationgrouptable}.  We list the elements of our group, with the
identity first, across the top of the table and down the left side of the
table, using the same order both times.  Then in the row labeled by the
group element $\sigma$ and the column labelled by the group element
$\varphi$, we write the composition $\sigma\circ \varphi$, expressed in
terms of the elements we have listed on the top and on the left side. 
Since a group of permutations is closed under composition, the result
$\sigma\circ \varphi$ will always be expressible as one of these elements.


\begin{table}[htb]\caption{The group
table for the rotations of a square.}\label{rotationgrouptable}\smallskip
\begin{center}
\begin{tabular}{|c|c c c c|}
\hline
$\circ$&$\iota$&$\rho$&$\rho^2$&$\rho^3$\\
\hline
$\iota$&$\iota$&$\rho$&$\rho^2$&$\rho^3$\\
$\rho$&$\rho$&$\rho^2$&$\rho^3$&$\iota$\\
$\rho^2$&$\rho^2$&$\rho^3$&$\iota$&$\rho$\\
$\rho^3$&$\rho^3$&$\iota$&$\rho$&$\rho^2$\\
\hline
\end{tabular}
\end{center}  
\end{table}

\bp
\item In Table \ref{rotationgrouptable}, all the entries in a row (not
including the first entry, the one to the left of the line) are
different.  Will this be true in any group table for a permutation
group?  Why or why not? Also in Table
\ref{rotationgrouptable} all the entries in a column (not including the
first entry, the one above the line) are different.  Will this be true in
any group table for a permutation group?  Why or why not?
\solution{It will always be the case that all the entries of a row or column below and
to the left of the lines in a group table are different.  If two entries of the row
labelled by $\sigma$ were equal, that would mean $\sigma\circ\varphi_1 =
\sigma\circ\varphi_2$ for two different elements $\varphi_1$ and $\varphi_2$.  But if
we multiply both sides of the equation by $\sigma^{-1}$ we get 
$\sigma^{-1}(\sigma\circ\varphi_1) =
\sigma^{-1}(\sigma\circ\varphi_2)$, and by using the associative law and the identity
property, we get $\varphi_1=\varphi_2$, a contradiction.  The same sort of argument
(with $\sigma$ on the right) works for columns.}

\item In Table \ref{rotationgrouptable}, every element of the group
appears in every row (even if you don't include the first element, the
one before the  line).  Will this be true in any group table for a
permutation group?  Why or why not?  Also in Table
\ref{rotationgrouptable} every element of the group appears in every
column (even if you don't include the first entry, the one before the
line).  Will this be true in any group table for a permutation group?  Why
or why not?
\solution{Every element appears in each row and every element appears in
each column (after and below the lines).  This is because the number of
entries in a row or column is the number of elements of the group.  By the
pigeonhole principle, if not all the elements appear in a row, then two are
the same, so $\tau\circ \sigma_1 =\tau\circ\sigma_2$ for some
$\sigma_1$,
$\sigma_2$, and $\tau$ ($\tau$ is the greek letter tau that rhymes with cow)
in the group with
$\sigma_1\not=
\sigma_2$.  Multiplying on the right gives us $\sigma_1 = \sigma_2$, a
contradiction.  The same kind of argument works on columns, though you now
have  
$\sigma_1\circ \tau =\sigma_2\circ \tau$ for some $\sigma_1$,
$\sigma_2$, and $\tau$ in the group with $\sigma_1\not= \sigma_2$.}

\iteme Write down the group table for the dihedral group $D_4$.  Use the
$\varphi$ notation described above to denote the flips.  (Hints:  Part of the
table has already been written down.  Will you need to think hard to write
down the last row?  Will you need to think hard to write down the last
column?\label{dihedral3}  When you multiply a product like $\varphi_{1/3}
\circ \rho$ remember that we defined $\varphi_{1/3}$ to be the flip that
fixes the vertex in position 1 and the vertex in position 3, {\em not} the
one that fixes the vertex on the square labelled 1 and the vertex on the
square labelled 3.)
\solution{\\
\hglue-.25in\begin{tabular}{|c|c c c c c c c c|}
\hline
$\circ$&$\iota$&$\rho$&$\rho^2$&$\rho^3$&$\varphi_{1/3}$ &$\varphi_{2/4}$
&$\varphi_{12/34}$&$\varphi_{14/23}$\\
\hline
$\iota$&$\iota$&$\rho$&$\rho^2$&$\rho^3$&$\varphi_{1/3}$ &$\varphi_{2/4}$
&$\varphi_{12/34}$&$\varphi_{14/23}$\\
$\rho$&$\rho$&$\rho^2$&$\rho^3$&$\iota$&$\varphi_{12/34}$&$\varphi_{23/14}$
&$\varphi_{2/4}$&$\varphi_{1/3}$\\
$\rho^2$&$\rho^2$&$\rho^3$&$\iota$&$\rho$&$\varphi_{2/4}$&$\varphi_{1/3}$
&$\varphi_{14/23}$&$\varphi_{12/34}$\\
$\rho^3$&$\rho^3$&$\iota$&$\rho$&$\rho^2$&$\varphi_{23/14}$&$\varphi_{12/34}$
&$\varphi_{1/3}$&$\varphi_{2/4}$\\
$\varphi_{1/3}$&$\varphi_{1/3}$&$\varphi_{14/23}$&$\varphi_{2/4}$
&$\varphi_{12/34}$ &$\iota$&$\rho^2$&$\rho$&$\rho^3$\\
$\varphi_{2/4}$&$\varphi_{2/4}$& $\varphi_{12/34}$&$\varphi_{1/3}$&
$\varphi_{14/23}$&$\rho^2$&$\iota$&$\rho^3$&$\rho$\\
$\varphi_{12/34}$&$\varphi_{13/24}$&$\varphi_{1/3}$&$\varphi_{14/23}$&$
\varphi_{2/4}$&$\rho$&$\rho^3$&$\iota$&$\rho^2$\\
$\varphi_{14/23}$&$\varphi_{14/23}$&$\varphi_{2/4}$&$\varphi_{12/34}$&
$\varphi_{1/3}$&$\rho^3$&$\rho$&$\rho^2$&$\iota$\\
\hline
\end{tabular}
}
\ep

You may notice that the associative law, the identity property, and the
inverse property are three of the most important rules that we use in
regrouping parentheses in algebraic expressions when solving equations. 
There is one property we have not yet mentioned, the {\em commutative
law}\index{commutative law} which would say that $\sigma\circ \varphi =
\varphi\circ\sigma$.  It is easy to see from the group table of $R_4$ that it
satisfies the commutative law.

\bp \item Does the commutative law hold in all permutation groups?
\solution{No.  In the group $D_3$ a nontrivial rotation does not commute
with a flip.}
\ep

In your group table for the dihedral group $D_4$, you have a copy of the
group of rotations of the square.  When one group $G$ of permutations of a
set $S$ is a subset of another group $G'$ of permutations of $S$, we say
that $G$ is a {\bf subgroup}\index{subgroup} of $G'$.  The reason why we
introduce the new word subgroup is to emphasize that the composition
operation gives the same result whether it is performed in the larger
group or the smaller group.

\bp 
\iteme Find all subgroups of the group $D_4$.
\solution{\hspace*{-12 pt}
$\{\rho^2,\!\iota\}$, $\{\varphi_{1/3},\!\iota\}$, $\{\varphi_{2/4},\!\iota\}$,
$\{\varphi_{12/34},\!\iota\}$, $\{\varphi_{14/23},\!\iota\}$,
$\{\iota,\rho,\rho^2,\rho^3\}$,\break
 $\{\iota,\varphi_{1/3},\rho^2,\varphi_{2/4}\}$, 
$\{\iota,\varphi_{12/34},\rho^2,\varphi_{12/34}\}$, $\{\iota\}$, and all of $D_4$. 
Notice that once a subgroup has $\rho$ or $\rho^3$ in it, it has all powers of $\rho$,
and if it has all powers of $\rho$ and just one flip, it has all the flips in it.  But
the product of two ``different kinds'' of flips is $\rho$ or $\rho^3$, and this shows
that our list of subgroups is complete.
}


\item  Can you find subgroups of the symmetric group $S_4$
 with two elements? Three elements?  Four elements?  Six
elements? (For each positive answer, describe a subgroup.  For each
negative answer, explain why not.)\label{S4}
\solution{Yes to all of the above.  For two elements,
$\{\iota,\pmatrix{1&2&3&4\cr2&1&3&4}\}$, for three elements, 
$\{\iota,\pmatrix{1&2&3&4\cr3&1&2&4}, \pmatrix{1&2&3&4\cr2&3&1&4}\}$ (which is the
rotations of a triangle on $\{1,2,3\}$), for  four elements,  the rotations of a
square as above, and for six elements, the rotations and flips of a triangle on
$\{1,2,3\}$.  (There is no subgroup with five elements, but at this point, that is
too hard to verify.)}



\ep



\subsection{The cycle structure of a permutation}
There is an even more efficient way to write down permutations.  Notice
that the product in Figure \ref{permutationproduct} is
$\pmatrix{1&2&3&4\cr2&3&1&4}$. 
  We have drawn the directed graph of this
permutation in Figure \ref{permutationcycledigraph}. 
\begin{figure}[htb]\caption{The
directed graph of the permutation
$\protect\pmatrix{1&2&3&4\cr2&3&1&4}$.}
\label{permutationcycledigraph}\smallskip
\begin{center}\mbox{\psfig{figure=permutationcycledigraph.eps
}}
\end{center}  
\end{figure}
  You see that the
graph has two directed cycles, the rather trivial one with vertex 4
pointing to itself, and the nontrivial one with vertex 1 pointing to
vertex 2 pointing to vertex 3 which points back to vertex 1.  A
permutation is called a {\bf cycle}\index{cycle (of a
permutation)}\index{permutation!cycle of} if its digraph consists of
exactly one
 cycle.  Thus $\pmatrix{1&2&3\cr2&3&1}$ is a cycle but
$\pmatrix{1&2&3&4\cr2&3&1&4}$ is not a cycle by our definition.  We write
$(1\ 2\ 3)$ or $(2\ 3\ 1)$ or $(3\ 1\ 2)$ to stand for the cycle
$\sigma =\pmatrix{1&2&3\cr2&3&1}$.  We can describe cycles in another way
as well.  A {\bf cycle}\index{cycle of a
permutation}\index{permutation!cycle of} of the permutation
$\sigma$ is a list
$(i\ 
\sigma(i)\ 
\sigma^2(i)
\ \ldots\ 
\sigma^n(i))$ that does not have repeated elements
while the list $(i\ \sigma(i)\ 
\sigma^2(i)\ \ldots\ \sigma^n(i))\ \sigma^{n+1}(i))$  does have repeated
elements. We say that the cycles $(i\ 
\sigma(i)\ 
\sigma^2(i)\ \ldots\ \sigma^n(i))$ and $(j\ 
\sigma(j)\ 
\sigma^2(j)\ \ldots\ \sigma^n(j))$ are {\em equivalent} if there is an
integer $k$ such that 
$j=
\sigma^k(i)$.

\bp
\iteme Find the cycles of the permutations $\rho$, $\varphi_{1/3}$ and
$\varphi_{12/34}$ in the group $D_4$.
\solution{$\rho$ has one cycle, $(1\ 2\ 3\ 4)$, $\varphi_{1/3}$ has one cycle
$(2\ 4)$, $\varphi_{12/34}$ has two cycles, namely $(1\ 2)$ and
$(3\ 4)$.}
\item Find the cycles of the permutation $\pmatrix{1&2&3&4&5&6&7&8&9\cr
3&4&6&2&9&7&1&5&8}$.
\solution{
$\pmatrix{1&3&6&7}$, $\pmatrix{2&4}$, and $\pmatrix{5&9&8}$.}

\item Show that if $(i\ 
\sigma(i)\ 
\sigma^2(i)
\ \ldots\ 
\sigma^n(i))$  does not have repeated elements\label{firstrepeat} 
while the list $(i\ \sigma(i)\ 
\sigma^2(i)\ \ldots\ \sigma^n(i))\ \sigma^{n+1}(i))$  does have repeated
elements, then $\sigma^{n+1}(i)=i$, and so all other $\sigma^s(i)$ are in the cycle.
\solution{If $\sigma^{n+1}(i) = \sigma^p(i)$ and $0\le p\le n$, then $\sigma^{n+1-p}(i)
=i$, and so the first repeat was not $\sigma^{n+1}(i)$ after all.  Therefore
$\sigma^{n+1}(i)=i$, and acting on $i$ with succeeding powers of $\sigma$ just carries
us through the list again and again.}

\iffalse \item Show that if $(i\ 
\sigma(i)\ 
\sigma^2(i)\ \ldots\ \sigma^n(i))$ and $(j\ 
\sigma(j)\ 
\sigma^2(j)\ \ldots\ \sigma^n(j))$ are equivalent (with $j=
\sigma^k(i)$), then
$$(\sigma^k(i)\ \sigma^{k+1}(i)\ \ldots\  \sigma^n(i)\  i\ 
\sigma(i)\ \ldots\ 
\sigma^{k-1}(i))
=(j\ 
\sigma(j)\ 
\sigma^2(j)\ \ldots\ \sigma^n(j)).$$
\solution{The hardest thing here is reading the problem!  The point is that if
$j=\sigma^k(i)$, then $\sigma(j) = \sigma(\sigma^{k}(i))=\sigma^{k+1}(i)$, and so on,
so the cycles starting with $\sigma^i(k)$ and $j$ are identical.}

\item If two cycles of $\sigma$ have an element in common, what can we
say about them? 
\solution{They are the same cycle, by the previous problem since as permutations,}
\fi

\iteme The support set of the cycle $(a_1\ a_2\ \ldots\ a_n)$ is the set
$\{a_1,a_2,\ldots,a_n\}$.  Show that the support sets of the cycles of a permutation of
the set $S$ form a partition of $S$.\label{cyclepartition}
\solution{Every element is in a cycle by the definition of cycles, so suppose two
support sets have an element in common.  If the cycles are 
$(i\ 
\sigma(i)\ 
\sigma^2(i)\ \ldots\ \sigma^n(i))$  and $(j\ 
\sigma(j)\ 
\sigma^2(j)\ \ldots\ \sigma^m(j))$ and $\sigma^p(i) =\sigma^q(j)$ is a common
element, then,
$j=\sigma^{p-q}(i)$.  Thus if $p\ge q$, $j$ is in the first cycle.  Otherwise $i$ is in
the second cycle.  If $j$ is in the first cycle, it is $\sigma^k(i)$ for some $i$, and
as we take successive powers, we get, by Problem \ref{firstrepeat} that $i$ is
$\sigma^h(j)$, and then successive powers of $\sigma$ just give us the rest of the
cycle beginning with $i$.  Therefore the two support sets are identical. A similar
argument works if $i$ is in the second cycle. Thus the support sets partition  $S$.}


\ep

We regard a cycle $(i\ 
\sigma(i)\ 
\sigma^2(i)\ \ldots\ \sigma^n(i))$ as standing for the permutation 
$$\pmatrix{i&
\sigma(i)&
\cdots&\sigma^{n-1}(i)&\sigma^n(i)\cr
\sigma(i)&
\sigma^2(i)&\cdots&\sigma^n(i)&i}.$$  Since interchanging the columns in
the two row notation for a permutation does not change the permutation\footnote{Note
that not all ways of interchanging the columns list the bottom row in the order of a
cycle of $\sigma$, though.}, this means that equivalent cycles represent the same
permutation.  Thus we consider equivalent cycles to be equal in the same way we
consider
${1\over 2}$ and $2\over 4$ to be equal.  In particular, this means that $(i_1\ i_2\
\ldots i_n) = (i_j\ \ i_{j+1}\ \ldots\ i_n\ \ i_1\ \ i_2\ \ldots\ i_{j-1})$.

 If
$\gamma_1$,
$\gamma_2$,
\ldots, $\gamma_n$  are the (inequivalent) cycles of a
permutation
$\sigma$ of the set $S$, then we write
$$\sigma = \gamma_1\gamma_2\cdots\gamma_n= \gamma_{i_1}\gamma_{i_2}
\cdots \gamma_{i_n},$$
that is, we say that $\sigma$ is the product of its cycles in any order. 
There are two ways to think of this notation.  You can think of it as a
brand new kind of multiplication where we multiply a permutation of one
set by a permutation of  a disjoint set to get a new permutation of the
union of the two sets. In this notation $\sigma(x)$ is determined by which
cycle $\gamma_k$ contains $x$ in its support set and $\sigma(x) =
\gamma_k(x)$. You can also think of identifying
$\gamma$ with the product of $\gamma$ with a collection of one-cycles, one
one-cycle for each element of $S$ not in the support set of $\gamma$. For example we
identify the  cycle $(2\  4\ 5)$ in $S_6$ with $(1)(2\  4\ 5)(3)(6)$. Similarly we
identify $(3)$ with $(1)(2)(3)(4)(5)(6)$.  But when we use $(3)$ as a notation for
$(1)(2)(3)(4)(5)(6)$, we still say its support set is $\{3\}$.  Then the product above
is simply composition of functions.  We can summarize this discussion with the
following theorem\index{disjoint cycles}\index{cycles!disjoint}\footnote{In the
statement of the theorem we use the word disjoint; two cycles are said to be {\em
disjoint} if their support sets are disjoint.}.  The proof of the theorem is
essentially your solution to Problem
\ref{cyclepartition}.

\begin{theorem} Every permutation can be written, up to order, in one and
only one way as a product of disjoint cycles.\end{theorem}

We usually don't bother writing down one-cycles when we write a permutation as a
product of cycles; it is a convention to make less work for us.  We just have to
remember that they are implicitly there. You should recognize that the first
interpretation we gave for the multiplication of disjoint cycles applies only when the
cycles are disjoint.  From now on  we will use the phrase ``product of permutations"
to mean the same thing as ``composition of permutations," and will use either the
notation
$\sigma\circ\varphi$ or
$\sigma\varphi$ for the composition of
$\sigma$ and $\varphi$, whichever is convenient.

\bp
\iteme Write the permutations $\rho$, $\varphi_{1/3}$ and $\varphi_{12/34}$
in $D_4$ as products of disjoint cycles.
\solution{$\rho = (1\ 2\ 3\ 4)$, $\varphi_{1/3} = (2\ 4) = (1)(2\ 4)(3)$ (either side
of the equation is a right answer), $\varphi_{12/34} = (1\ 2)(3\        4)$.}

\item Write the permutation $\pmatrix{1&2&3&4&5&6&7&8&9\cr
3&4&6&2&9&7&1&5&8}$ as a product of disjoint cycles.
\solution{$(1\ 3\ 6\ 7)(1\ 4)(5\ 9\ 8)$.}


\item Explain why, if $\gamma$ and $\gamma'$ are
two disjoint cycles, then $\gamma\gamma' =
\gamma'\gamma$.\label{disjoint-commutative}
\solution{When we compute $\gamma\gamma'(i)$, the element $i$ is fixed (i.e. mapped to
itself) by either $\gamma$ or $\gamma'$.  But if $\gamma'$ fixes $i$ and
$\gamma(i)=j$, with $j\not=i$ then $j$ is in the support set of $\gamma$, so it is not
in the support set of $\gamma'$.  Thus $(\gamma\circ\gamma')(i) =j$.  However since
$\gamma'$ fixes $j$ as well, then $(\gamma'\circ\gamma)(i) =\gamma'(j) =j$ also.  If
$\gamma(i)=i$ also, then $\gamma\gamma'(i) =\gamma'\gamma(i) = i$.  A similar argument
works if $\gamma$ fixes $i$, so in all cases, $\gamma\gamma'(i) = \gamma'\gamma(i)$,
which is what it means to say that $\gamma\gamma'=\gamma'\gamma$.}



\itemi Write a recurrence for the number $c(k,n)$ for the number of
permutations of $[k]$ that have exactly $n$ cycles, including 1-cycles. 
Use it to write a table of $c(k,n)$ for $k$ between 1 and 7 inclusive. 
Can you find a relationship between $c(k,n)$ and any of the other
families of special numbers such as binomial coefficients, Stirling
numbers, Lah numbers, etc. we have studied?  If you find such a
relationship, prove you are right.
\solution{The element $k$ is either in a cycle by itself or it isn't.  The number of
permutations in which it is in a cycle by itself is $c(k-1,n-1)$.  If it is in a cycle
with something else, it can come after any of the $k-1$ elements, and each choice of
which one it comes after gives a different permutation.  (You can always shift a cycle
around so that $k$ doesn't come first.)  Thus $c(k,n) = c(k-1,n-1) + (k-1)c(k-1,n)$. 
The table is:\par
\begin{tabular}{|c|c|c|c|c|c|c|c|c|}
$k/n$&0&1&2&3&4&5&6&7\\
\hline
0&1&0&0&0&0&0&0&0\\
1&0&1&0&0&0&0&0&0\\
2&0&1&1&0&0&0&0&0\\
3&0&2&3&1&0&0&0&0\\
4&0&6&11&6&1&0&0&0\\
5&0&24&50&35&10&1&0&0\\
6&0&120&274&225&85&15&1&0\\
7&0&720&1764&1624&735&175&21&1
\end{tabular}\par
We see that the table is identical with our earlier table of the Stirling numbers of
the first kind, except that in that table there was an alternating pattern of minus
signs, offset by one place in successive rows.  Thus it must be the case that $c(k,n)=
|s(k,n)|$.  To prove this, note that it is the case when $k=0$, and also when $n=1$. 
Now assume inductively it is true when $k= m-1$ and notice that if $n>0$,
$s(m,n)=s(m-1,n-1)-(m-1)s(m-1,n)$, that
$s(m-1,n-1)$ and $s(m-1,n)$ have opposite signs, so
\begin{eqnarray*}\hspace*{-.5in}|s(m-1)-(m-1)s(m-1,n)|&=& |s(m-1| + 
(m-1)|s(m-1,n)|\\
&=&c(m-1,n-1)+(m-1)c(m-1,n)\\ &=& c(m,n).\end{eqnarray*} Thus by the
principle of mathematical induction $c(k,n) = |s(k,n)|$ for all nonnegative numbers
$k$ (and all
$n$ between 0 and $k$.)
}

\itemesi (Relevant to Appendix \ref{expogenfun}.) A permutation $\sigma$ is called an
{\bf involution}\index{involution} if $\sigma^2=\iota$.  When you write an
involution as a product of disjoint cycles, what is special about the
cycles?
\solution{They are all 2-cycles or 1-cycles.}



\item Write the product $(1\ 2)(1\ 3)(1\ 4)$ in two row notation and
then write it in terms of of {\em disjoint} cycles.
\solution{$\pmatrix{1&2&3&4\cr4&1&2&3}$, $(1\ 4\ 3\ 2)$.}

\iteme How many permutations in $S_n$ may be written as a product of not
necessarily disjoint two cycles?
\solution{All of them, because each cycle can be written as a product on
non-disjoint cycles.}

\subsection{Signs of permutations and the alternating group}

\item (May be time-consuming.) Consider the permutations of $\{1,2,3,4\}$ which
are
$\iota$ or have one of the forms $(i\ j\ k)$ or $(i\ j)(h\ k)$, where $i$,
$j$, $k$, and $h$ are distinct.  Do these permutations form a subgroup of
the group $S_4$?  How many of these permutations are there?
\solution{Note that two cycles in $S_4$ of the form $(i\ j\ k)$ have two or three
elements in common, so they have the same kind of overlap as $(1\ 2\ 3)$ and $(2\ 3\
4)$,
$(1\ 2\ 3)$ and $(3\ 2\ 4)$,  $(1\ 2\ 3)$ and $(1\ 3\ 2)$, or $(1\ 2\ 3)$ and $(1\
2\ 3)$.  The products of these four pairs are, respectively, $(2\ 1)(3\ 4)$, $(2\ 4\
1)$,
$(1)(2)(3)= (1)(2)(3)(4)=\iota$, and $(1\ 3\ 2)$.  If we take a three-cycle times a
product of two two-cycles it will have the same form as $(1\ 2\ 3)(1\ 2)(3\ 4)$ or
$(1\ 2\ 3)(1\ 3)(2\ 4)$.  The first of these products is $(1\ 3\ 4)$ and the second
is $(2\ 4\ 3)$.  If we have a product of two elements each of which is a product of
two two-cycles, it will have the same form as either $(1\ 2)(3\ 4)(1\ 2)(3\ 4)$ or
$(1\ 2)(3\ 4)(1\ 3)(2\ 4)$.  The first of these is $\iota$ and the second is $(1\
4)(2\ 3)$.  Thus our set is closed under composition, and so it is a subgroup of
$S_4$.  The subgroup has size 12.}

\itemes The {\em sign} of a 1-cycle, 3-cycle, \ldots, $(2k-1)$-cycle is
defined to be 1, while the {\em sign} of a 2-cycle, 4-cycle, \ldots,
$2k$-cycle is defined to be -1.  The {\bf sign} of a product of disjoint
cycles is the product of the signs of its cycles.\index{permutation!sign
of}\index{sign of a permutation}
\begin{enumerate}
\item What happens to the sign of an $n$-cycle if you multiply it by a
2-cycle (which might or might not be disjoint from it)?
\solution{If the $m$-cycle and 2-cycle are disjoint, then by definition, the sign
changes to the opposite value.  If they are not disjoint, and the two cycle is $(i\
j)$, either
$i$ or $j$ or both are in the original cycle.  Suppose that only $i$ is in the cycle
(and the same argument works if only $j$ is in the cycle), and the original cycle is
$(i\ i_2\ \ldots\ i_n)$. Then the product
$(i\ i_2\
\ldots\ i_n)(i\ j)$ is
$(i\ j\ i_2\ \ldots\ i_n)$, and the product $(i\ j)(i\ i_2\ \ldots\ i_n)$ is $(j\ i\
i_2\
\ldots\ i_n)$.  Thus the sign changes to the opposite sign.  Suppose that $i$ and $j$
are both in the original cycle.  Then $j= i_k$ for some $k$.  Then $(i\ i_2\ \ldots\
i_n)(i\ i_k)=(i i_{k+1}\ i_{k+2}\ \ldots\ i_n)(i_k\ i_2\ \ldots\ i_{k-1})$.  If the
original sign was 1, then we started with an cycle on an odd number of entries  and
so now have an cycle on an even number of entries times an cycle on an odd number of entries, so the sign has changed.  It the
original sign was -1, then we started with an cycle on an even number of entries, so we now have a product
of two cycles on an odd number of entries, which gives sign 1, or a product of two
cycles on an even number of entries, which gives us sign 1. Thus in every case, the
sign changes to the opposite sign.} 
\item What happens to the sign of a product of an $n$-cycle and a
disjoint $m$-cycle if you multiply it by a two cycle which is disjoint
from neither of the other two?
\solution{The product becomes an $m+n$ cycle, and so if both cycles were on an even
number of entries or both were on an odd number, we now have an cycle on an even
number of entries, so its sign is -1, but previously the product had sign 1.  Thus
the sign changes.  If one of the original cycles was on an even number of entries
and one was on an odd number, we now have a cycle on an odd number of entries, so it
has sign 1, while the original product has sign -1.  Thus the sign changes.}
\item What happens to the sign of a permutation if you multiply it by a
2-cycle?\solution{The cases we just analyzed cover all the possibilities, so the sign
changes.}
\item A 2-cycle is often called a {\em
transposition}\index{transposition}.  What is the sign of a product of
$k$ (not necessarily disjoint) transpositions? (Don't forget to prove you are
correct.)
\solution{$(-1)^k$.  The proof is probably easiest by induction, for the sign of one
transposition is -1, and if we take a product of $i$ transpositions and multiply by
one more transposition, the sign changes.  Thus if the product of $i$ transpositions
has sign $(-1)^i$, the product of $i+1$ transpositions has sign $(-1)^{i+1}$, and so
by the principle of mathematical induction, the sign of a product of $k$
transpositions if
$(-1)^k$ for all positive integers $k$.  In fact, had we taken $\iota$ for our base
case, we could have concluded the same result for all nonnegative integers.}
\item Can you compute the sign of a product of two permutations from the
signs of the permutations?  Why or why not?
\solution{You can compute it; it is the product of the signs.  Just write them as
products of transpositions, multiply all those transpositions together and use the fact
that, for transpositions, the sign of the product is the product of the signs.}
\end{enumerate}
\itemes Do the elements of $S_n$ with  sign -1 form a subgroup of $S_n$? 
Do the elements of $S_n$ with sign 1 form a subgroup of $S_n$?
\solution{The elements of sign $-1$ are not closed under composition so they do not
form a subgroup.  The elements of sign $1$ are closed under composition so they do
form a subgroup.}
\itemes Describe a subgroup of $S_n$ of size $n!/2$.  This subgroup is
called the {\bf alternating group}\index{alternating group} $A_n$. 
Describe explicitly a partition of $S_n$ that you can use with the
quotient principle to prove bijectively that the subgroup you described has size
$n!/2$.\label{alternatinggroup}
\solution{Partition $S_n$ into the permutations of sign 1 and sign -1.  Choose a
transposition.  Multiplying the elements in one block gives you elements in the other
block.  Multiplying by the transposition twice gives you the same permutation back, so
the function that multiplies by this transposition has an inverse and so is a
bijection.}
\ep

\section{Groups Acting on Sets}
We defined the rotation group $R_4$ and the dihedral group $D_4$ as
groups of permutations of the vertices of a square.  These permutations
represent rigid motions of the square in the plane and in three
dimensional space respectively.   The square has geometric features of interest other
than its vertices; for example its diagonals, or
its edges. Any geometric motion of the square that returns it to its
original position takes each diagonal to a possibly different diagonal,
and takes each edge to a possibly different edge.  

We could similarly
define the  group $R$ of rigid motions of a cube as the group of
permutations of its eight vertices that correspond to three dimensional
motions that return the cube to its original position, but perhaps with
vertices in different places than they were originally.  Now each of
these motions also takes the 12 edges of the cube to (possibly different)
edges and takes the six faces of the cube to (possible different) faces. 
So we can think of our group $R$ as acting not only on the vertices of our
geometric figure, but on edges, diagonals, faces, or perhaps other
geometric features.  

We have seen above that the fact that we have defined a permutation group
as the permutations of some specific set doesn't preclude us from
thinking of the elements of that group as permuting the elements of some
other set as well.  In order to keep track of which permutations of which
set we are using to define our group and which other set is being
permuted as well, we introduce some new language.  We are going to say
that the group $D_4$ ``acts" on the edges and diagonals of  a square and
the group $R$   of permutations of the vertices of a cube
that arise from rigid motions of the cube ``acts" on the edges, faces,
 diagonals, etc. of the cube.  Instead of talking
about permutations of these edges, faces, etc., we will use an equivalent
word and talk about bijections of these objects.  In this way it will be
easier for us to keep track of which permutations form our original group
and how this group relates to other objects of interest. 

\begin{figure}[htb]\caption{A cube with the positions of its vertices and
faces labelled.  The curved arrows point to the positions that are blocked by
the cube.}
\label{cube1}\smallskip
\begin{center}\mbox{\psfig{figure=cube1.eps
}}
\end{center}  
\end{figure}

\bp 
\iteme In Figure \ref{cube1} we show a cube with the positions of its
vertices and faces labelled.  As with motions of the square, we let we let 
$\varphi(x)$ be the label of the place where vertex  previously in
position
$x$ is now.
\begin{enumerate}
\item Write in two row-notation the permutation $\rho$ of the vertices
that corresponds to rotating the cube 90 degrees around a vertical axis
through the faces $t$ (for top) and $u$ (for underneath).  (Rotate in a
right-handed fashion around this axis, meaning that vertex 6 goes to the
back and vertex 8 comes to the front.)  Write in two-row notation the
bijection
$\beta_{\rho}$ of the faces that corresponds to this member $\rho$ of
$R$.\label{actsonparta}
\solution{$\!\!\!\pmatrix{1&2&3&4&5&6&7&8\cr
2&3&4&1&6&7&8&5}$~~$\pmatrix{t&f&r&b&l&u\cr t&r&b&l&f&u}$}
\item Write in two-row notation the permutation $\varphi$ that rotates
the cube 120 degrees around the diagonal from vertex 1 to vertex 7 and
carries vertex 8 to vertex 6.  Write in two-row notation the bijection
$\beta_\varphi$ of the faces that corresponds to this member of $R$.
\solution{$\!\!\!\pmatrix{1&2&3&4&5&6&7&8\cr1&4&8&5&2&3&7&6}~
~\pmatrix{t&f&r&b&l&u\cr r&u&b&t&f&l}$}
\item Compute the two-row notation for $\rho\circ\varphi$, for
$\beta_{\rho}\circ\beta_{\varphi}$ ($\beta_{\rho}$ was defined in Part
\ref{actsonparta}), and write in two-row notation the bijection 
$\beta_{\rho\circ\varphi}$ of the faces that corresponds to the action of
the permutation $\rho\circ\varphi$ on the faces of the cube (for this
question it helps to think geometrically about what motion of the cube is
carried out by $\rho\circ\varphi$).  What do you observe about
$\beta_{\rho\circ\varphi}$ and
$\beta_{\rho}\circ\beta_{\varphi}$?\label{cube2}
\solution{\begin{eqnarray*}
\pmatrix{1&2&3&4&5&6&7&8\cr
2&3&4&1&6&7&8&5}\pmatrix{1&2&3&4&5&6&7&8\cr1&4&8&5&2&3&7&6}&=&\\
\pmatrix{1&2&3&4&5&6&7&8\cr 2&1&5&6&3&4&8&7}&&
\end{eqnarray*}
\begin{eqnarray*}
\pmatrix{t&f&r&b&l&u\cr t&r&b&l&f&u}\pmatrix{t&f&r&b&l&u\cr r&u&b&t&f&l}&=&\\
\pmatrix{t&f&r&b&l&u\cr b&u&l&t&r&f}&&
\end{eqnarray*}
$$\beta_{\rho\circ\varphi} =\pmatrix{t&f&r&b&l&u\cr
b&u&l&t&r&f}=\beta_{\rho}\circ\beta_{\varphi}$$ }
\end{enumerate}

\itemei How many permutations are in the group $R$?  $R$ is sometimes
called the ``rotation group" of the cube.  Can you justify this?
\solution{There are eight places where vertex one can go.  Once vertex 1 is
placed, there are three ways the three edges ``sticking out" of it can be
placed, and then the location of all the vertices is determined.  Thus there
are 24 elements of the group.  Of them, we have the identity, three
rotations about each of the three axes through the midpoints of the faces,
one additional rotation about each of the six axes joining opposite faces,
and $2\cdot4=8$ additional rotations about each of the four axes joining a
vertex to the diagonally (in three dimensions) opposite vertex.  This gives
us a total of 10+6+8=24 rotations, so every element of the group is a
rotation.}
\ep

We say that a permutation group $G$ {\bf acts}\index{group acting on a
set}\index{action of a group on a set} on a set
$S$ if, for each member
$\sigma$ of $G$ there is a bijection $\beta_{\sigma}$ of $S$ such
that
$$\beta_{\sigma\circ\varphi} = \beta_{\sigma}\circ\beta_{\varphi}$$ for
every member $\sigma$ and $\varphi$ of $G$. In Problem \ref{cube2} you
saw one example of this condition.  If we think intuitively of $\rho$ and
$\varphi$ as motions in space, then following the action of $\varphi$ by
the action of $\rho$ does give us the action of $\rho\circ\varphi$.  We
can also reason directly with the permutations in $R$ to show that $R$
acts on the faces of the cube.  

\bp 
\iteme Suppose that $\sigma$ and $\varphi$ are permutations in the group
$R$ of rigid motions of the cube. We have argued already that each rigid
motion sends a face to a face.  Thus $\sigma$ and $\varphi$ both send the
vertices on one face to the vertices on another face. Let
$\{h,i,j,k\}$ be the set of vertices on a face $F$.
\begin{enumerate}
\item  What
are the vertices of the face
$F'$ that
$F$ is sent to by $\varphi$?  
\solution{$\varphi(h)$, $\varphi(i)$, $\varphi(j)$, $\varphi(k)$.}
\item What are the vertices of the face $F''$
that
$F'$ is sent to by $\sigma$? 
\solution{$\sigma(\varphi(h))$, $\sigma(\varphi(i))$, $\sigma(\varphi(j))$,
$\sigma(\varphi(k))$.}
\item What are the vertices of the face $F'''$
that $F$ is sent to by $\sigma\circ\varphi$? 
\solution{$(\sigma\circ\varphi)(h)=\sigma(\varphi(h))$, $\sigma(\varphi(i))$,
$\sigma(\varphi(j))$,
$\sigma(\varphi(k))$.}
\item How have you just shown
that the group
$R$ acts on the faces?
\solution{ We have just shown that for each face $F$,
$\beta_{\sigma\circ\varphi}(F) = \beta_{\sigma} \circ \beta_{\varphi}(F)$,
so that $\beta_{\sigma\circ\varphi} = \beta_{\sigma} \circ
\beta_{\varphi}$.}
\end{enumerate}
\iteme The group $D_4$ is a group of permutations of
$\{1,2,3,4\}$ as in Problem \ref{dihedral1}.  We are going to show in this 
problem and in Problem
\ref{orbits1} how this group acts in a natural way on the two-element subsets of
$\{1,2,3,4\}$.  In particular, for each two-element subset $\{i,j\}$ of
$\{1,2,3,4\}$ and each member
$\sigma$ of
$D_4$ we define $\beta_{\sigma}(\{i,j\}) = \{\sigma(i),\sigma(j)\}$. 
Show that with this definition of $\beta$, the group $D_4$ acts on the
two-element subsets of $\{1,2,3,4\}$.\label{D_4on2-sets}
\solution{
The action has been defined for us, so all we need to show is that
it is indeed an action.  We must show that for each permutation $\sigma$ in
$D_4$,
$\beta_{\sigma}$ is a bijection and in addition we must show that
$\beta_{\sigma}\circ\beta_{\tau}=
\beta_{\sigma\circ\tau}$.  If
$\beta_{\sigma}(\{i,j\})=\beta_{\sigma}(\{h,k\})$, then either
$\sigma(i)=\sigma(h)$ and $\sigma(j)=\sigma(k)$ or else
$\sigma(i)=\sigma(k)$ and $\sigma(j) =\sigma(h)$.  Since $\sigma$ is a
permutation, in the first case we get $i=h$ and $j=k$, so that $\{i,j\}=
\{h,k\}$, and in the second case we get that $i=k$ and $j=h$ so that
$\{i,j\} = \{k,h\} =\{h,k\}$.  Thus in either case $\{i,j\} = \{h,k\}$, so
that $\beta_{\sigma}$ a one-to-one function from the finite set of two
element subsets of $\{1,2,3,4\}$ to itself, and so it is a bijection.

To show that $\beta_{\sigma}\circ\beta_{\tau} = \beta_{\sigma\circ\tau}$, we
note that 
\begin{eqnarray*}(\beta_{\sigma}\circ\beta_{\tau})(\{i,j\}) &=&
\beta_{\sigma}(\beta_{\tau}(\{i,j\})=\beta_{\sigma}(\{\tau(i).
\tau(j)\})=\\
\sigma({\tau(i),\tau(j)})&=&\{(\sigma\circ\tau)(i),(\sigma\circ\tau)(j)\} =\\
&&\beta_{\sigma\circ\tau}(\{i,j\})\end{eqnarray*}

}
\iteme In Problem \ref{D_4on2-sets}, what is the
set of two element subsets that you get by computing
$\beta_{\sigma}(\{1,2\})$ for all $\sigma$ in $D_4$?  What is the set of
two-element subsets you get by computing $\beta_{\sigma}(\{1,3\})$ for all
$\sigma$ in $D_4$?  Describe these two sets geometrically in terms of the
square.\label{orbits1}
\solution{From $\{1,2\}$ we get $\{1,2\}$, $\{2,3\}$, $\{3,4\}$, and
$\{3,4\}$.  From $\{1,3\}$ we get $\{1,3\}$ and $\{2,4\}$.  Thus we get the
set of edges and the set of diagonals of the square.
}
\iteme Using the notation of Problem \ref{dihedral3}, what is the effect
 of a 180 degree rotation on the diagonals of a square?  What is the
effect of the flip $\varphi_{1/3}$ on the diagonals of a square?  How many
elements of $D_4$ send each diagonal to itself?  How many elements of
$D_4$ interchange the diagonals of a square?
\solution{The 180 degree rotation sends the diagonals to themselves, i.e.,
it fixes the diagonals. 
$\varphi_{1/3}$  fixes the
diagonals.  So does $\varphi_{2/4}$ and the identity.  However, $\rho$,
$\rho^3$, $\varphi_{12/34}$ and $\varphi_{14/23}$ all exchange the
diagonals.  Thus four elements fix each diagonal and four elements
interchange them.
}
 \begin{figure}[htb]\caption{The colored square corresponding to the
function $f$ with $f(1) =R$, $f(2)
= R$,
$f(3)=B$, $f(4)=B$.}\label{ColoredSquare}\smallskip
\begin{center}\mbox{\psfig{figure=ColoredSquare.eps}}
\vspace*{-.3in}
\end{center}  
\end{figure}
\iteme Recall that when you were asked in Problem \ref{twocolorsofbeads} to
find the number of ways to place two red beads and two blue beads at the
corners of a square free to move in three dimensional space, you were
not able to apply the quotient principle to answer the question.  This is
a simple prototype of the problems we will solve by using permutation
groups.  Though we don't yet have the tools to solve it, we can get an
interesting example of a group acting on a set from it.  An assignment of
red and blue beads to the corners of a square can be thought of as a
function $f$ from the vertices of the square to the set $\{R,B\}$.  
For example, in Figure \ref{ColoredSquare} we would have $f(1) =R$, $f(2)
= R$,
$f(3)=B$, $f(4)=B$.  What is the original position of the vertex that
$\sigma$ moves to position
$j$?  In terms of $f$, $\sigma$ and $j$, give an expression that
represents the color of the vertex in position $j$ after we apply the
permutation
$\sigma$ to the vertices.\label{actiononfunctions}
\solution{The original position of the vertex that $\sigma$ moves to
position $j$ is $\sigma^{-1}(j)$.  Thus the color of the vertex in position
$j$ after we apply $\sigma$ is $f(\sigma^{-1}(j))$, which is also $(f\circ
\sigma^{-1})(j)$.
}

  
\iteme 
Problem \ref{actiononfunctions} suggests that when we have a group of
permutations of a set
$S$ and we want to know how it acts on the functions from $S$ to a set $C$
(of ``colors''), we should define
$\beta_{\sigma}(f)$ to be $f\circ\sigma^{-1}$.   For the coloring function 
$f$ in Figure
\ref{ColoredSquare}, let $g=f\circ\rho^{-1}$ and compute $g(1)$, $g(2)$,
$g(3)$, and $g(4)$ so you can see how in this case $f\circ\rho^{-1}$ is
the coloring that results from rotating the colored square by the
rotation $\rho$. Show if a group
$G$ of permutations acts on a set
$S$, then
$\beta_{\sigma}(f) =f\circ\sigma^{-1}$ satisfies the definition of of $G$
acting on the set of functions $f$ from $S$ to $C$, while 
$\beta_{\sigma}(f) =f\circ\sigma$ does not.\label{actingonfunctions2}
\solution{
$g(1)=B$, $g(2)=R$, $g(3)=R$, and $g(4)=B$.

If $\beta_{\sigma}(f) = f\circ\sigma^{-1}$, then
\begin{eqnarray*}
(\beta_{\sigma}\circ\beta_{\tau})(f)&=&\beta_{\sigma}(\beta_{\tau}(f))\\
&=&
\beta_{\sigma}(f\circ \tau^{-1})\\ &=&(f\circ\tau^{-1})\circ\sigma^{-1}\\ &=&
f\circ(\tau^{-1}\circ \sigma^{-1})\\ &=& f\circ (\sigma\circ\tau)^{-1}\\
&=&\beta_{\sigma\circ\tau}(f).
\end{eqnarray*} Therefore $\beta_{\sigma}(f) = f\circ\sigma^{-1}$ defines an
action of a group of permutations of a set $X$ on functions defined from the
set $X$ to a set $Y$.  

On the other hand, defining $\beta_{\sigma}(f) = f\circ\sigma$ does not
define a group action.  For example, if we consider the function $f(i)=i$
on the four numbered corners of the square in Figure  \ref{ColoredSquare},
and take the permutations $\rho$ and $\varphi_{1/3}$, we know from the fact
that $f$ is the identity function and $\rho$ and $\varphi_{1/3}$ do not
commute that
$\beta_{\varphi_{1/3}\circ\rho}(f) = f\circ(\varphi_{1/3}\circ\rho)\not =
f\circ(\rho\circ\varphi_{1/3})=(f\circ\rho)\circ\varphi_{1/3}=\beta_{\rho}(f)
\circ\varphi_{1/3}= \beta_{\varphi_{13}}(\beta_\rho(f)) =(\beta_{\varphi_{1/3}}\circ
\beta_{\rho})(f).$  Thus defining $\beta_\sigma(f)= f\circ \sigma$ does not give
an action of $D_4$ on functions from the vertices of the square to $\{1,2,3,4\}$.}


\iteme  If we are interested in something like the number of ways of
painting the faces of a cube, and we know how the rotation group of the
cube acts on the faces of the cube, then we are interested in how the
rotation group acts on functions from the set of faces of the cube to the
set of colors.  The difference between this problem and Problem
\ref{actingonfunctions2} is that in Problem \ref{actingonfunctions2} it
was the members $\sigma$ of our permutation group that were permuting the
elements of $S$, while now we have bijections $\beta_{\sigma}$ that are
permuting the elements of $S$, in this case the faces of the cube.  We
want to define a new action $\beta'_{\sigma}$ of our permutation group on
functions from $S$ to a set $C$.  How should we define
$\beta'_{\sigma}(f)$?  (Hint: think about the example of the faces of the
square.  After we act on the faces with $\beta_{\sigma}$, what is the
color of the vertex of the cube that is now in
position~$x$?)\label{actingonfunctions3}
\solution{We should define $\beta'_{\sigma}(f) = f\circ
\beta_{\sigma}^{-1}$.  The color of the vertex now in position $x$ is
$f(\beta_{\sigma}^{-1}(x))$ (which is the same as
$(f\circ\beta_{\sigma}^{-1})(x)$.
}

\ep

\subsection{Orbits} In Problem \ref{orbits1} you saw that the action of
the dihedral group $D_4$ on two element subsets of $\{1,2,3,4\}$ seems to
split them into two sets, one with two elements and one with 4.  We call
these two sets the ``orbits" of $D_4$ acting on the two elements subsets
of $\{1,2,3,4\}$.  More generally, the {\bf orbit}\index{orbit} of a
permutation group
$G$  determined by an element $x$ of a set $S$ on which $G$ acts is 
$$\{\beta_{\sigma}(x)| \sigma \in G\},$$ and is denoted by $Gx$. 

When we used the quotient principle to count circular seating
arrangements or necklaces, we partitioned up a set of lists of people or
beads into blocks of equivalent lists.  In the case of seating $n$ people
around a round table, what made two lists equivalent was, in retrospect,
the action of the rotation group $R_n$.  In the case of stringing $n$
beads on a string to make a necklace, what made two lists equivalent was
the action of the dihedral group.  Thus the blocks of our partitions were
orbits of the rotation group or the dihedral group, and we were counting
the number of orbits of the group action.  With this understanding, we
will aim to develop tools that allow us to count the number of orbits
of a group acting on a set even when the orbits have different sizes. 
First, though, we have to learn to analyze what the possible sizes of
orbits are.

\begin{figure}[htb]\caption{The four
possible results of rotating a colored square
and maintaining its position.}\label{RotationsOfColoredSquare}\smallskip
\begin{center}\mbox{\psfig{figure=RotationsOfColoredSquare.eps,width =
5.4in }}
\end{center}  
\end{figure}
 In
Figure \ref{RotationsOfColoredSquare} we show a square which has had its
vertices colored with three colors, and we show how the rotation group
$R_4$ acts on this coloring.   
If we denote our original coloring by $f$,
then $g=f\circ\rho^{-1}$ is the function with $g(1)=B$, $g(2) = R$,
$g(3)=B$, and $g(4)=G$. The other functions in the orbit containing $f$
are shown in Figure \ref{RotationsOfColoredSquare}.

\bp
\iteme Draw a figure like Figure \ref{RotationsOfColoredSquare} that
illustrates the action of the dihedral group on the function $f$ above. 
How many elements are in the orbit of $f$ under the action of dihedral
group? How many elements of the dihedral group fix $f$; that is, for how
many $\sigma \in D_4$ is $\beta_{\sigma}(f) =
f$?\label{dihedralorbits1}  
 How many elements of the dihedral group take $f$ to $g$ (where $g=f\circ
\rho^{-1}$ as earlier)?
\solution{\begin{center}\hspace*{-.4in}\mbox{\psfig{figure=MotionsOfColoredSquare.eps,width
= 5.4in }}
\end{center}
There are four elements in the orbit.  Two elements fix $f$.  Two elements
take
$f$ to
$g$.}

\itemm If $f_1(1)=R$, $f_1(2) = B$, $f_1(3) = G$ and $f_1(4) = G$, how many
elements are in the orbit of $f_1$ under the action of the dihedral group?
How many elements of the dihedral group fix $f_1$; that is, for how
many $\sigma \in D_4$ is $\beta_{\sigma}(f_1) =
f_1$?\label{dihedralorbits2}  How many elements of the dihedral group
take $f_1$ to $f_1\circ\rho^{-1}$?
\solution{8,1,1.}

\itemm If $f_2(1)=R$, $f_2(2) = B$, $f_2(3) = R$ and $f_2(4) = B$, how many
elements are in the orbit of $f_2$ under the action of the dihedral group?
How many elements of the dihedral group fix $f_2$; that is, for how
many $\sigma \in D_4$ is $\beta_{\sigma}(f_2) =
f_2$?\label{dihedralorbits3}   How many elements of the dihedral group
take $f_2$ to $f_2\circ\rho^{-1}$?
\solution{2,4,4.} 


\itemm The rotation group of the square with vertices 1, 2, 3, and 4 in
Figure \ref{cube1} can be thought of as a subgroup of the rotation group
of the cube.  Namely it corresponds to the rotations of the cube around
the axis that joins the center of the bottom face with the center of the
top face. Recall that we have used $R$ to stand for the rotation group of the
cube.  Does
$\beta_{\rho^i}(\sigma)=\rho^i\circ
\sigma$ define an action of $R_4$ on $R$?  Does $\beta_{\rho^i}(\sigma) =
\sigma\circ\rho^i$ define a group action of $R_4$ on $R$?  Does
$\beta_{\rho^i}(\sigma) =
\sigma\circ\rho^{-i}$ define a group action of $R_4$ on $R$?  For each
group action that you found, describe the orbits as simply as you
can.  How many orbits are there and what are their sizes?  (Hint:  to
do this problem, you don't need to know anything about the
rotation group of the cube except for the fact that it is a group that
has $R_4$ as a subgroup and the sizes of $R_4$ and $R$ to answer this
question.)\label{cosets1}
\solution{Yes, $\beta_{\rho^i}(\sigma)=\rho^i\circ
\sigma$ does define an action of $R_4$ on $R$. The orbits are sets of the form
$\{\rho^i\circ\sigma|i=0,1,2,3\}$ for fixed elements $\sigma$ of $R$.  (For each
fixed element $\sigma$ we get an orbit, but four different elements $\sigma$ can
give the same orbit.) Yes, 
$\beta_{\rho^i}(\sigma) =
\sigma\circ\rho^{i}$ does  define a group action of $R_4$ on $R$.  Notice
that this gives a group action only because the group $R_4$ is commutative.  The
orbits  are sets of the form $\{\sigma\circ\rho^i|i=0,1,2,3\}$ for fixed elements
$\sigma$ of $R$.  Yes, 
$\beta_{\rho^i}(\sigma) =
\sigma\circ\rho^{-i}$ does define a group action of $R_4$ on $R$.  The orbits
consist again of sets of the form $\{\sigma\circ\rho^i|i=0,1,2,3\}$ for fixed
elements $\sigma$ of $R$.  In each case the orbits have size 4 (because if
$\rho^i\circ\sigma=\rho^j\circ\sigma$, then $\rho^i =\rho^j$, and similarly for
$\sigma\circ\rho^i$), and there are six of them. }

\iteme What are the sizes of the orbits, and how many have each size, for
the action of
$D_6$ on the two element subsets of $\{1,2,3,4,5,6\}$.
\solution{There are two orbits of size 6 and one orbit of size 3.  (Think of
the two-element subsets as the edges and diagonals of a hexagon.  The edges
form one orbit of size six; the diagonals that jump over two edges form
another orbit of size six; and the diagonals connecting opposite vertices
form an orbit of size three.}

\iteme Suppose we draw identical circles at the vertices of a regular
hexagon.  Suppose we color these circles with two colors, red and blue.  We
may think of a coloring as a function from the set $\{1,2,3,4,5,6\}$ to the
set $\{R,B\}$ by numbering the vertices from one to six consecutively around
the hexagon.  To do this problem it will be helpful to use the notation
$RBRRBB$ to stand for the function $f$ with $f(1) =R$,
$f(2)=B$,
$f(3)=R$, $f(4) = R$, $f(5)=B$, $f(6)=B$.  How many functions are there from the set
$\{1,2,3,4,5,6\}$ to the set $\{R,B\}$?  These functions are partitioned into orbits by
the action of the rotation group on the hexagon.  Using our simplified notation, write
down all these orbits and observe what the possible sizes of orbits
are.\label{coloredhex} 
\solution{There are 64 functions from a six-element set to a
two-element set.

\hspace*{-.25 in}$\{RRRRRR\}$,\\ \hspace*{-.25 in}$\{RRRRRB, BRRRRR, RBRRRR, RRBRRR,
RRRBRR, RRRRBR\}$,\\
\hspace*{-.25 in}$\{RRRRBB, BRRRRB, BBRRRR, RBBRRR, RRBBRR, RRRBBR\}$,\\
\hspace*{-.25 in}$\{RRRBRB, BRRRBR, RBRRRB, BRBRRR, RBRBRR, RRBRBR\}$,\\
\hspace*{-.25 in}$\{RRRBBB, BRRRBB, BBRRRB, BBBRRR, RBBBRR, RRBBBR\}$,\\
\hspace*{-.25 in}$\{RRBRRB, BRRBRR, RBRRBR\}$,\\
\hspace*{-.25 in}$\{RRBRBB, BRRBRB, BBRRBR, RBBRRB, BRBBRR, RBRBBR\}$,\\
\hspace*{-.25 in}$\{RRBBRR, RRRBBR, RRRRBB, BRRRRB, BBRRRR, RBBRRR\}$,\\ 
\hspace*{-.25 in}$\{RRBBRB, BRRBBR, RBRRBB, BRBRRB, BBRBRR, RBBRBR\}$\\
\hspace*{-.25 in}$\{RRBBBB, BRRBBB, BBRRBB, BBBRRB, BBBBRR, RBBBBR\}$\\
\hspace*{-.25 in}$\{RBRBRB,
BRBRBR\}$\\
\hspace*{-.25 in}$\{RBBRBB, BRBBRB, BBRBBR \}$\\
\hspace*{-.25 in}$\{RBBBBB, BRBBBB, BBRBBB, BBBRBB,BBBBRB, BBBBBR\}$\\
\hspace*{-.25 in}$\{BBBBBB\}$\\
We have two orbits of size one, one of size two, two of size three, and nine
of size six. }

\iteme Either show that, when $G$ is a group acting on a set $S$, the
orbits of $G$ partition $S$ or give a counter-example.
\solution{Each element $x$ is in at least one orbit, namely $Gx$.  But if
$x\in Gy$, then $x=\sigma y$ for some $\sigma \in G$, so $$Gx=\{\tau x|\tau
\in G\} =
\{\tau\sigma y|\tau\in G\}= \{\tau' y|\tau' \in G\} = Gy.$$  Thus each
element of $S$ is in one and only one orbit of $G$, so the orbits of $G$
partition $S$.
}

\itemm In Problems \ref{dihedralorbits1}, \ref{dihedralorbits2}, and
\ref{dihedralorbits3}, what set of elements fixes $f$ and what set of
elements takes $f$ to $g$, what set of elements fixes $f_1$ and what set
of elements takes $f_1$ to $f_1\circ\rho^{-1}$, and what set of elements
fixes $f_2$ and what set of elements takes $f_2$ to
$f_2\circ\rho^{-1}$?\label{fixandmove}
\solution{In Problem \ref{dihedralorbits1}, the set of elements fixing $f$
is $\{\iota=\rho^0, \varphi_{1/3}\}$.  The set of orbits taking $f$ to $g$
is $\{\rho, \varphi_{12/34}\}$.   In Problem
\ref{dihedralorbits2} the set of elements fixing $f_1$ is $\{\iota\}$, and
the set of elements taking $f_1$ to $f_1\circ\rho^{-1}$ is $\{\rho\}$.  In
Problem \ref{dihedralorbits3}, the set of elements fixing $f_2$ is $\{\iota,\rho^2, 
\varphi_{1/3}, \varphi_{2/4}\}
$, and the set of elements taking $f_2$ to
$f_2\circ\rho^{-1}$ is
$\{\rho,
\rho^3, \varphi_{12/34}, \varphi_{14/23}\}$.
}

\itemei In Problem \ref{fixandmove} the subsets of $D_4$ that fix a function
have a special property.  What is it?\label{subgroupfixingfunction}
\solution{In each case, the subsets that fix a function form a subgroup.  To
prove this, note that the subset of group elements fixing a function is
closed under composition.}

\item Could there be a function $h$ from $\{1,2,3,4\}$ to the set $\{R, B,
G\}$ such that the orbit of $h$ under the action of $D_4$ has size 3 or
5?  If so, find one.  If not, explain why not.
\solution{This is difficult to do at this stage except by ``brute force." 
Here is an organized brute force solution.  There is no orbit of size
3 or 5.  If a function is constant, then clearly its orbit has size one.  If
a function has exactly one $R$ (or
$B$ or $G$), then its orbit under the rotations has size 4. Since every flip
is a rotation times
$\varphi_{1/3}$, if $\varphi_{1/3}$ fixes the function, then the orbit has
size 4, but if $\varphi_{1/3}$ does not take the function to the result of a
rotation, then neither does a rotation times $\varphi_{1/3}$, so the orbit
has size 8.  If the function has exactly three $R$'s, then it has exactly
one element of some other color, so the argument is the same.  If the
function has exactly two $R$'s and exactly two elements of another color,
then if the elements of the same color are adjacent, the orbit under the
rotations has size 4, and the result of a flip is the same as the result of
a rotation so the orbit has size 4.  If the two $R$'s are not adjacent, then
the orbit under the rotations has size 2 and the result of each flip is the
same as the result of a rotation, so the orbit has size 2.  If the function
has exactly two $R$'s, one $G$, and one $B$, then the size of the orbit is
the same as the orbit in Problem \ref{dihedral1} or \ref{dihedral3}. Any
other case is equivalent to one we just covered by changing the names of the
colors. Thus there are no orbits of size 3 or 5 (or 6 or 7 either).    One
can also argue as we will in Problem
\ref{orbitsize}. }
 
\itemei Make a conjecture about how the size of an orbit of the permutation
group $G$ acting on the set $S$ relates to the size of
$G$.\label{orbitsize}  Prove your conjecture.
\solution{The size of an orbit divides the size of the group.  To prove the
conjecture, the most natural approach is to try to use the quotient
principle.  This requires that we divide the group up into disjoint subsets
of the same size.  The elements of the group that fix a given element $x$
form a subgroup $H$ of $G$.  In each case we have seen, the elements that
take $x$ to $\sigma(x)$ were of the form $\sigma\circ\tau$, where $\tau$ was
in the subgroup fixing $H$.  Clearly any such element takes $x$ to
$\sigma(x)$.  The number of such elements is the size of $H$.  And if
$\sigma'(x) = \sigma(x)$, then $\sigma^{-1}\circ \sigma'$ fixes $x$ and so is
a member $\tau$ of  the subgroup. But then $\sigma'=\sigma\circ\tau$, where
$\tau=\sigma^{-1}\sigma'$. Thus the number of elements
$\sigma$ that take $x$ to an element $\sigma(x)$ is the size of $H$.  And the
set of elements that take $x$ to $\sigma(x)$ has nothing in common with the
set of elements that take $x$ to $\sigma_1(x)$, so we have partitioned up
the group into subsets each of which has the same size as $H$, and therefore
the size of $H$ time the number of subsets is the size of $G$.  However the
number of subsets is the number of orbits, and so the number of orbits
divides the size of the group.}

\iteme Make a conjecture about how the size a subgroup of the permutation
group $G$ relates to the size of $G$. \label{sizeofsubgroup}
\solution{The size of a subgroup of a permutation group $G$ divides the size
of
$G$.}
\ep

Problems \ref{alternatinggroup}, \ref{cosets1}, \ref{fixandmove} and
\ref{subgroupfixingfunction} (and less directly Problems
\ref{actiononfunctions}, \ref{actingonfunctions2} and
\ref{actingonfunctions3}) have a common theme.  

In Problem
\ref{alternatinggroup}, a natural way to show that the group of even
permutations, those with sign 1, has size $n!/2$ is to observe that for
any two-cycle (or any odd permutation) $\tau$, the element  $\sigma\tau$ is odd
whenever
$\sigma$ is even and is even whenever $\sigma$ is odd.  Thus if we let $\tau
A_n$ stand for the set
$$\tau A_n = \{\tau\sigma|\sigma \in A_n\},$$
then $A_n$ and $tA_n$ are two disjoint sets that partition $S_n$. 
\bp
\iteme Explain why, for any subgroup $H$ of a permutation group $G$, $H$
and
$\tau H$ have the same size.  \label{cosetsize}
\solution{There is a bijection between $H$ and $\tau H$ given by $f(\sigma)
= \tau \sigma$.  (It is a bijection because $\tau$ has an inverse.)}
\ep
By problem \ref{cosetsize} $A_n$ and $\tau A_n$ have the same size and
since their union has $n!$ elements, each has size $n!/2$.  The set $\tau
A_n$ is called a {\em left coset} of the subgroup $A_n$ of $S_n$. 

In Problem \ref{cosets1} you probably saw that $\beta_{\rho^i}(\sigma) =
\rho^i\circ\sigma$ and $\beta_{\rho^i}(\sigma) = \sigma\circ\rho^{-i}$
both describe actions of $R_4$ on $R$.  The orbits for the first action
are of the form
$$R_4\sigma=\{\rho^i \sigma| i=1,2,3,4\},$$
for various elements $\sigma$  of $R$.  The orbits of the second action
are
$$\{\sigma r^{-i}| i=1,2,3,4\} = \{\sigma r^i|i = 1,2,3,4\} = \sigma
R_4,$$ for various elements $\sigma$ of $R$.  The set $R_4\sigma$ is
called a {\em right coset} of $R_4$ in $R$, and as before, the set $\sigma
R_4$ is called a left coset of $R_4$ in $R$.  In general when $H$ is a
subgroup of $G$ and $\sigma \in G$, we call
$$\sigma H = \{\sigma \varphi|\varphi \in H\}$$
a {\bf left coset} of $H$ in $G$ and we call
$$H\sigma  = \{\varphi\sigma  |\varphi \in H\}$$
a {\bf right coset}\index{coset}\index{left coset}\index{right coset} of
$H$ in $G$.  Notice that $H$ is a coset of itself, since we may take
$\sigma=\iota$.  In Problem \ref{cosetsize} you showed that a subgroup
and all its (left) cosets have the same size.  Of course the right cosets
have that size too.

When a group $G$ acts on a set $S$, we say that an element $\sigma \in
G$ fixes an element $x\in S$ if $\beta_{\sigma}(x) = x$.  We say that an
element $\sigma$ takes $x$ to $y$ if $\beta_{\sigma}(x) =y$.   In Problem
\ref{subgroupfixingfunction} the subset of
$D_4$ that fixed
$f$ was a subgroup of $D_4$, the subset of $D_4$ fixing $f_2$ was a
subgroup of $D_4$ and the subset of $D_4$ fixing $f_3$ was a subgroup of
$D_4$.  The set of group elements that sent $f$ to $g$, $f_2$ to
$f_2\circ\rho^{-1}$ and $f_3$ to $f_3\circ\rho^{-1}$ was a coset of the
subset that fixed the function in each case.

\bp
\iteme Can two distinct cosets of a subgroup $H$ of a group $G$ have an
element in common?\label{cosetspartitionG}
\solution{No, because if $\tau\in \sigma H$, then $\tau H =\sigma H$
and if $\tau \in \varphi H$, then $\tau H = \varphi H$, so that $\sigma
H=\varphi H$.}

\iteme Prove your conjecture in Problem \ref{sizeofsubgroup}.
\solution{By Problem \ref{cosetspartitionG}, the cosets of a subgroup of $G$
partition $G$.  But by problem \ref{cosetsize}, the cosets all have the same
size, namely the size of $H$.  Thus by the quotient principle, the size of
$H$ is a factor of the size of $G$.}

\iteme Show that whenever a group $G$ acts on a set $S$, then for each $x$
in $S$, the set of all elements $\sigma$ fixing $x$ is a subgroup of $G$.    We will
denote this subgroup by ${\rm Fix}(x)$.
\solution{The set of elements fixing $x$ is closed under composition, and
therefore ${\rm Fix}(x)$ is a subgroup of $G$.}

\iteme Show that whenever a group $G$ acts on a set $S$, then for each $x$
and $y$ the set of all $\tau \in G$ such that $\beta_{\tau}(x) = y$
is a coset of the subgroup ${\rm Fix}(x)$ of all elements of $G$ fixing $x$.
\solution{First, if $\beta_{\tau}(x)=y$ and $\beta_{\sigma}$ fixes $x$, then
$\beta_{\tau}\circ \beta_{\sigma}=\beta_{\tau\circ\sigma}$ takes $x$ to
$y$.  Second, if $\beta_{\varphi}(x)=y$ as well, then
$\beta_{\tau^{-1}}\circ\beta_{\varphi}=\beta_{\tau^{-1}\circ\varphi}$
fixes $x$ and so is in ${\rm Fix}(x)$, so that $\varphi$ is $\tau$ times
an element (namely $\tau^{-1}\varphi$) of ${\rm Fix}(x)$.  Thus every element of the
coset
$\tau{\rm Fix}(x)$ takes $x$ to $y$, and every element that takes $x$ to $y$ is an
element of the coset $\tau{\rm
Fix}(x)$.  Thus the set of elements $\tau$ such that $\beta_{\tau}(x)=y$
is a coset of ${\rm Fix}(x)$.}


\iteme When a group $G$ acts on a set $S$, how is the number of elements
in the orbit containing the element $x$ of $S$ related to the size of $G$
and the size of ${\rm Fix}(x)$?  Find a bijection that proves that what
you say is correct.
\solution{The number of elements in the orbit containing the element $x$ is
the number of  cosets of ${\rm Fix}(x)$, which is also
$|G|/|{\rm Fix}(x)|$.  A bijection that proves this is the function $f$
that takes the coset $\varphi{\rm Fix}(x)$ to the element $\varphi(x)$. 
This function is well-defined because all the elements in $\varphi{\rm
Fix}(x)$ take $x$ to $\varphi(x)$.  It is onto because if $\tau(x) =y$,
then the coset $\tau{\rm Fix}(x)$ is mapped to $y$.  It is one-to-one
because if $f(\varphi{\rm Fix}(x))=f(\tau{\rm Fix}(x))$, then 
$\varphi\in \tau{\rm Fix}(x)$ and $\tau \in \varphi{\rm Fix}(x)$, which
means the two cosets are identical.  The cosets all have the same size,
namely $|{\rm Fix}(x)|$ and therefore by the quotient principle, the number
of cosets is $|G|/|{\rm Fix}(x)|$.}
\ep

We can summarize our findings in this section with a theorem.

\begin{theorem}  When a group $G$ acts on a set, the set ${\rm Fix}(x)$ of
group elements fixing any given any given element $x$ of $S$  is a
subgroup of $G$.  The set of group elements $\tau$ such that
$\beta_{\tau}(x) = y$ is a coset of ${\rm Fix}(x)$, and the size of the
orbit of $x$ under the action of $G$ is $|G|/|{\rm Fix}(x)|$.\end{theorem}

\subsection{Multiorbits}
\bp 
\item Draw  figures like Figure \ref{RotationsOfColoredSquare} to show
the eight results of acting with the dihedral group on the squares
corresponding to the functions $f$, $f_1$, and $f_2$ of Problems
\ref{dihedralorbits1},
\ref{dihedralorbits2}, and $\ref{dihedralorbits3}$. \label{multiorbits1}
\solution{\begin{center}\hspace*{-.4in}\mbox{\psfig{figure=
MotionsOfColoredSquare.eps,height=1.45in }}
\end{center}
\begin{center}\hspace*{-.4in}\mbox{\psfig{figure=
MotionsOfColoredSquare1.eps,height=1.45 in }}
\end{center}
\begin{center}\hspace*{-.4in}\mbox{\psfig{figure=
MotionsOfColoredSquare2.eps,height= 1.45in}}
\end{center}}
\ep

As Problem \ref{multiorbits1} shows, in some ways it is more natural to
think of an orbit as a multiset rather than a set.  For example with the
function $f_2$ from Problem \ref{dihedralorbits3} we get $f_2$ itself four
times when we act with the dihedral group and we get $f_2\circ\rho^{-1}$
four times when we act on $f_2$ with the dihedral group.  If we think
only about the orbit of the dihedral group, then, we lose some
information.  To avoid this, we define the {\bf multiorbit} of an element
$x$ of a set $S$ under the action of a group $G$ to be the multiset
$$Gx_{\mbox{\scriptsize multi}} = \{\beta_{\sigma}(x)|\sigma \in
G\}_{\mbox{\scriptsize multi}}.$$
It is immediate that the size of each multiorbit is then exactly the size
of
$G$. 

\bp
\iteme In how many different multiorbits of the action of $G$ on $S$ will
a given element of $S$ appear?\label{NumberContainingx}
\solution{Only one, because the orbits partition $S$, and there is a
bijection between the orbits and the multiorbits given by $f(Gx) =
Gx_{\mbox{\scriptsize multi}}$.}

\iteme What will the multiplicity of $x$ be in
$Gx_{\mbox{\scriptsize multi}}$?
\solution{$|{\rm Fix}(x)|$.}
\ep

It is also immediate from Problem \ref{NumberContainingx} that the number
of orbits is the number of multiorbits, because we get the orbits from the
multiorbits by deleting repeated elements, and the problem shows that
each orbit corresponds to exactly one multiorbit\footnote{You might want to
argue that we have a multiset of multiorbits, one for each element of $S$,
 but that would defeat our purpose since
we have  bijection between the set of orbits and the {\em set} of
multiorbits, and this bijection will let us count the number of orbits by
computing the size of the set of multiorbits.}.  We define the union of multisets
$M_1$,
$M_2$,
\ldots,
$M_n$ to be the multiset in which the multiplicity of an element $x$ is
the sum of its multiplicities in the individual multisets $M_i$.  Thus
$$\{a,a,b,b,b\}\cup \{a,b,b,c,c,c\}=\{a,a,a,b,b,b,b,b,c,c,c\}.$$  We will
now get a formula for the number of multiorbits by using the fact that
they all have the same size and the idea of multiset union.

\bp
\iteme How does the size of the union of the set of multiorbits of a group
$G$ acting on a set $S$ relate to the number of multiorbits and the size
of
$G$?\label{numbermultiorbits1}
\solution{It is simply the product of the number of multiorbits and the size
of $G$.}

\iteme How does the size of the union of the set of multiorbits of a group
$G$ acting on a set $S$ relate to the numbers $|{\rm
Fix}(x)|$?\label{numbermultiorbits2}
\solution{Since it is the sum of the multiplicities of the elements of $S$
in the union, it is the sum of $|{\rm Fix}(x)|$ over all $x$ in $S$.}
\iteme In Problems \ref{numbermultiorbits1} and \ref{numbermultiorbits2}
you computed the size of the union of the set of multiorbits of a group
$G$ acting on a set $S$ in two different ways, getting two different
expressions which must be equal.  Write the
equation that says they are equal and solve for the number of multiorbits, and
therefore the number of orbits.\label{numbermultiorbits3}
\solution{Using $m$ for the number of multiorbits, we get 
$$m|G|=\sum_{x:x\in S} |{\rm Fix}(x).$$  Therefore, $$m={1\over
|G|}\sum_{x:x\in S} |{\rm Fix}(x)|.$$}\ep

\subsection{The Cauchy-Frobenius-Burnside Theorem}
\bp
\iteme In Problem \ref{numbermultiorbits3} you stated and proved a theorem
that expresses the number of orbits in terms of the number of group
elements fixing each element of $S$.  It is often easier to find the
number of elements fixed by a given group element than to find the number
of group elements fixing an element of $S$.  For this purpose,
  \begin{enumerate} 
\item Let $\chi(\sigma,x)=1$ if $\sigma(x)=x$ and let
$\chi(\sigma,x) =0$ otherwise.  Use $\chi$ to convert the single summation
in Problem
\ref{numbermultiorbits3} into a double summation over elements $x$ of $S$
and elements $\sigma$ of $G$.
\solution{$$m={1\over |G|}\sum_{x:x\in S}\sum_{\sigma:\sigma\in G}
\chi(\sigma,x).$$}
\item Reverse the order of the previous summation in order to convert it
into a single sum involving the function $\chi$ given by\footnote{The reason
for using the Greek letter $\chi$ is that $\chi(\sigma)$ is an example of
what is called a group character.  Character theory is a major ingredient of
the representation theory of groups, an advanced subject that combines
abstract beauty with amazing utility.}
$$\chi(\sigma) =
\mbox{the number of elements of $S$ left fixed by $\sigma$}.$$
\solution{$$m={1\over |G|}\sum_{\sigma:\sigma \in G} \chi(\sigma).$$}
\end{enumerate}\label{numbermultiorbits4}


\ep

In Problem \ref{numbermultiorbits4} you gave a formula for the number of
orbits of a group $G$  acting on a set $X$.  This formula was first
worked out by Cauchy in the case of the symmetric group, and then for
more general groups by Frobenius.  In his pioneering book on Group
Theory, Burnside used this result as a lemma, and while he attributed the
result to Cauchy and Frobenius in the first edition of his book, in later
editions, he did not.  Later on, other mathematicians who used his book
named the result ``Burnside's Lemma,"\index{Burnside's Lemma} which is the
name by which it is still most commonly known.  Let us agree to call this
result the Cauchy-Frobenius-Burnside
Theorem,\index{Cauchy-Frobenius-Burnside Theorem} or CFB Theorem for short
in a compromise between historical accuracy and common usage.

  \bp 
\itemi In how many ways may we string four (identical) red, six (identical)
blue, and seven (identical)  green beads on a necklace?
\solution{We are stringing 17 beads on our necklace, so we are asking for
the number of orbits of the group $D_{17}$ on lists of four $R$s, six $B$s,
and seven $G$s.  For a rotation $\rho^i$ to fix a list, it must take an $R$
to a
$R$.  The powers of $\rho^i$ form a group.  Thus the set of places that
contain an
$R$ in a list that is fixed by
$\rho^i$ must be an orbit of that group.  But the size of the orbit must be
a divisor of the size of the subgroup, which must be a divisor of seventeen,
so the size of the orbit is one or 17.  If it is 1, then $i=0$.  Thus no
elements are fixed by any nontrivial rotation, and all $17\choose 4,6,7$ 
lists of $R$s, $B$s and $G$s are fixed by the identity.  Each flip will be a
flip \strut around a line from a bead to the spot between the two ``opposite"
beads.  This line divides the list into the eight beads on its left, the
eight beads on its right and the one bead it goes through.  If a flip fixes
a list, the eight beads on the left must be identical to the eight beads on
the right, meaning that the bead the line goes through must be one of the
seven green beads.  Thus the number of lists that are fixed by a flip is the
number of ways to place three green beads, two red beads, and three blue
beads in 8 slots, which is
$8\choose3,2,3$, or $8!\over3!2!3!$.  There are 17 flips, so there are
$17\cdot8!\over3!2!3!$ elements fixed by flips.  Therefore we have $${1\over
2\cdot 17}\left({17!\over 4!6!7!} + {17\cdot8!\over 3!2!2!}\right) =
{8\cdot15!\over 4!6!7!} + {4\cdot7!\over 3!2!3!}= 120,400$$ necklaces.  The
numerical answer, which is unimportant here, was obtained from Maple.}

\itemi If we have an unlimited supply of identical red beads and identical
blue beads, in how many ways may we string 17 of them on a necklace?
\solution{We are asking for the number of orbits of $D_{17}$ on the set of
functions from $[17]$ to $\{R,B\}$.   Every function is fixed by the
identity.  The only functions fixed by a nontrivial rotation are the constant
functions.  Each flip is around a line from a bead to the space between the
two ``opposite" beads.  The bead the line goes through can be either color,
and then the eight beads to the left of this one must be identical to the
eight beads to the right of this one.  There are $2^8$ ways to assign beads
to the positions on the left, so a flip fixes $2^9$ functions.  Therefore by
the CFB theorem, we have
$${1\over 2\cdot 17}\left(2^{17}+ 17\cdot2^9+16\cdot 2\right)={2^{16}\over
17}+2^8 +{2^4\over 17}=16{4097\over17}+256=4112$$ necklaces.}

\itemi If we have five (identical) red, five (identical)  blue, and five
(identical) green beads, in how many ways may we string them on a
necklace?
\solution{Here we need to consider the action of $D_{15}$ on lists
(functions from $[15]$ to $\{R,B,G\}$) with  five $R$s, five $B$s, and five
$G$s. The identity will fix $15\choose5,5,5$ lists.  A rotation through
3,6,9,or 12 places will fix any function that has the same value in places
1, 4, 7, 10, and 13, the same value in places 2, 5, 8, 11, and 14, and the
same value in places 3, 6, 9, 12, and 15.  There are $3!$ such lists.  There
is no other rotation that fixes any lists.  Each flip is around an axis that
goes from bead to the space between two opposite beads.  If it fixed a list,
the seven colors to the left of the axis would have to equal the seven
colors to the right of the axis.  Thus the number of beads of each color on
the left and right sides would have to be equal.  So except for the color of
the bead the axis goes through, we would have to have an even number of
beads of each color for a flip to fix a list of beads.  Thus no flip fixes
any functions. Therefore by the CFB theorem we have
$${1\over2\cdot15}\left({15!\over5!5!5!} +4\cdot 3!\right)=25,226$$
necklaces.  (The numerical answer, which was obtained from Maple, is not
important here.)}


\itemi In how many ways may we paint the faces of a cube with six
different colors, using all six?
\solution{Here we must consider the action of the rotation group of the cube
on lists of six distinct colors.  But no nontrivial rotation will fix a cube
with all its faces colored differently.  The identity rotation will fix all
$6!$ lists, and there are 24 members of the rotation group, so we have
$6!/24=6\cdot5=30$ ways to paint the faces of a cube with six distinct
colors, using each color}

\item In how many ways may we paint the faces of a cube with two colors
of paint?  What if both colors must be used?
\solution{We must consider the action of the rotation group of the cube on
functions from $[6]$ to  $\{R,B\}$ or some other two-element set of colors. 
There are five kinds of elements in the rotation group of the cube.  There
is one identity, there are six rotations by 90 degrees or  270 degrees around
an axis connecting the centers of two opposite faces, there are three
rotations of 180 degrees around such an axis, there are 8 rotations (of 120
degrees and 240 degrees, respectively) around an axis connecting two
diagonally opposite vertices, and there are 6 rotations of 180 degrees
around an axis connecting the centers of two opposite edges.  The identity
fixes $2^6$ functions.  There are eight functions fixed by a 90 degree or
270 degree rotation.  There are 16 functions fixed by a 180 degree rotation
along an axis through two faces.  There are 8 functions fixed by a 180
degree rotation along an axis joining the centers of two opposite sides. 
There are 4 functions fixed by a 120 degree or 240 degree rotation.  Thus by
the CFB theorem, we have
$${1\over24}(64+6\cdot8 + 3\cdot16+ 8\cdot 4 +6\cdot8)= 10$$ ways to paint
the faces of a cube with two colors of paint. Two of these colorings use
only one color, so there are eight colorings that use both colors.}

\itemi In how many ways may we color the edges of a (regular) $(2n+1)$-gon
free to move around in the {\em plane} (so it cannot be flipped) if  we use
red $n$ times and blue
$n+1$ times?  If this is a number you have seen before, identify
it.\label{Catalancircle}\index{Catalan Number!(Problem
\ref{Catalancircle})}
\solution{The set of all powers of a rotation is a subgroup of the
rotation group of the $(2n+1)$-gon.  If a given rotation fixes a
coloring, all powers of that rotation fix the coloring.  The set of
edges to which a given edge is taken by that rotation is an orbit of the
group of powers of the rotation.  The size of this orbit is a divisor of
the size of the subgroup, which is a divisor of $2n+1$.  If the edge is
colored red, the size of the orbit is also a divisor of $n$, and if the
edge is colored blue, the size of the orbit is also a divisor of $n+1$. 
But neither $n$ nor $n+1$ have common divisors with $2n+1$, except for
one.  Therefore the only rotation that fixes a coloring is the identity
rotation, and it fixes all $2n+1\choose n$ colorings.  Thus the number of
orbits is $${1\over 2n+1}{2n+1 \choose n} = {1\over n+1}{2n\choose n},$$ a
Catalan number.}

\itemih In how many ways may we color the edges of a (regular) $(2n+1)$-gon
free to move in  {\em three-dimensional space} so that $n$ edges are 
colored red and
$n+1$ edges are colored blue.  Hint: your answer may depend on whether $n$ is even or
odd.
\solution{
The set of all powers of a rotation is a subgroup of the
dihedral group of the $(2n+1)$-gon.  If a given rotation fixes a
coloring, all powers of that rotation fix the coloring.  The set of
edges to which a given edge is taken by that rotation is an orbit of the
group of powers of the rotation.  The size of this orbit is a divisor of
the size of the subgroup, which is a divisor of $2n+1$.  If the edge is
colored red, the size of the orbit is also a divisor of $n$, and if the
edge is colored blue, the size of the orbit is also a divisor of $n+1$. 
But neither $n$ nor $n+1$ have common divisors with $2n+1$, except for
one.  Therefore the only rotation that fixes a coloring is the identity
rotation, and it fixes all $2n+1\choose n$ colorings.  

A flip, on the
other hand can fix some colorings.  In particular, if $n$ is even, we
color one edge blue, leaving an even number $n$ of edges to be colored
blue and an even number $n$ of edges to be colored red.  It the flip over
the axis perpendicular to the side we picked fixes the coloring, the $n$
edges to the right of the chosen edge must be colored identically with
the $n$ edges to the left of the chosen edge.  There are $n\choose n/2$
ways to color the $n$ edges to the right of the chosen edge (choose which
edges get red), and so this is the number of colorings fixed by this
flip.  We have $2n+1$ flips, and since we have one flip of the type described for each
edge, all flips have the form given above. Thus each flip fixes
$n\choose n/2$ colorings.

Thus the number of orbits is $${1\over 2(2n+1)}\left({2n+1\choose n} +
(2n+1){n\choose n/2}\right).$$
If $n$ is odd, we color one edge red, leaving an even number $n-1$ of edges to be
colored red and an even number $n+1$ of edges to be colored blue.  With an argument
similar to the previous one we see that there are
$${1\over 2(2n+1)}\left({2n+1\choose n} + (2n+1)({n\choose (n-1)/2}\right)$$ orbits. 
}


\itemih (Not unusually hard for someone who has worked on chromatic
polynomials.) How many different  proper colorings with four colors are there
of the vertices of a graph which is cycle on five vertices? (If we get one
coloring by rotating or flipping another one, they aren't really different.)
\solution{We are asking for the number of orbits of $D_5$ on lists of five 
colors chosen from the given four with no two adjacent colors equal; we consider the
first and last position adjacent as well.  The identity fixes all such colorings. 
For those who are familiar with the chromatic polynomial, as in Problem
\ref{chrompolydel/cont}, the number of such colorings is the chromatic
polynomial of the cycle on five vertices evaluated at 4.  The number of ways
to properly color a cycle on five vertices is the number of ways to color a
path on five vertices minus the number of ways to color a path on five
vertices so that its first and last vertices are identical, which is the
number of ways to color a cycle on four vertices.  The number of ways to
color a cycle on four vertices is the number of ways to color a path on four
vertices minus the number of ways to color a path on four vertices so that
the first and last vertices are the same color, which is the same as the
number of ways to color a cycle on three vertices.  The number of ways to
properly color a path on five vertices with four colors is $4\cdot3^4$.  The
number of ways to properly color a path on four vertices with four colors is
$4\cdot3^3$.  The number ways to properly color a cycle on three vertices
with four colors is $4\cdot3\cdot2$.  Thus the number of proper colorings of
a five vertex cycle with four colors is $4\cdot3^4-4\cdot3^3
+4\cdot3\cdot2=240$.  This is the number of colorings fixed by the
identity.  No proper coloring is fixed by a nontrivial rotation, because
if a rotation of the five cycle fixes a coloring, all the vertices must
have the same color.  No proper coloring is fixed by a flip, because for a
flip to fix a coloring, the coloring must give the same color to two
adjacent vertices.  Thus the number of really different proper four-colorings
of a cycle on 5 vertices is ${240\over 10}=24$.}

\itemih How many different  proper colorings with four colors are there of the
graph in Figure \ref{starhexagon}?  Two graphs are the same if we
can redraw one of the graphs, not changing the vertex set or edge set, so that it is
identical to the other one.  This is equivalent to permuting the vertices in some
way so that when we apply the permutation to the endpoints of the edges to get a new
edge set, the new edge set is equal to the old one.  Such a permutation is called an
{\em automorphism}\index{automorphism (of a graph)} of the graph.  Thus two
colorings are different if there is no automorphism of the graph that carries one to
the other one.
\begin{figure}[htb]\caption{A graph on six vertices.}\label{starhexagon}\smallskip
\begin{center}\mbox{\psfig{figure=NumberedHexagonalNetwork.eps
}}
\end{center}  
\end{figure}
\solution{
We want the number of orbits of the set of proper colorings under the action of the group
of automorphisms of the graph.  An automorphism $\sigma$
maps vertex 1 to any of six vertices.  Vertex 2 can be mapped to any of the four
vertices adjacent to the image of vertex 1.  Vertex 3 is adjacent to vertices 1 and
2, so it must be mapped to a vertex adjacent to the images of both vertex 1 and
vertex 2; by checking cases you can see that there are always exactly two vertices
adjacent to the images of vertex 1 and vertex 2, and mapping vertex 3 to either of
these vertices preserves all the edges among vertices 1, 2, and 3.  However each of
the other three vertices is adjacent to exactly two vertices of the set
$\{1,2,3\}$, and thus it must be mapped to the unique vertex adjacent to the
corresponding two of
$\sigma(1)$,
$\sigma(2)$ and $\sigma(3)$.  (It is always the case that each vertex is adjacent to
exactly two of these, as you can see by considering the cases with $\sigma(1)=1$.)  Thus
there are $6\cdot4\cdot2=48$ elements in the group.  Now to apply the CFB theorem we would
need to know how many proper colorings are fixed by each group element, so we would need
to know what the group elements are.   We have observed that a permutation that preserves
the edges is determined by where the triangle $\{1,2,3\}$ goes.  We can see eight
triangles in the graph, triangles of the form $\{i,i+1,i+2\}$, where we identify 7 with 1
and 8 with 2, and the triangles $\{1,3,5\}$ and $\{2,4,6\}$.  We can map the set
$\{1,2,3\}$ to any of these eight sets by six one-to-one maps, so each group element is
determined uniquely by one of these mappings.   However focusing on these triangles
makes our job here simpler in another way.  In a proper coloring, vertices 1, 2, and
3 must be colored differently.  We have four choices for the color of vertex 1,
three different ones for vertex 2 and two still different ones for vertex 3, so
there are 24 ways to color this triangle.  Clearly the only difference among these
ways is the actual names of the colors.  That is, we can assume that vertex 1 is
colored red, vertex 2 is colored blue and vertex 3 is colored green, then determine
the proper colorings starting with these three colors, and up to changing the names
of the colors, we will have determined all the proper colorings.  Then we can ask
which group elements fix a coloring rather than which colorings are fixed by a group
element.  This turns out to be easier.  Let us write
$RBGRBG$ for the coloring that colors vertices 1 and 4 red, two and five blue, and three
and six green. An examination of the figure shows that this is a proper coloring.  In
fact, it is the only proper coloring that starts $RBG$ and uses only three colors.  Suppose
we were to use a fourth color, $Y$.  Then among vertices 4, 5, and 6, it could be used in
just one place, because those three vertices are mutually adjacent.  Each of the other two
vertices is adjacent to two of the original three vertices colored $RBG$, and so there is
only one color available to use on it.  In summary, the colorings that start $RBG$ are
\begin{enumerate}\item $RBGRBG$\label{coloring1}
\item $RBGYBG$\label{coloring2}
\item $RBGRYG$\label{coloring3}
\item $RBGRBY$.\label{coloring4}
\end{enumerate}
Thus for any of the 24 choices of colorings of the first three vertices, there are four
ways to complete it to a proper coloring of the whole graph, so there are 96 proper
colorings of the graph.  Among the ones that start RGB, let us analyze which group
elements fix them.  Note that the 2-cycles $(1\ 4)$, $(2\ 5)$ and $(3\ 6)$ are all
permutations that fix coloring one. Further, interchanging vertices 1 and 4 does not
change the endpoints of any edges, nor does interchanging 2 and 5 nor 3 and 6.  So all
these two cycles are automorphisms of the graph.  A composition of automorphisms must be an
automorphism(this follows directly from the definition of automorphism) and so the eight
permutations $\iota$, $(1\ 4)$, $(2\ 5)$, $(3\ 6)$, $(1\ 4)(2\ 5)$, $(1\ 4)
(3\ 6)$, $(2\ 5)(3\ 6)$, and $(1\ 4)(2\ 5)(3\ 6)$ all are automorphisms of the graph, and
they are all in the subgroup of the automorphism group that fixes the coloring in
\ref{coloring1}.  Any permutation not in the list will take some vertex to a vertex of
another color, and so the eight permutations we listed are the subgroup fixing the
coloring in \ref{coloring1}.  The subgroup fixing the coloring in \ref{coloring2} is
$\iota$, $(1\ 3)$, $(2\ 4)$, and $(1\ 3)(2\ 4)$. The subgroups fixing the colorings in
\ref{coloring3} and \ref{coloring4} also have size 4.  Thus there are $8+12=20$ pairs of a
coloring with $R$, $B$, and $G$, in that order, on vertices 1, 2, and 3 and an automorphism
fixing that coloring.  Since there are $4\cdot3\cdot2=24$ ways to color vertices 1, 2, and
3 properly, and each gives rise to 20 pairs of a proper coloring and an automorphism fixing
that coloring, there are $20\cdot24=480$ such pairs.  Since the automorphism group has
size 48, this means that there are 10 proper colorings of this graph, up to
automorphisms.  Note that we did not really use the CFB theorem, though we did use the
fact that its formula is proved by dividing the number of ordered pairs of a set member
and a group element fixing that member by the size of the group.}




\ep

\section{P\'olya-Redfield Enumeration Theory}
George P\'olya and Robert Redfield independently developed a theory of
generating functions that describe the action of a group $G$ on functions
from a set $S$ to a set $T$ when we know the action of $G$ on $S$.  P\'olya's
work on the subject is very accessible in its exposition, and so the
subject has become popularly known as P\'olya theory, though
P\'olya-Redfield theory would be a better name.  In this section we
develop the elements of this theory.


Our language will be more intuitive if we think of $T$ as a set of ``colors." 
To illustrate that using this language is not restrictive, the set $S$ might be
the positions in a hydrocarbon molecule which are occupied by hydrogen, and the
group could be the group of spatial symmetries of the molecule (that is, the group of
permutations of the atoms of the molecule that move the molecule around so that in its
final position the molecule cannot be distinguished from the original molecule).  
The colors could then be radicals (including hydrogen itself) that we could
substitute for each hydrogen position in the molecule.  Then the number of orbits
of colorings is the number of chemically different compounds we could create by
using these substitutions.\footnote{There is a fascinating subtle issue of what
makes two molecules different.  For example, suppose we have a molecule in the
form of a cube, with one atom at each vertex.  If we interchange the top and
bottom faces of the cube, each atom is still connected to exactly the same atoms
as before.   However we cannot achieve this permutation of the vertices by a
member of the rotation group of the cube.  It could well be that the two versions
of the molecule interact with other molecules in different ways, in which case we
would consider them chemically different.  On the other hand if the two versions
interact with other molecules in the same way, we would have no reason to
consider them chemically different.  This kind of symmetry is an example of what is
called {\em chirality} in chemistry.}

So think intuitively about some ``figure'' that has places to be colored. 
(Think of the faces of a cube, the beads on a necklace, circles at the
vertices
of an $n$-gon, etc.)  How can we picture the coloring?  If we number the places
to be colored, say 1 to $n$, then a function from $[n]$ to the colors is
exactly
our coloring; if our colors are blue, green and red,  then $BBGRRGBG$ describes
a
typical coloring of 8 such places.  Unless the places are somehow ``naturally''
numbered, this idea of a coloring imposes structure that is not really there. 
Even if the structure is there, visualizing  our colorings in this way doesn't
``pull together'' any common features of different colorings; we are simply
visualizing all possible functions. We have a group (think of it as symmetries
of
the figure you are imagining) that acts on the places.  That group then acts in
a
natural way on the colorings of the places and we are interested in orbits of
the
colorings.  Thus we want a picture that pulls together the common features of
the
colorings in an orbit.   One way to pull together similarities of colorings
would
be to let the letters we are using as pictures of colors commute as we did with
our pictures in Chapter 4; then our picture
$BBGRRGBG$ becomes $B^3G^3R^2$, so our picture now records simply how many
times
we use each color. If you think about how we defined the action of a group on a
set of functions, you will see that a group element won't change how many times
each color is used;
it simply moves colors to different places.  Thus the picture we now have of a
given coloring is an equally appropriate picture for each coloring in an orbit. 
One natural question for us to ask is ``How many orbits have a given picture?'' 
We can think of a multivariable generating function in which the letters we use
to picture individual colors are the variables, and the coefficient of a
picture
is the number of orbits with that picture.  Such a generating function is an
answer to our natural question, and so it is this sort of generating function
we
will seek.  Since the CFB theorem was our primary tool for saying how many
orbits
we have, it makes sense to think about whether the CFB theorem has an analog in
terms of pictures of orbits.

\subsection{The Orbit-Fixed Point Theorem}

\bp


\iteme   Suppose that $P_1$ and $P_2$ are picture functions on sets $S_1$ and
$S_2$ in the sense of Section \ref{picturefunction}. Define $P$ on $S_1 \times
S_2$ by
$P(x_1,x_2) = P_1(x_1)P_2(x_2)$.  How are
$E_{P_1}$, $E_{P_1}$, and $E_{P}$ related?  (Hint: you may have already done this
problem in another context!)\label{ProdPrincPictureEnumerators}
\solution{\begin{eqnarray*}E_P(S_1\times S_2)&=&\sum_{x_1\in S_1,\ x_2\in
S_2}P(x_1)P(x_2)=\sum_{x_1\in S_1}\sum_{x_2 \in S_2}P(x_1)P(x_2)\\&=& \sum_{x_1\in
S_1}P_1(x_1)\sum_{x_2 \in S_2}P_2(x_2)=E_{P_1}(S_1)E_{P_2}(S_2).\end{eqnarray*}}

\iteme   Suppose $P$ is a picture function on a set $T$.  Suppose that we define
the picture of a function from some other set $S$ to the set $T$ to be the
product
of the pictures of the values of $f$, i.e.

$$\hat P(f) = \prod_{x:x\in S}P(f(x)).$$

How does the picture enumerator $E_{\hat P}$ of the set $T^S$ of all functions
from
$S$ to $T$ relate to the picture enumerator of $P$ on the set $T$?  (You may assume
that both
$S$ and
$T$ are finite.)\label{PictureEnumeratorforFunctions}
\solution{$E_{\hat P}(T^S)=E_P(T)^{|S|}$.  To prove this, note that we can think of a
function $f$ from $S$ to $T$ as an $|S|$-tuple of values, $$(f(x_1),f(x_2),\ldots,
f(x_{|S|}),$$ then think of an $|S|$-tuple an the ordered pair $$(f(x_1),(f(x_2),
f(x_3),\ldots f(x_{|S|})),$$ and apply induction and Problem
\ref{ProdPrincPictureEnumerators}.}

\iteme   Suppose now we have a  group $G$ acting on a set and we have a picture
function on that set with the additional feature that for each orbit of the
group, all its elements have the same picture. In this circumstance we define
the
picture of an orbit or multiorbit to be the picture of any one of its members. The
{\em orbit
enumerator}\index{orbit enumerator} ${\rm Orb}(G,S)$ is the sum of all the pictures
of all the orbits.  The
{\em fixed point enumerator}\index{fixed point enumerator} ${\rm Fix}(G,S)$ is
the sum of all the pictures of all
the fixed points of all the elements of $G$.  We are going to construct a
generating function analog of the CFB theorem. The main idea of the proof of
the
CFB theorem was to try to compute in two different ways the number of elements
(i.e. the sum of all the multiplicities of the elements) in the union of all
the
multiorbits of a group acting on a set.  Suppose instead we try to compute the
sum of all the pictures of all the elements in the union of the multiorbits of
a
group acting on a set.  By thinking about how this sum relates to ${\rm Orb}(G,S)$
and
${\rm Fix}(G,S)$, find an analog of the CFB theorem that relates these two
enumerators. 
State and prove this theorem.\label{Orbit-FixedPoint}
\solution{
Let $E$, for enumerator, be the sum of all the pictures of all the elements in the
union of the multiorbits of $G$ acting on a set $S$. Recall that for any multiorbit
$M$ the picture
$P(M)$ is the picture $P(x)$ of any element $x$ of $M$, and the number of elements
of a multiorbit $M$ is always the size of $G$. This lets us write
\begin{eqnarray*}
E&=&\sum_{M:M\mbox{\ is a multiorbit of $G$\ \ }}\sum_{x:x\in M} P(x)\\
&=&\sum_{M:M\mbox{\ is a multiorbit of $G$} }|G|P(M)\\
&=&|G|\sum_{M:M\mbox{\ is a multiorbit  of $G$}} P(M)\\
&=&|G|{\rm Orb}(G,S)
\end{eqnarray*}
Recall also that the multiplicity of an element $x$ in its multiorbit, and thus in
the union of the multiorbits is $|Fix(x)|$.  This lets us write
\begin{eqnarray*}E&=&\sum_{x: x\in S} {\rm|Fix}(x)|P(x)\\
&=&\sum_{x:x\in S}\sum_{\sigma: \sigma \in G} \chi(\sigma,x)P(x)\\
&=&\sum_{\sigma: \sigma\in G}\sum_{x: \sigma{x}=x} P(x)\\
&=& {\rm Fix}(G,S).  
\end{eqnarray*}
Setting these two values of $E$ equal and solving for ${\rm Orb}(G,S)$ gives  us
$${\rm Orb}(G,S) ={1\over |G|}{\rm Fix}(G,S).$$}


\iteme  We will call the theorem of  Problem
\ref{Orbit-FixedPoint}\index{Orbit-Fixed Point Theorem} the {\em Orbit-Fixed Point
Theorem}.  Use it  to determine the Orbit Enumerator for the colorings, with two
colors (red and blue), of six circles placed at the vertices of a hexagon which is
free to move in the plane.  Compare the coefficients of the resulting polynomial
with the various orbits you found in Problem \ref{coloredhex}.\label{polya1}
\solution{Let us take the  pictures of red and blue to be  $R$ and
$B$.  Since the hexagon is free to move in the plane, our group
$G$ is the group
$R_6$ of rotations of a regular hexagon. There are four kinds of elements of $G$: the
identity, the rotation $\rho$ through 60 degrees and the rotation $\rho^5$ through
300 degrees (when written as a product of disjoint cycles, the corresponding
permutations are six-cycles), the rotations $\rho^2$ and $\rho^4$ through 120 and
240 degrees respectively (when written as a product of disjoint cycles, the
corresponding permutations are products of two three cycles), and the rotation
$\rho^3$ through 180 degrees (when written as a product of disjoint cycles this is a
product of three two-cycles).  All $2^6$ functions are fixed by the identity. By
Problem \ref{PictureEnumeratorforFunctions} the enumerator for these functions is
$(R+B)^6$.  Only the two constant functions (that color every circle the same color)
are fixed by
$\rho$ or $\rho^5$. The picture enumerator of the two constant functions is
$R^6+B^6$. If the vertices are numbered one through six clockwise, then
$\rho^2$ is $(1\ 3\ 5)(2\ 4\ 6)$. Thus for a function $f$ to be fixed by $\rho^2$,
$f(1)=f(3)=f(5)$ and $f(2)=f(4)=f(6)$.   Thus the
color of vertex 1 is repeated on vertex 3 and 5, and the color of vertex 2 is
repeated on vertices 4 and 6.  Therefore the picture enumerator for functions fixed
by
$\rho^2$ and $\rho^4$ is $R^3R^3 +R^3B^3+B^3R^3 +B^3B^3=(R^3+B^3)^2$.  We can also
think of this as the picture enumerator for functions defined on the cycles of the
permutation.  There are two possible pictures of a colored cycle, $R^3$ and $B^3$,
so the picture enumerator for one cycle is $R^3+B^3$.  That is, the picture of
assigning red to a cycle is $R^3$ and the picture of assigning blue to a cycle is
$B^3$.  By Problem
\ref{PictureEnumeratorforFunctions}, the picture enumerator for functions defined on
the two cycles is then $(R^3+B^3)^2$. If a function
$f$ is fixed by
$\rho^3=(1\ 4)(2\ 5)(3\ 6)$, then $f(1)=f(4)$, $f(2)=f(5)$, and $f(3)=f(6)$, so $f$
is determined by the three values $f(1)$, $f(2)$, and $f(3)$.   The picture
enumerator for these functions is, by Problem
\ref{PictureEnumeratorforFunctions} applied to colorings of the cycles,
$(R^2+B^2)^3$.  Therefore the fixed point enumerator for the action of $G$ on the
colorings is
$${\rm Fix}(G,S)= (R+B)^6 +(R^2+B^2)^3 +2(R^3+B^3) + 2(R^6+B^6).$$  Then the orbit
enumerator for the action of $G$ on the colorings is
$${\rm Orb}(G,S)={1\over 6}\left((R+B)^6 +(R^2+B^2)^3 +2(R^3+B^3)^2 +
2(R^6+B^6)\right).$$  Expanding this gives us
$$R^6+R^5B+3R^4B^2+4R^3B^3+3R^2B^4+RB5+B^6.$$  Thus we have one all-red orbit, one
all-blue orbit, one orbit with five reds and a blue, one with five blues and a red,
three orbits with four reds and two blues as well as three with two reds and four
blues, and four orbits with three reds and three blues.}

\item   Find the generating function (in variables $R$, $B$) for colorings of
the
faces of a cube with two colors (red and blue).\label{polya2}  What does the
generating function tell you about the number of ways to color the cube (up to
spatial movement) with various combinations of the two colors.
\solution{
We want to think of the rotation group of the cube acting on the faces of
the cube, in order to see what kinds of colorings are left fixed.  For this purpose
we note that each element of the rotation group gives us a permutation of the faces
of the cube, and if two faces are in the same cycle of this permutation, they must
have the same color, but if they are in different cycles, they may get different
colors.  The 90 and 270 degree rotations around an axis through two faces have a
four-cycle and two one-cycles.  The 180 degree rotation around an axis through two
faces has two two-cycles and two one-cycles.  The 120 and 240 degree rotations
around an axis connecting diagonally opposite vertices have two three-cycles.  The
180 degree rotations around an axis connecting two opposite edges have three
two-cycles, and the identity has six one-cycles. A coloring is
fixed by a permutation if and only if it is constant on (i.e. assigns the same color
to all elements of) each cycle of the permutation.  The picture enumerator for
(constant) colorings of an
$i$-cycle is
$R^i+B^i$.  Thus, using Problem \ref{PictureEnumeratorforFunctions}, if $\sigma$ is a
90 or 270 degree rotation, its fixed point enumerator is $(R^4+B^4)(R+B)^2$, if it
is a 180 degree rotation around an axis connecting two opposite faces, its fixed
point enumerator is
$(R^2+B^2)^2(R+B)^2$, if it is a 180 degree rotation around an axis connecting two
opposite edges, its fixed point enumerator is $(R^2+B^2)^3$,  if it is a 120 or 240
degree rotation its fixed point enumerator is
$(R^3+B^3)^2$, and if it is the identity, then its fixed point enumerator is
$(R+B)^6$.  Therefore the generating function is
\begin{eqnarray*}&&{1\over 24}\big((R+B)^6 + 8(R^3+B^3)^2 + 3(R^2+B^2)^2(R+B)^2 +\\
&&6
(R^2+B^2)^3 + 6 (R^4+B^4)(R+B)^2\big),\end{eqnarray*} which expands to
$$RB^5+2R^4B^2+R^5B+2R^3B^3+2R^2B^4+R^6+B^6.$$  There is one way to color the
cube all red or all blue, one way to color it with exactly five red or exactly five
blue faces, there are two ways to color it with exactly four red or four blue faces
and two ways to color it with exactly three red (and three blue) faces.}



\ep  

\subsection{The P\'olya-Redfield Theorem}

P\'olya's (and Redfield's) famed enumeration theorem deals with situations such as
those in Problems
\ref{polya1} and \ref{polya2} in which we want a generating function for the
set
of all functions from a set $S$ to a set $T$ on which a picture function is
defined, and the picture of a function is the product of the pictures of its
multiset of values.  The point of the next series of problems is to analyze the
solution to Problems \ref{polya1} and \ref{polya2} in order to see what P\'olya and
Redfield saw (though they didn't see it in this notation or using this
terminology).

\bp

\iteme   In Problem \ref{polya1} we have four kinds of group elements: the
identity
(which fixes every coloring), the rotations through 60 or 300 degrees, the
rotations through 120 and 240 degrees, and the rotation through 180 degrees. 
The fixed point enumerator for the rotation group acting on the functions is by
definition the sum of the fixed point enumerators of colorings fixed by the
identity, of colorings fixed by 60 or 300 degree rotations, of colorings fixed by
120 or 240 degree rotations, and of
colorings fixed by the 180 degree rotation.   Write down each of these
enumerators (one for each kind of permutation) individually and factor each one
(over the integers) as completely as you can.
\label{polya3}
\solution{In the solution to Problem \ref{polya1} we wrote: ``The picture enumerator
of the two constant functions is
$R^6+B^6$. If the vertices are numbered one through six clockwise, then
$\rho^2$ is $(1\ 3\ 5)(2\ 4\ 6)$. Thus for a function $f$ to be fixed by $\rho^2$,
$f(1)=f(3)=f(5)$ and $f(2)=f(4)=f(6)$.   Thus the
color of vertex 1 is repeated on vertex 3 and 5, and the color of vertex 2 is
repeated on vertices 4 and 6.  Therefore the picture enumerator for functions fixed
by
$\rho^2$ and $\rho^4$ is $R^3R^3 +R^3B^3+B^3R^3 +B^3B^3=(R^3+B^3)^2$.  We can also
think of this as the picture enumerator for functions defined on the cycles of the
permutation.  There are two possible pictures of a colored cycle, $R^3$ and $B^3$,
so the picture enumerator for one cycle is $R^3+B^3$.  Thus the picture of assigning
red to a cycle is $R^3$ and the picture of assigning blue to a cycle is $B^3$.  By
Problem
\ref{PictureEnumeratorforFunctions}, the picture enumerator for functions defined on
the two cycles is then $(R^3+B^3)^2$. If a function
$f$ is fixed by
$\rho^3=(1\ 4)(2\ 5)(3\ 6)$, then $f(1)=f(4)$, $f(2)=f(5)$, and $f(3)=f(6)$, so $f$
is determined by the three values $f(1)$, $f(2)$, and $f(3)$.   The picture
enumerator for these functions is, by Problem
\ref{PictureEnumeratorforFunctions} applied to colorings of the cycles,
$(R^2+B^2)^3$.''  In factored form, these enumerators are $R^6+B^6$, $(R+B)^6$,
$(R^3+B^3)^2$,  and
$R^2+B^2)^3$.}

\iteme  In Problem \ref{polya2} we have five different kinds of group elements,
and
the fixed point enumerator is the sum of the fixed point enumerators of each of
these kinds of group elements.  For each kind of element, write down the fixed
point enumerator for the elements of that kind.  Factor the enumerators as
completely as you can.\label{polya4}
\solution{ In the solution to Problem \ref{polya2}, we wrote ``Thus, using Problem
\ref{PictureEnumeratorforFunctions}, if $\sigma$ is a 90 or 270 degree rotation, its
fixed point enumerator is $(R^4+B^4)(R+B)^2$, if it is a 180 degree rotation around
an axis connecting two opposite faces, its fixed point enumerator is
$(R^2+B^2)^2(R+B)^2$, if it is a 180 degree rotation around an axis connecting two
opposite edges, its fixed point enumerator is $(R^2+B^2)^3$,  if it is a 120 or 240
degree rotation its fixed point enumerator is
$(R^3+B^3)^2$, and if it is the identity, then its fixed point enumerator is
$(R+B)^6$.''  We just wrote the enumerators out in factored form.}





\iteme  In Problem \ref{polya3}, each
``kind" of group element
has a ``kind" of cycle structure.  For example, a rotation through 180 degrees has three
cycles of size two. What kind of cycle structure does a rotation through 60 or 300
degrees have?  What kind of cycle structure does a rotation through 120 or 240 degrees
have? Discuss the relationship between the cycle structures and the factored 
enumerators of fixed points of the permutations in
 Problem \ref{polya3}.\label{polya3.5}
\solution{A rotation through 60 or 300 degrees is a five-cycle; a rotation through
120 or 240 degrees is a product of two three-cycles.  The cycle structure determines
the factored enumerator; a cycle of size $i$ gives a factor of $(R^i+B^i)$.  That is
because a coloring fixed by a group element has to be constant on the cycles of that
group element.  If a cycle has size $i$, it contributes a summand of $P^i$ to the
picture enumerator for colorings of that cycle for each picture $P$ of a possible
value of the function.  Problem
\ref{PictureEnumeratorforFunctions} tells us to multiply these individual picture
enumerators together.}

\ep


Recall that we said that a group of permutations acts on a set if, for each
member
$\sigma$ of $G$ there is a bijection $\beta_{\sigma}$ of $S$ such
that
$$\beta_{\sigma\circ\varphi} = \beta_{\sigma}\circ\beta_{\varphi}$$ for
every member $\sigma$ and $\varphi$ of $G$.  Since $\beta_{\sigma}$ is a bijection of $S$
to itself, it is in fact a permutation of $S$.  Thus $\beta_{\sigma}$ has a cycle
structure (that is, it is a product of disjoint cycles) as a permutation of $S$ (in
addition to whatever its cycle structure is in the original permutation group $G$).

\bp

\iteme  In Problem \ref{polya4}, each
``kind" of group element
has a ``kind" of cycle structure in the action of the rotation group of the cube on the
faces of the cube.  For example, a rotation of the cube through 180 degrees around a
vertical axis through the centers of the top and bottom faces has two cycles of size two
and two cycles of size one. How many such rotations does the group have? What are the
other ``kinds" of group elements, and what are their cycle structures? Discuss the
relationship between the cycle structure and the factored enumerator in
 Problem \ref{polya4}.
\solution{We effectively answered this question in our solution to Problem \ref{polya2}.
In particular, there are three rotations of 180 degrees through the centers of opposite
faces.  The other kinds of group elements are as follows.
\begin{itemize}
\item the 90 and 270 degree rotations around an axis
through two faces, of which we have six.  Their cycle structure consists of a four-cycle
and two one-cycles.
\item the 120 and 240 degree rotations around an axis connecting two diagonally opposite
vertices.  Their cycle structure consists of two three-cycles.  We have eight of these.
\item the 180 degree rotations around an axis connecting two opposite edges; their cycle
structure consists of three two-cycles.  We have six of these.
\item the identity, whose cycle structure is six one-cycles.  
\end{itemize}  As we said in the solution to Problem \ref{polya3.5} ``The cycle structure
determines the factored enumerator; a cycle of size $i$ gives a factor of $(R^i+B^i)$. 
That is because a coloring fixed by a group element has to be constant on the cycles
of that group element.  If a cycle has size $i$, it contributes a summand of $P^i$
to the picture enumerator for colorings of that cycle for each picture $P$ of a
possible value of the function.  Problem
\ref{PictureEnumeratorforFunctions} tells us to multiply these individual picture
enumerators together.''}

\iteme   The usual way of describing the P\'olya-Redfield enumeration theorem
involves the ``cycle indicator'' or ``cycle index'' of a group acting on a set. 
Suppose we have a group $G$  acting on a finite set $S$.  Since each group element
$\sigma$ gives us a permutation $\beta_{\sigma}$ of $S$, as such it has a
decomposition into disjoint cycles as a permutation of $S$.  Suppose $\sigma$ has
$c_1$ cycles of size 1, $c_2$ cycles of size 2, ...,
$c_n$ cycles of size $n$.  Then the {\it cycle monomial} of $\sigma$ is 
$$z(\sigma) = z_1^{c_1}z_2^{c_2}\cdots z_n^{c_n}.$$  The {\bf cycle indicator} or
{\bf cycle index}\index{cycle index} of $G$ acting on $S$ is
$$Z(G,S) = {1\over |G|}\sum_{\sigma: \sigma \in G} z(\sigma).$$
What is the cycle index for the group $D_6$ acting on the vertices of a hexagon?  What is
the cycle index for the group of rotations of the cube acting on the faces of the cube?
\solution{For $D_6$, we get\begin{eqnarray*}{1\over 12}\left(z_1^6
+2z_6+2z_3^2+z_2^3+3z_2^3+3z_2^2z_1^2\right) &=&\\
{1\over 12}\left(z_1^6
+2z_6+2z_3^2+4z_2^3+3z_1^2z_2^2\right)&&
\end{eqnarray*}
For the rotation group of the cube, we get
\begin{eqnarray*}{1\over 24}\left(3z_1^2z_2^2+ 6z_1^2z_4+8z_3^2 +6z_3^2+z_1^6\right)&=&\\
{1\over 24}\left(3z_1^2z_2^2+ 6z_1^2z_4 +14z_3^2+z_1^6\right)&&
\end{eqnarray*}} 

\itemei
How can you compute the Orbit Enumerator of
$G$ acting on functions from
$S$ to a
finite set $T$ from the cycle index of $G$ acting on $S$? (Use $P(t)$ as the notation
for the picture of an element $t$ of $T$.)  State and prove the relevant theorem!  This is
P\'olya's and Redfield's famous enumeration theorem.\index{P\'olya-Redfield Theorem}
\solution{To compute the orbit enumerator of $G$ acting on functions from $S$ to a finite
set $T$, we substitute $\sum_{t:t\in T}P(t)^i$ for $z_i$ in the cycle index for $G$
acting on $S$.  (This is the P\'olya-Redfield Theorem.)  By the Orbit-Fixed  Point
Theorem, we need to sum the fixed point enumerators of the permutations in $G$. 
However if $\sigma$ fixes $f$, then
$f$ is constant on the cycles of
$\sigma$. Thus we may think of our function $f$ as a function defined on the cycles of
$\sigma$. If an
$i$-cycle is sent to the member
$t$ of
$T$, then the picture of the restriction of our function to that cycle is $P(t)^i$.  The
picture enumerator for all functions defined and constant on that cycle is then
$\sum_{t:t\in T} P(t)^i$.  By Problem
\ref{PictureEnumeratorforFunctions} the picture enumerator for all functions constant on
the cycles of $\sigma$, then is the result of substituting $\sum_{t:t\in T}P(t)^i$ for
$z_i$ in $z(\sigma)$. This proves the P\'olya-Redfield Theorem.}  

\itemi   Suppose we make a necklace by stringing 12 pieces of brightly colored
plastic tubing onto a string and fastening the ends of the string together.  We
have ample supplies blue, green, red, and yellow tubing available.  Give a
generating function in which the coefficient of $B^iG^jR^kY^h$ is the number of
necklaces we can make with $i$ blues, $j$ greens, $k$ reds, and $h$ yellows.  How many
terms would this generating function have if you expanded it in terms of powers of $B$,
$G$, $R$, and $Y$?  Does it make sense to do this expansion?  How many of these
necklaces have 3 blues, 3 greens, 2 reds, and 4 yellows? 
\solution{
We are asking for a generating function for the orbits of a colored 12-gon under
the action of $D_{12}$.  To apply the P\'olya-Redfield theorem we need the cycle
index for
$D_{12}$.  If $\rho$ is a 30 degree rotation, then $\rho$, $\rho^5$, $\rho^7$ and
$\rho^{11}$ are 12-cycles.  The elements $\rho^2$ and $\rho^{10}$ are products of two
six-cycles.  The elements $\rho^3$ and $\rho^9$ are products of three four-cycles.  The
elements $\rho^4$ and $\rho^8$ are products of four three-cycles.  The element $\rho^6$ is
a product of six two-cycles.  The element $\iota=\rho^0$ is a product of 12 one-cycles. 
There are six flips around axes through opposite vertices; each is a product of five
two-cycles and two one-cycles.  There are six flips around axes perpendicular to two
opposite sides; each is a product of six two-cycles.  Summarizing this in the cycle index,
we write
\begin{eqnarray*}
Z(G,S) &=&{1\over 24}\left( z_1^{12}+z_2^6+ 2z_3^4+ 2z_4^3 + 2z_6^2 +4z_{12} +
6z_2^5z_1^2 + 6z_2^6\right)\\ &=&{1\over24}\left( z_1^{12}+7z_2^6+ 2z_3^4+ 2z_4^3 +
2z_6^2 +4z_{12} + 6z_2^5z_1^2\right).
\end{eqnarray*}
When we substitute $B^i+G^i+R^i+Y^i$ for $z_i$ and expand, we would get get $12^4$
terms, one for each possible term $B^iG^jR^kY^h$.  Thus it does not make sense to
expand the polynomial.  The unexpanded form is

${1\over 24}\Big((B+G+R+Y)^{12}+7(B^2+G^2+R^2+Y^2)^6+
2(B^3+G^3+R^3+Y^3)^4+\\
2(B^4+G^4+R^4+Y^4)^3 +  2(B^6+G^6+R^6+Y^6)^2
+\\4(B^{12}+G^{12}+R^{12}+Y^{12}) +  6((B^2+G^2+R^2+Y^2)^5)(B+G+R+Y)^2\Big)
$

We can compute the
coefficient of
$B^3G^3R^2Y^4$ by computing the contribution of each term of the sum to the coefficient.
We get
\begin{eqnarray*}{1\over24}\left({12\choose 3,3,2,4}+ 6{5\choose
1,1,1,2}{2\choose1,1}\right)&=&{12!\over 24\cdot3!3!2!4!}+{6\cdot5!2!\over
24\cdot2!}\\&=&{11!\over2\cdot3!3!2!4!}+5
\cdot3\cdot2\\ 
&=&11\cdot5\cdot3\cdot2\cdot7\cdot5+30\\
&=&11,550+30=11,580\end{eqnarray*}
}

\itemei What should we substitute for the pictures of each of the elements of $T$ in
the orbit enumerator of $G$ acting on the set of functions from $S$ to $T$ in order to
compute the total number of orbits of $G$ acting on the set of functions?  What should we
substitute into the variables in the cycle index of a group $G$ acting on a set $S$ in
order to compute the total number of orbits of $G$ acting on the functions from $S$ to a
set $T$?
  Find the
number of ways to color the faces of a cube with four colors. 
\solution{Substitute the number one for each picture in the picture enumerator. 
Substitute
$|T|$ for each variable in the cycle index. The cycle index for the rotation group of
the cube acting on the faces is 
$${1\over 24}\left(3z_1^2z_2^2+ 6z_1^2z_4 +14z_3^2+z_1^6\right.)$$  Substituting 4 for
each variable gives us ${1\over 24}\left(3\cdot4^4+6\cdot4^3+14\cdot 4^2+4^6\right)\\=228$.
}

\itemi  We have red, green, and blue sticks all of the same length, with a dozen
sticks of each color.  We are going to make the skeleton of a cube by taking eight
identical lumps of modeling clay and pushing three sticks into each lump so that the
lumps become the vertices of the cube.  (Clearly we won't need all the sticks!)  In
how many different ways could we make our cube?  How many cubes have four edges of
each color?  How many have two red, four green, and six blue
edges?\label{coloredsticks}
%\solution{
For this problem we are interested in the action of the rotation group of the
cube on the edges. Now we think of the group elements as permutations of the edges and
analyze their cycle structure.
\begin{itemize}
\item The identity is a product of 12 one-cycles.
\item A 90 or 270 degree rotation around an axis perpendicular to two opposite faces is a
product of three four-cycles.
\item A 180 degree rotation around an axis perpendicular to two opposite faces is a
product of six two-cycles.
\item A 180 degree rotation around an axis perpendicular to two opposite edges is a
product of five two-cycles and two one-cycles.
\item A 120 degree rotation around an axis connecting two diagonally opposite vertices is
a product of four three-cycles.
\end{itemize}
Thus the cycle index is 
$${1\over 24}\left(z_1^{12}+6z_4^3+3z_2^6+6z_2^5z_1^2 + 8z_3^4\right).$$  We substitute
the number three for each of the variables to get
$${1\over24}\left(3^{12}+6\cdot 3^3+3\cdot3^6+6\cdot 3^7 +8\cdot3^4\right)=22815$$ ways to
make the cube.  To compute the number of ways with four sticks of each color, we
need to apply the P\'olya-Redfield theorem.  Substituting $R^i+B^i+G^i$ for $z_i$ in
the cycle index gives us 

${1\over
24}\Big((R+B+G)^{12}+6((R^4+B^4+G^4)^3+3(R^2+B^2+G^2)^6+\\6(R^2+B^2+B^2)^5(R+B+G)^2 
+8(R^3+B^3+G^3)^4\Big)$. 

The coefficient of $R^4B^4G^4$ is
\begin{eqnarray*}{1\over
24}\left({12\choose4,4,4}+6{3\choose1,1,1}+3{6\choose2,2,2}+
6{3\choose1}{5\choose2,2,1}{2\choose2,0,0}
\right)&=&\\
{1\over
24}\left({12!\over4!4!4!}+6\cdot3!+3{6!\over2!2!2!}+6\cdot3{5!\over2!2!1!}\right)=1479&&
\end{eqnarray*}

The coefficient of $R^2B^4G^6$ is 
\begin{eqnarray*}{1\over 24}\left({12\choose2,4,6}
+3{6\choose1,2,3}+6\left({5\choose0,2,3}+{5\choose
1,1,3}+{5\choose1,2,2}\right)\right)&=&\\
{1\over24}\left({12!\over2!4!6!}+3{6!\over1!2!3!}+6\left({5!\over2!3!}+{5!\over1!1!3!}
+{5!\over1!2!2!}\right)\right)=600.
\end{eqnarray*}
%}

\itemi  How many cubes can we make in Problem \ref{coloredsticks} if
the lumps of modelling clay can be any of four colors?
\solution{We can first consider all ways of coloring the vertices of the cube with the
four colors; once those vertices are in place, the number of ways to put the sticks in is
the result of Problem \ref{coloredsticks}, so by the product principle then number of
ways to choose the colors of the vertices and edges is the product of the number of
ways to choose each.  Thus we just need to consider the action of the rotation group
of the cube on the vertices:
\begin{itemize}
\item The identity is the product of eight one-cycles.
\item A 90 or 270 degree rotation around an axis perpendicular to two opposite faces is a
product of two four-cycles.
\item A 180 degree rotation around an axis perpendicular to two opposite faces is a
product of four two-cycles.
\item A 180 degree rotation around an axis joining two opposite edges is a product of four
two-cycles.
\item A 120 or 240 degree rotation around an axis through two diagonally opposite vertices
is a product of two three-cycles and two one-cycles.
\end{itemize}
Thus the cycle index for this action is
$${1\over 24}\left(z_1^8+6z_4^2 + 9z_2^4 + 8z_3^2z_1^2\right).$$
Substituting 4 for each variable gives us 
$${1\over 24}\left(4^8+6\cdot4^2 +9\cdot 4^4 + 8\cdot4^2\cdot4^2\right)=2916.$$  Thus we
have  $22815\cdot2916=66,528,540$ possible colored cubes.}

\begin{figure}[htb]\caption{A possible
computer network.}\label{HexNet}\smallskip
\begin{center}\mbox{\psfig{figure=NumberedHexagonalNetwork.eps
}}
\end{center}  
\end{figure} 

\itemi In Figure \ref{HexNet} we see a graph with six vertices.  Suppose we
have three different kinds of computers that can be placed at the six vertices
of the graph to form a network.  In how many different ways may the computers
be placed? (Two graphs are not different if we
can redraw one of the graphs so that it is identical to
the other one.)  This is equivalent to permuting the vertices in some way so that when we
apply the permutation to the endpoints of the edges to get a new edge set, the new edge
set is equal to the old one.  Such a permutation is called an {\em
automorphism}\index{automorphism (of a graph)} of the graph.  Then two computer
placements are the same if there is an automorphism of the graph that carries one to the
other.  
\solution{
The computer placements are functions from the vertices of the graph to the set
of three kinds of computers, say $\{A,B,C\}$.  Thus we are asking for the number of
orbits of the automorphism group of the graph on functions from the vertices of $\{A, B, 
C\}$.  To find this number we need to compute the cycle index of the automorphism group. 
An automorphism will send vertices 1, 2, and 3 to three vertices that are mutually
connected; i.e. a triangle.  Further, each of vertices 4, 5, and 6 is adjacent to exactly
two of vertices of 1, 2, and 3.  In fact, for any of the eight triangles in the graph,
each vertex not in the triangle is adjacent to exactly two vertices of the triangle (and a
different two for each vertex).  Therefore we can send the vertices 1, 2, and 3, to
any of the eight triangles in any of six orders, and this completely determines an
automorphism.  Thus there are $6\cdot8=48$ elements in the group of automorphisms. 
The dihedral group
$D_6$ is a subgroup of the group of automorphisms.  The two-cycles $(1\ 4)$, $(2\ 5)$, and
$(3\ 6)$ are also in the group of automorphisms.  (For example, 1 is adjacent to
everything but 4, and 4 is adjacent to everything but 1, so interchanging them leaves us
with exactly the same edges.)  We will write out $D_6$ in disjoint cycle notation, and
then the coset $(1\ 4)D_6$, and from those we will be able to get the cycle index of the
automorphism group acting on the vertices.  In cycle notation, $D_6$ is
$$\begin{array}{cccc}
(1\ 2\ 3\ 4\ 5\ 6)&(1\ 3\ 5)(2\ 4\ 6)&(1\ 4)(2\ 5)(3 \ 6)& (1\ 5\ 3)(2\ 6\ 4)\\(1\ 6\ 5\ 4\
3\ 2)&(1)(2)(3)(4)(5)(6)&(1)(4)(2\ 6)(3\ 5)&(2)(5)(1\ 3)(4\ 6)\\
(3)(6)(1\ 5)(2\ 4)&(1\ 2)(3\ 6)(4\ 5)&(1\ 6)(2\ 5)(3\ 4)&(1\ 4)(2\ 3)(5\ 6)
\end{array}$$
and $(1\ 4)D_6$ is
$$\begin{array}{cccc}
(1\ 2\ 3) (4\ 5\ 6)&(1\ 3\ 5\ 4\ 6\ 2)&(1) (4)(2\ 5)(3 \ 6)& (1\ 5\ 3\ 4\ 2\ 6)\\(1\ 6\
5)(2\ 4\ 3)&(1\ 4)(2)(3)(5)(6)&(1\ 4)(2\ 6)(3\ 5)&(2)(5)(1\ 3\ 4\ 6)\\
(3)(6)(1\ 5\ 4\ 2)&(1\ 2\ 4\ 5)(3\ 6)&(1\ 6\ 4\ 3)(2\ 5)&(1)(4)(2\ 3)(5\ 6).
\end{array}$$
Notice that neither $(2\ 5)$ nor $(3\ 6)$ is in the coset, so our group is the union
$$D_6\cup (1\ 4)D_6 \cup(2\ 5)D_6 \cup (3\ 6)D_6.$$ But by symmetry, the cycle structure of
$(1\ 4)D_6$,
$(2\ 5)D_6$ and
$(3\ 6)D_6$ will be the same, so the  cycle index for our group is.
$${1 \over 48} \left(Z^6_1 + 8Z^2_3 
+8z_6^1 +7z_2^3+9z_1^2z_2^2 +3z_1^4z_2+
6z_2z_4+6z_1^2z_4\right).$$
Since we have three kinds of computers, we substitute 3 for each variable to get
$${1 \over 48} \left(3^6 + 8\cdot3^2 
+8\cdot3^1 +7\cdot3^3+9\cdot3^4 +3\cdot3^5+
6\cdot3^2+6\cdot3^3\right)=56.$$}

\itemi Two simple graphs on the set $[n]= \{1,2,\ldots, n\}$ with edge sets $E$ and
$E'$ (which we think of a sets of two-element sets for this problem) are said to be
{\em isomorphic} if there is a permutation $\sigma$ of 
$[n]$ which, in its action of two-element sets, carries $E$ to $E'$.  We say two graphs
are different if they are not isomorphic.  Thus the number of different graphs is the
number of orbits of the set of all two-element subsets of $[n]$ under the action of the
group $S_n$.  We can represent an edge set by its characteristic function (as in
problem \ref{charfunction}).  That is we define $$\Chi_E(\{u,v\}) = \left\{
\begin{array}{ll}
1 & \mbox{if $\{u,v\}\in E$}\\
0 & \mbox{otherwise.}\end{array}\right. $$  Thus we can think of the set of graphs as a
set of functions defined on the set of all two-element subsets of $[n]$.  The number of
different graphs with vertex set $[n]$ is thus the number of orbits of this set of
characteristic functions under the action of the symmetric group $S_n$ on the set of
two-element subsets of $[n]$.  Use this to find the number of different graphs on five
vertices. 
\solution{For this problem we need the cycle index for the action of the symmetric group
on five letters acting on two-element subsets of those five letters.  Each way of
partitioning the number five describes the cycle structure of an element of $S_5$ acting
on $[5]$.  The cycle structure of a permutation $\sigma$ on the two-element subsets of
$[5]$ will be determined by its cycle structure on $[5]$.  The partitions of five and
the cycle structures they give on two-element subsets are:
\begin{itemize}
\item $(1,1,1,1,1) =1^5$ is the cycle structure of the identity; acting on two-element
subsets the identity is a product of ${5\choose2}=10$ one-cycles.  Of course there is only
one identity permutation.
\item A permutation with cycle structure $2^1 1^3$ will fix the two element set in
the two cycle and each pair of the three elements outside for a total of four
one-cycles; each pair of an element in the two cycle and an element not will be in a
two cycle.  There are 6 such pairs and thus three two-cycles, so such a permutation
has 3 two-cycles and four one-cycles.  There are ${5\choose 2}=10$ such permutations.
\item A permutation with cycle structure $2^2 1^1$ will have a one-cycle of two-sets
for each two cycle, any other two-set will be in a  two cycle.  There are $4+2+2=8$
such two-sets and thus four two-cycles, so such a permutation has four two cycles
and two one cycles.  There are ${5\choose2}{3\choose2}/2 =15$ such permutations.
\item A permutation with cycle structure $3^1 1^2$ will have one cycle of size 1 on
two-sets from the two one-element cycles of the original action; each other two-element
subset will be in a three cycle.  There are ${3\choose 2}+3\cdot2=9$ such pairs, and so
there are three three-cycles of two-element subsets.  Thus such a permutation has one
one-cycle and three three-cycles in its action of two-sets. There are $2{5\choose 3}=20$
such permutations.
\item A permutation with cycle structure $3^1 2^1$ will have one one-cycle on two-sets from
the two-cycle of the original action, it will have $3\choose 2$  two-sets in three
cycles,  and will have six two-sets in
six-cycles, and so will have just one six-cycle of two-sets.  Thus such a
permutation has one one-cycle, one three-cycle and one six-cycle when it acts on
two-sets.  There are $2{5\choose 3}=20$ such permutations.
\item A permutation with cycle structure $4^1 1^1$ has a four-cycle of two-subsets
and one two-cycle of two-subsets from its own four-cycle and another four cycle of
pairs of members of the four-cycle and one cycle. thus it has two four-cycles and
two one-cycles There are
$6{5\choose4}=30$ such permutations.
\item A permutation with cycle structure $5^1$ will have two five-cycles in its action on
two-sets.  There are $4!=24$ such permutations.
\end{itemize}
Thus for the group $S_5$ acting on pairs from $[5]$, the cycle index is
$${1\over120}\left(z_1^{10}+10z_2^3z_1^4+15z_2^4z_2+20z_3^3z_1+20z_6z_3z_1+
30z_4^2z_2+ 24z_5^2\right).$$  Substituting 2 for each variable gives us that there
are 
$${1\over120}\left(2^{10}+10\cdot2^7+15\cdot2^5+20\cdot2^4+20\cdot2^3+30\cdot2^3+
24\cdot2^2\right)=30$$ graphs on five vertices.}
\ep


\section{Supplementary Problems}
\begin{enumerate}
\item Show that a function from $S$ to $T$ has an inverse (defined on $T$)
if and only if it is a bijection.

\item How many elements are in the dihedral group $D_3$?  The symmetric
group $S_3$?  What can you conclude about $D_3$ and $S_3$?

\item A tetrahedron is a thee dimensional geometric figure with four
vertices, six edges, and four triangular faces.  Suppose we start with a
tetrahedron in space and consider the set of all permutations of the
vertices of the tetrahedron that correspond to moving the tetrahedron in
space and returning it to its original location, perhaps with the
vertices in different places.  Explain why these permutations form a
group.  What is the size of this group?  Write down in two-row notation a
permutation that is {\em not} in this group.

\item Find a three-element subgroup of the group $S_3$.  Can you find a
different three-element subgroup of $S_3$?

\item  Prove true or demonstrate false with a counterexample:  ``In a
permutation group, $(\sigma\varphi)^n = \sigma^n\varphi^n$.''


\item Describe a permutation group with 60 elements.

\item If a group $G$ acts on a set $S$, and if $\sigma(x) =y$, is there
anything interesting we can say about the subgroups ${\rm Fix}(x)$ and
${\rm Fix}(y)$?

\item Find the number of ways to color the faces of a tetrahedron with
two colors.

\item Find the number of ways to color the faces of a tetrahedron with
four colors so that each color is used.

\item Find the cycle index of the group of spatial symmetries of the tetrahedron
acting on the vertices.  Find the cycle index for the same group acting on the
faces.

\item Find the generating function for the number of ways to color the faces of
the tetrahedron with red, blue, green and yellow.

\itemi Find the generating function for the number of ways to color the faces of a
cube with four colors so that all four colors are used.

\itemi How many different graphs are there on six vertices with seven edges?
\end{enumerate}